% !TeX spellcheck = sl_SI
% vim: set spell spelllang=sl:
% za preverjanje črkovanja, če se uporablja Texstudio ali vim
\documentclass[12pt,a4paper,twoside]{article}
\usepackage[utf8]{inputenc}  % pravilno razpoznavanje unicode znakov

% NASLEDNJE UKAZE USTREZNO POPRAVI
\newcommand{\program}{Matematika} % ime studijskega programa
\newcommand{\imeavtorja}{Katarina Černe} % ime avtorja
\newcommand{\imementorja}{prof.~dr.~} % akademski naziv in ime mentorja, uporabi poln naziv, prof.~dr.~, doc.~dr., ali izr.~prof.~dr.
\newcommand{\imesomentorja}{} % akademski naziv in ime somentorja, če ga imate
\newcommand{\naslovdela}{Naslov vašega dela}
\newcommand{\letnica}{2020} % letnica magistriranja
\newcommand{\opis}{}  % Opis dela v eni povedi. Ne sme vsebovati matematičnih simbolov v $ $.
\newcommand{\kljucnebesede}{\sep } % ključne besede, ločene z \sep, da se PDF metapodatki prav procesirajo
\newcommand{\keywords}{\sep } % ključne besede v angleščini
\newcommand{\organization}{Univerza v Ljubljani, Fakulteta za matematiko in fiziko} % fakulteta
\newcommand{\literatura}{literatura}  % pot do datoteke z literaturo (brez .bib končnice)
\newcommand{\sep}{, }  % separator med ključnimi besedami v besedilu
% KONEC PODATKOV

\usepackage{bibentry}         % za navajanje literature v programu dela s celim imenom
\nobibliography{\literatura}
\newcommand{\plancite}[1]{\item[\cite{#1}] \bibentry{#1}} % citiranje v programu dela

\usepackage{filecontents}  % za pisanje datoteke s PDF metapodatki
\usepackage{silence} \WarningFilter{latex}{Overwriting file}  % odstrani annoying warning o obstoju datoteke
% datoteka s PDF metapodatki, zgenerira se kot magisterij.xmpdata
\begin{filecontents*}{\jobname.xmpdata}
  \Title{\naslovdela}
  \Author{\imeavtorja}
  \Keywords{\kljucnebesede}
  \Subject{\opis}
  \Org{\organization}
\end{filecontents*}

\usepackage[a-1b]{pdfx}  % zgenerira PDF v tem PDF/A-1b formatu, kot zahteva knjižnica
\hypersetup{bookmarksopen, bookmarksdepth=3, colorlinks=true,
  linkcolor=black, anchorcolor=black, citecolor=black, filecolor=black,
  menucolor=black, runcolor=black, urlcolor=black, pdfencoding=auto,
  breaklinks=true, psdextra}

\usepackage[slovene]{babel}  % slovenščina
\usepackage[T1]{fontenc}     % naprednejše kodiranje fonta
\usepackage{amsmath,amssymb,amsfonts,amsthm} % matematični paketi
%\usepackage[dvipsnames,usenames]{color} % barve
\usepackage{graphicx}     % za slike
\usepackage{emptypage}    % prazne strani so neoštevilčene, ampak so štete
\usepackage{units}        % fizikalne enote kot \unit[12]{kg} s polovico nedeljivega presledka, glej primer v kodi
\usepackage{makeidx}      % za stvarno kazalo, lahko zakomentiraš, če ne rabiš
\makeindex                % za stvarno kazalo, lahko zakomentiraš, če ne rabiš
% oblika strani
\usepackage[
  top=3cm,
  bottom=3cm,
  inner=3.5cm,      % margini za dvostransko tiskanje
  outer=2.5cm,
  footskip=40pt     % pozicija številke strani
]{geometry}

% VEČ ZANIMIVIH PAKETOV
% \usepackage{array}      % več možnosti za tabele
% \usepackage[list=true,listformat=simple]{subcaption}  % več kot ena slika na figure, omogoči slika 1a, slika 1b
% \usepackage[all]{xy}    % diagrami
% \usepackage{doi}        % za clickable DOI entrye v bibliografiji
% \usepackage{enumerate}     % več možnosti za sezname

% Za barvanje source kode
% \usepackage{minted}
% \renewcommand\listingscaption{Program}

% Za pisanje psevdokode
% \usepackage{algpseudocode}  % za psevdokodo
% \usepackage{algorithm}
% \floatname{algorithm}{Algoritem}
% \renewcommand{\listalgorithmname}{Kazalo algoritmov}

% DRUGI TVOJI PAKETI:
% tukaj
\usepackage[utf8]{inputenc}
\usepackage{lmodern}
\usepackage{eurosym}
\usepackage{hyperref}
\usepackage{subfigure}
\usepackage{xcolor}
\usepackage{tcolorbox}
\usepackage{enumitem}

\setlength{\overfullrule}{50pt} % označi predlogo vrstico
\pagestyle{plain}               % samo številka strani na dnu, nobene glave / noge

% ukazi za matematična okolja
\theoremstyle{definition} % tekst napisan pokončno
\newtheorem{definicija}{Definicija}[section]
\newtheorem{primer}[definicija]{Primer}
\newtheorem{opomba}[definicija]{Opomba}
\newtheorem{aksiom}{Aksiom}

\theoremstyle{plain} % tekst napisan poševno
\newtheorem{lema}[definicija]{Lema}
\newtheorem{izrek}[definicija]{Izrek}
\newtheorem{trditev}[definicija]{Trditev}
\newtheorem{posledica}[definicija]{Posledica}

\numberwithin{equation}{section}  % števec za enačbe zgleda kot (2.7) in se resetira v vsakem poglavju

% Matematični ukazi
\newcommand{\R}{\mathbb R}
\newcommand{\N}{\mathbb N}
\newcommand{\Z}{\mathbb Z}
\renewcommand{\C}{\mathbb C}
\newcommand{\Q}{\mathbb Q}

\newcommand{\todo}[1]{{\color{red}{#1}}}
\newcommand{\nevem}[1]{{\color{blue}{#1}}}

% \DeclareMathOperator{\tr}{tr}  % morda potrebuješ operator za sled ali kaj drugega?

% bold matematika znotraj \textbf{ }, tudi v naslovih, kot \omega spodaj
\makeatletter \g@addto@macro\bfseries{\boldmath} \makeatother

% Poimenuj kazalo slik kot ``Kazalo slik'' in ne ``Slike''
\addto\captionsslovene{
  \renewcommand{\listfigurename}{Kazalo slik}%
}

% če želiš, da se poglavja začnejo na lihih straneh zgoraj
% \let\oldsection\section
% \def\section{\cleardoublepage\oldsection}

%%%%%%%%%%%%%%%%%%%%%%%%%%%%%%%%%%%%%%%%%%
%%%%%%           DOCUMENT           %%%%%%
%%%%%%%%%%%%%%%%%%%%%%%%%%%%%%%%%%%%%%%%%%

\begin{document}

\pagenumbering{roman} % začnemo z rimskimi številkami
\thispagestyle{empty} % ampak na prvi strani ni številke

\noindent{\large
UNIVERZA V LJUBLJANI\\[1mm]
FAKULTETA ZA MATEMATIKO IN FIZIKO\\[5mm]
\program\ -- 2.~stopnja}
% ustrezno dopolni za IŠRM
\vfill

\begin{center}
  \large
  \imeavtorja\\[3mm]
  \Large
  \textbf{\MakeUppercase{\naslovdela}}\\[10mm]
  \large
  Magistrsko delo \\[1cm]
  Mentor: \imementorja \\[2mm] % ustrezno popravi spol
%   Somentor: \imesomentorja   % dodaj, če potrebno
\end{center}
\vfill

\noindent{\large Ljubljana, \letnica}

\cleardoublepage

%% IZJAVA O AVTORSTVU
%\pdfbookmark[1]{Izjava o avtorstvu}{izjava} % bookmark v PDF, \pdfbookmark[nivo]{text}{label}
%
%% izjava: po potrebi spremeni v žensko obliko
%\setlength\topsep{0pt}
%\setlength\parskip{0pt}
%\begin{center}
%  \textbf{Univerza v Ljubljani} \\
%  \textbf{Fakulteta za matematiko in fiziko}
%
%  \vfill
%
%  \underline{Izjava o avtorstvu, istovetnosti tiskane in elektronske verzije magistrskega dela in} \\
%  \underline{objavi osebnih podatkov študenta}
%
%  \vfill
%
%  \setlength\topsep{0pt}
%  \setlength\parskip{0pt}
%  \begin{flushleft}
%    Spodaj podpisani študent \imeavtorja{} avtor magistrskega dela (v nadaljevanju: pisnega
%    zaključnega dela študija) z naslovom:
%  \end{flushleft}
%
%  \vfill
%
%  \textbf{\naslovdela}
%
%  \vfill
%
%  IZJAVLJAM
%\end{center}
%
%\begin{enumerate}[1. ]
%  \item \emph{Obkrožite eno od variant a) ali b)}
%  \begin{enumerate}[a)]
%    \item da sem pisno zaključno delo študija izdelal samostojno;
%    \item da je pisno zaključno delo študija rezultat lastnega dela več kandidatov in izpolnjuje
%      pogoje, ki jih Statut UL določa za skupna zaključna dela študija ter je v zahtevanem deležu
%      rezultat mojega samostojnega dela;
%  \end{enumerate}
%  pod mentorstvom IZPOLNI. % dopiši \imementorja v rodilniku
%%   \\ in somentorstvom IZPOLNI. % dopiši \imesomentorja v rodilniku
%  \item da je tiskana oblika pisnega zaključnega dela študija istovetna elektronski obliki
%    pisnega zaključnega dela študija;
%  \item da sem pridobil vsa potrebna dovoljenja za uporabo podatkov in avtorskih del v pisnem
%    zaključnem delu študija in jih v pisnem zaključnem delu študija jasno označil;
%  \item da sem pri pripravi pisnega zaključnega dela študija ravnal v skladu z etičnimi načeli in,
%    kjer je to potrebno, za raziskavo pridobil soglasje etične komisije;
%  \item da soglašam, da se elektronska oblika pisnega zaključnega dela študija uporabi za preverjanje
%    podobnosti vsebine z drugimi deli s programsko  opremo za preverjanje podobnosti
%    vsebine, ki je povezana s študijskim informacijskim sistemom fakultete;
%  \item da na UL neodplačno, neizključno, prostorsko in časovno neomejeno prenašam pravico shranitve
%    avtorskega dela v elektronski obliki, pravico reproduciranja ter pravico dajanja pisnega
%    zaključnega dela študija na voljo javnosti na svetovnem spletu preko Repozitorija UL;
%  \item da dovoljujem objavo svojih osebnih podatkov, ki so navedeni v pisnem zaključnem delu študija
%    in tej izjavi, skupaj z objavo pisnega zaključnega dela študija.
%\end{enumerate}
%
%\vfill
%
%\noindent
%Kraj:  \hfill   Podpis študenta: \phantom{prostor za podpis}
%
%\vfill
%
%\noindent
%Datum:
%
%\cleardoublepage
%% END IZJAVA O AVTORSTVU

% zahvala
\pdfbookmark[1]{Zahvala}{zahvala} %
\section*{Zahvala}
Neobvezno.
Zahvaljujem se \dots
% end zahvala -- izbriši vse med zahvala in end zahvala, če je ne rabiš

\cleardoublepage

\pdfbookmark[1]{\contentsname}{kazalo-vsebine}
\tableofcontents

% list of figures
% \cleardoublepage
% \pdfbookmark[1]{\listfigurename}{kazalo-slik}
% \listoffigures
% end list of figures

\cleardoublepage

\section*{Program dela}
\addcontentsline{toc}{section}{Program dela} % dodajmo v kazalo
Mentor naj napiše program dela skupaj z osnovno literaturo. Na literaturo se
lahko sklicuje kot~\cite{lebedev2009introduction}, \cite{gurtin1982introduction},
\cite{zienkiewicz2000finite}, \cite{STtemplate}.

\section*{Osnovna literatura}
Literatura mora biti tukaj posebej samostojno navedena (po pomembnosti) in ne
le citirana. V tem razdelku literature ne oštevilčimo po svoje, ampak uporabljamo
okolje itemize in ukaz plancite, saj je celotna literatura oštevilčena na koncu.
\begin{itemize}
  \plancite{lebedev2009introduction}
  \plancite{gurtin1982introduction}
  \plancite{zienkiewicz2000finite}
  \plancite{STtemplate}
\end{itemize}

\vspace{2cm}
\hspace*{\fill} Podpis mentorja: \phantom{prostor za podpis}

% \vspace{2cm}
% \hspace*{\fill} Podpis somentorja: \phantom{prostor za podpis}

\cleardoublepage
\pdfbookmark[1]{Povzetek}{abstract}

\begin{center}
\textbf{\naslovdela} \\[3mm]
\textsc{Povzetek} \\[2mm]
\end{center}
Tukaj napišemo povzetek vsebine. Sem sodi razlaga vsebine in ne opis tega, kako je delo
organizirano.

\vfill
\begin{center}
\textbf{English translation of the title} \\[3mm] % prevod slovenskega naslova dela
\textsc{Abstract}\\[2mm]
\end{center}

An abstract of the work is written here. This includes a short description of
the content and not the structure of your work.

\vfill\noindent
\textbf{Math.~Subj.~Class.~(2010):} oznake kot 74B05, 65N99, na voljo so na naslovu
\url{http://www.ams.org/msc/msc2010.html?t=65Mxx} \\[1mm]
\textbf{Ključne besede:} \kljucnebesede \\[1mm]
\textbf{Keywords:} \keywords

\cleardoublepage

\setcounter{page}{1}    % od sedaj naprej začni zopet z 1
\pagenumbering{arabic}  % in z arabskimi številkami

\section{Uvod}


\section{Geometrijska zveznost} %odloči se še za naslove

\begin{definicija}
  Ploskev pripada razredu $G^n$ oziroma je geometrijsko zvezna z redom $n$, če v okolici vsake njene točke obstaja lokalna regularna parametrizacija razreda $C^n$.
\end{definicija}

\todo{definicija regularne ploskve}
\todo{razloži lokalnost?}

\nevem{Naj bosta $R: \Omega_1 \subseteq \R^2 \rightarrow \R^3$ in 
$S: \Omega_2 \subseteq \R^2 \rightarrow \R^3$ regularni parametrizaciji}

\begin{definicija}
  Naj bosta $R(x,y)$ in $S(u,v)$ regularni $C^n$ parametrizaciji dveh ploskev, ki se stikata v krivulji $C(y)=R(x_0,y)=S(u_0,y)$. %povej da na C je y=v
  Pravimo, da se $R$ in $S$ stikata z $G^n$-zveznostjo vzdolž krivulje $C$, če za vsako točko $b_0=C(y_0)$ obstaja lokalno regularna $C^n$ reparametrizacijska funkcija $f(x,y)=(u(x,y),v(x,y))$, da je $f(x_0,y)=(u_0,y)$ za vsak $y \in I_0$ in da velja
  $$\frac{\partial^{m+k}}{\partial x^m \partial y^k}R\Bigr|_{\substack{(x_0,y)}}=\frac{\partial^{m+k}}{\partial x^m \partial y^k}(S\circ f)\Bigr|_{\substack{(x_0,y)}} \textrm{ za } m+k=1,\ldots,n,$$
  kjer je $I_0$ neka okolica $y_0$. %bolj natančno, kaj je ta okolica
\end{definicija}

Zaradi stikanja ploskev v krivulji $C$ so delni odvodi parametrizacij 
po spremenljivki $y$ vzdolž krivulje $C$ enaki, zato je dovolj, da pri obravnavi 
geometrijske zveznosti dveh ploskev opazujemo le delne odvode po 
spremenljivki $x$.
\todo{ali je to dovolj razloženo? razloži kot v poglavju G1?}
\todo{dodati sliko?}
Te delne odvode imenujemo \todo{??crossboundary derivatives}.

Oglejmo si pogoje za različne stopnje geometrijske zveznosti, ki nam jih ta definicija da. \todo{to še ni dokončno. zelo grdo}
Če so parametrizaciji $R$ in $S$, krivulja $C$ in reparametrizacijska 
funkcija $f$ kot v definiciji \todo{sklic}, 
je za geometrijsko zveznost razreda $G^0$ med njima dovolj pogoj 
$R=S \circ f$ vzdolž $C$. 
\todo{oziroma kar R=S vzdolž C??}
Da imamo na stiku geometrijsko zveznost stopnje $G^1$, mora poleg 
pogoja za $G^0$ veljati še $R_x = S_u u_x + S_v v_x$ vzdolž $C$, 
za $G^2$ mora poleg pogojev za $G^0$ in $G^1$ veljati še 
$G^2$: $R_xx = S_{uu} u_x^2 + 2S_{uv}u_x v_x + S_{vv} v_x^2 + S_u u_{xx} + S_v v_{xx}$ vzdolž $C$
in tako dalje.

Splošneje,\todo{grdo} za geometrijsko zveznost stopnje $n$ \todo{(ali se tako reče?)}, kjer je $n\in \N_0$ velja naslednje:
$$\frac{\partial^j R}{\partial x^j}\Bigr|_{\substack{C}}= \sum_{k=1}^j\sum_{h=0}^k A_{jkh}\frac{\partial^k S}{\partial u^h \partial v^{k-h}}\Bigr|_{\substack{C}}$$
za vsak $j=0,1,\ldots,n$.
Tu z $A_{jkh}$ označujemo koeficient
$$A_{jkh}={k \choose h}\sum_{\substack{m_1+\cdots +m_k =j \\ m_1, \ldots, m_k >0}}
\frac{j!}{k!m_1!\cdots m_k!} u_{x^{m_1}} \cdots u_{x^{m_h}} v_{x^{m_{h+1}}} \cdots v_{x^{m_k}}\Bigr|_{\substack{C}}.$$
Z $u_x^{m_i}$ je označen $m_i$-ti delni odvod funkcije $u$ po $x$.
\todo{ali je treba to grozo dokazati?}

-----------------------------------------------------------------------------

\todo{kaj je s to lemo?}
\begin{lema}\todo{?? ali je dokaz tega potreben?}
  Naj bo $f(u,v)$ funkcija razreda $C^n$ in $u(t)$ in $v(t)$ reparametrizaciji razreda $C^n$. Potem
  \begin{align*}
  \frac{d^k f}{dt^k}=& \sum_{i=1}^k\sum_{|\mathbf{m_i}|=k} A_{\mathbf{m_i}}^k \sum_{h=0}^i {i \choose h} u^{(m_1)}\cdots u^{(m_h)}(v) v^{(m_{h+1})}\cdots v^{(m_i)} \\
  &\cdot\frac{\partial^i f}{\partial u_r^h \partial v_r^{i-h}},
  \end{align*}
  kjer $k=1,\ldots,n$ ter\\
  $\mathbf{m_i} = (m1,m2,\ldots,m_i)$, $|\mathbf{m_i}|=m_1+m_2+\cdots+m_i$ in $A_{\mathbf{m_i}}^k = \frac{k!}{i!m_1!\cdots m_i!}$.
\end{lema}

------------------------------------------------------------------

\todo{zdaj si pogledamo ekvivalentno/alternativno definicijo G zveznosti? s pomočjo junction functions?}
\nevem{Z uvedbo funkcij $\alpha_1, \ldots, \alpha_n$ in $\beta_1 \ldots, \beta_n$ pridemo do nekoliko drugačne definicije
geometrijske zveznosti.}
\todo{funkcije $\alpha_1, \ldots, \alpha_n$ in $\beta_1 \ldots, \beta_n$ imenujemo
junction/connection functions}

Naj bodo $\alpha_1, \ldots, \alpha_n$ in $\beta_1 \ldots, \beta_n$ funkcije 
razreda $C^n$ ene spremenljivke. Definirajmo:\\
$$u(x,y) =  u_0 + \sum_{i=0}^n \frac{1}{i!}\alpha_i(y)(x-x_0)^i$$
$$v(x,y) = y + \sum_{i=0}^n \frac{1}{i!}\beta_i(y)(x-x_0)^i$$

\todo{ali moram povsod pisati, kam te funkcije slikajo? recimo paramterizacije in alfa, beta itd.}

Opazimo, da za $i=1, \ldots, k$ velja
$$\frac{\partial^i u}{\partial x^i}(x_0,y)=\alpha_i(y),$$
$$\frac{\partial^i v}{\partial x^i}(x_0,y)=\beta_i(y).$$
%\end{flalign*}

Stične funkcije lahko izberemo skoraj povsem poljubno. Upoštevati moramo le dva pogoja, 
ki omejujeta izbiro funkcije $\alpha_1$. Prvi pogoj sledi iz zahteve po regularnosti 
reparametrizacijske funkcije $f$.

\todo{def. regularnosti?}
\todo{kaj je lokalna regularnost?}

Reparametrizacijska funkcija $f$ je regularna vzdolž $C$, če sta oba njena parcialna odvoda prvega 
reda linearno neodvisna, torej če velja $\frac{\partial f}{\partial x}(x_0,y)\times \frac{\partial f}{\partial y}(x_0,y) \neq 0$.

Razpišimo oba odvoda reparametrizacijske funkcije $f(x,y)=(u(x,y),v(x,y))$ vzdolž krivulje $C$ 
in ju skušajmo zapisati s pomočjo stičnih funkcij. Za odvod po spremenljivki $x$ velja:

$$\frac{\partial f}{\partial x}(x_0,y)=(\frac{\partial u}{\partial x}(x_0,y),\frac{\partial v}{\partial x}(x_0,y))=(\alpha_1(y),\beta_1(y)),$$
pri čemer smo uporabili opazko \todo{sklic}.
Če razpišemo odvod po spremenljivki $y$, pa dobimo:

$$\frac{\partial f}{\partial y}(x_0,y)=(\frac{\partial u}{\partial y}(x_0,y),\frac{\partial v}{\partial y}(x_0,y))=(0,1).$$

Vektorski produkt $\frac{\partial f}{\partial x}(x_0,y)\times \frac{\partial f}{\partial y}(x_0,y)$
je torej enak 

$$\frac{\partial f}{\partial x}(x_0,y)\times \frac{\partial f}{\partial y}(x_0,y) = 
(\alpha_1(y),\beta_1(y))\times (0,1) = \alpha_1(y)$$

Sledi, da je reparametrizacija regularna, natanko tedaj \todo{?}, ko za pripradajočo stično \todo{junction?} 
funkcijo $\alpha_1$ velja $\alpha_1(y)\neq 0$ vzdolž stične krivulje $C$. 
\todo{ali je potrben podatek "vzdolž krivulje C"?}
\todo{to moraš nekako motivirati: zakaj si to sploh pogledamo? zato, ker to predstavlja 
pogoj za $\alpha_1$?}

Drugo, na kar moramo paziti pri izbiri funkcije $\alpha_1$ pa je njen predznak. 
Pri izbiri napačnega predznaka namreč lahko pride do stika v obliki "špice". 
\todo{tega ne razumem čisto. tu bi bil potreben kakšen primer.}

Vpeljava stičnih funkcij nas pripelje do nekoliko drugačne definicije geometrijske 
zveznosti.

\begin{definicija}
  Naj bosta $R(x,y)$ in $S(u,v)$ regularni $C^n$ parametrizaciji dveh ploskev, 
  ki se stikata v krivulji $C(y)=R(x_0,y)=S(u_0,y)$. 
  Pravimo, da se $R$ in $S$ stikata z $G^n$-zveznostjo vzdolž krivulje $C$, 
  če za vsako točko $b_0=C(y_0)$ obstajajo take funkcije razreda $C^n$ ene 
  spremenljivke $\alpha_1, \ldots, \alpha_n$ in $\beta_1 \ldots, \beta_n$, 
  da $\alpha_1(y)\neq0$, pri čemer mora imeti $\alpha_1$ ustrezen predznak \todo{??}, 
  in da velja
  $$\frac{\partial^j R}{\partial x^j}\Bigr|_{\substack{C}}= \sum_{k=1}^j\sum_{h=0}^k A_{jkh}\frac{\partial^k S}{\partial u^h \partial v^{k-h}}\Bigr|_{\substack{C}}$$
  za vsak $j=0,1,\ldots,n$.
  Tu z $A_{jkh}$ označujemo koeficient
  $$A_{jkh}={k \choose h}\sum_{\substack{m_1+\cdots +m_k =j \\ m_1, \ldots, m_k >0}}
  \frac{j!}{k!m_1!\cdots m_k!} \alpha_1 \cdots \alpha_{m_h} \beta_{m_{h+1}} \cdots \beta_{m_k}\Bigr|_{\substack{C}}$$
\end{definicija}

-----------------------------------------------------

\todo{kaj je s tem izrekom?}
\begin{izrek}
  Naj bosta $R(u_r,v_r)$ in $S(u_s,v_s)$ regularni $C^n$ parametrizaciji dveh ploskev, ki se stikata v krivulji $C(v)=R(u_{r0},v)=S(u_{s0},v)$. Ploskvi $R$ in $S$ sta $G^n$-zvezni vzdolž skupnega roba natanko tedaj ko obstajajo funkcije $p_i(v)$, $q_i(v)$, $i=1,\ldots, n$, da velja
  \begin{align*}
  \frac{\partial^k S}{\partial u_s^k}\Bigr|_{\substack{C}}=& \sum_{i=1}^k\sum_{|\mathbf{m_i}|=k} A_{\mathbf{m_i}}^k \sum_{h=0}^i {i \choose h} p_{m_1}(v) \cdots p_{m_h}(v) q_{m_{h+1}}(v) \cdots q_{m_i} \\
  &\cdot\frac{\partial^i R}{\partial u_r^h \partial v_r^{i-h}},
  \end{align*}
  kjer $k=1,\ldots,n$ ter\\
  $\mathbf{m_i} = (m1,m2,\ldots,m_i)$, $|\mathbf{m_i}|=m_1+m_2+\cdots+m_i$ in $A_{\mathbf{m_i}}^k = \frac{k!}{i!m_1!\cdots m_i!}$.
\end{izrek}

-----------------------------------------------------------------

\section{$G^1$ zveznost}

\todo{nekaj v stilu, da se bomo natančneje ukvarjali z G1 zveznostjo.}
\todo{lahko povem, da je to zveznost tangentnih ravnin oz. zveznost enotskih normal in da si 
bomo ogledali, kako do tega pridemo.}

Imejmo ploskvi $R(x,y)$ in $S(u,v)$, ki se v krivulji $C(y) = R(x_0,y) = S(u_0,y)$ 
stikata z geometrijsko zveznostjo $G^1$.  
Sledi, da je $R_y(x_0,y) = S_y(x_0,y) = S_v(x_0,y)$. Kot smo že videli \todo{ugh} v poglavju \todo{sklic}, 
nam je zato potrebno opazovati zgolj odvode v smeri $x$.

Ker je stik obeh ploskev v $C$ $G^1$-zvezen, po definiciji \todo{sklic} obstajata 
funkciji $\alpha_1$ in $\beta_1$, kjer je $\alpha_1(y) \neq 0$ za vsak $y$ 
in ima ustrezen predznak, da velja:

$$R_x(x_0,y)=\alpha_1(y) S_u(u_0,y)+\beta_1(y)S_v(u_0,y).$$

Zgornja enačba nam pove, da so parcialni odvodi $R_x(x_0,y)$, $S_u(u_0,y)$ in 
$S_v(u_0,y)$ v vsaki točki $y$ linearno neodvisni. Torej so v vsaki točki $y$ del 
iste tangentne ravnine na krivuljo $C$. Zato torej $G^1$-zveznost imenujemo tudi 
zveznost tangentnih ravnin.

\nevem{v predstavitvi sem šla tako: $R(x_0,y)=S(u_0,y)$,  
  $R_x(x_0,y)=u_x(x_0,y)S_u(u_0,y)+v_x(x_0,y)S_v(u_0,y)$,  
  $R_x(x_0,y)=\alpha_1(y) S_u(u_0,y)+\beta_1(y)S_v(u_0,y)$}  
\todo{na mestu najbrž kakšna slika}

Oglejmo si še, od kod pride poimenovanje ''zveznost enotskih normal''. 
Znova opazujemo enačbo 
$$R_x(x_0,y)=\alpha_1(y) S_u(u_0,y)+\beta_1(y)S_v(u_0,y).$$
\todo{ali je bolje, da se tu samo skličem?}
Enačbo sedaj z obeh strani vektorsko pomnožimo z $R_y(x_0,y)$:
$$R_x(x_0,y)\times R_y(x_0,y)=\alpha_1(y) S_u(u_0,y)\times R_y(x_0,y)+\beta_1(y)S_v(u_0,y)\times R_y(x_0,y).$$
Upoštevamo lahko, da je $R_y(x_0,y)=S_v(u_0,y)$. Dobimo:
$$R_x(x_0,y) \times R_y(x_0,y) = \alpha_1(y) S_u(u_0,y) \times S_v(u_0,y).$$
Od tod vidimo, da sta normali na ploskvi $R$ in $S$ na njunu stični 
krivulji vzporedni. \todo{grdo} Na skupnem robu imata torej obe ploskvi enaki 
enotski normali:
$$\frac{R_x(x_0,y) \times R_y(x_0,y)}{||R_x(x_0,y) \times R_y(x_0,y)||}=\frac{S_u(u_0,y) \times S_v(u_0,y)}{||S_u(u_0,y) \times S_v(u_0,y)||}.$$

\todo{tukaj pride nek vezni tekst? ali pa je to ok?}
Ker parcialni odvodi $R_x(x_0,y)$, $S_u(u_0,y)$ in $S_v(u_0,y)$ ležijo na isti 
tangentni ravnini, velja tudi: \todo{zelo grdo}
$$\det(R_x(x_0,y), S_u(u_0,y), S_v(u_0,y))=0.$$
Torej obstajajo funkcije \todo{povedati kakšne, iz kje kam?} $\lambda$, 
$\mu$ in $\gamma$, da velja: \todo{ali moram to kaj bolj natančno utemeljiti?}
\todo{treba motivirati, zakaj to gledamo}
$$\lambda(y)R_x(x_0,y)=\mu(y)S_u(u_0,y)+\gamma(y)S_v(u_0,y).$$

Če predpostavimo, da sta ploskvi $R$ in $S$ polinomski, lahko tudi za $\lambda$, $\mu$ in $\gamma$ 
izberemo polinome, \todo{izberemo? ali niso točno določeni?} kar nam zelo olajša 
konstrukcijo geometrijsko zveznih ploskev. 

\todo{najbrž lahko poveš, da se bomo v naslednjih poglavjih ukvarjali z izbiro 
teh polinomskih krivulj}

\todo{mogoče moraš tu napisati, kako se pride do teh polinomov: tiste prve komponente. 
ampak tega ne razumem.}

\section{Bézierjeve ploskve}
\todo{pogledamo si poseben primer polinomskih param ploskev, ki so tudi uporabne v praksi}

$i$-ti Bernsteinov bazni polinom
$$B_i^n(t)={n \choose i}t^i (1-t)^{n-i}\textrm{, } t\in[0,1]$$
Lastnosti:
\begin{itemize}
\item $B_i^n(0) =\delta_{i,0}$
\item $B_i^n(1)=\delta_{i,n}$
%tvorijo bazo prostora polinomov Pn
\end{itemize}

\begin{definicija}
  Naj bodo dane točke $\mathbf{b}_{i,j}\in \R^d$, $i=0,1,\ldots,m$, $j=0,1,\ldots,n$. Bézierjeva ploskev iz tenzorskega produkta je parametrično podana ploskev
  $$\mathbf{b}^{m,n} : [0,1]\times[0,1] \rightarrow \R^d$$
  s predpisom
  $$\mathbf{b}^{m,n}(u,v)=\sum_{i=0}^m \sum_{j=0}^n \mathbf{b_{i,j}} B_i^m(u) B_j^n(v).$$
  Točke $\mathbf{b}_{i,j}$ imenujemo kontrolne točke, poligon, ki jih povezuje, pa kontrolni poligon.
\end{definicija}
Velja: $\mathbf{b^{m,n}}(0,0)=\mathbf{b}_{0,0}$, $\mathbf{b^{m,n}}(1,0)=\mathbf{b}_{m,0}$, $\mathbf{b^{m,n}}(0,1)=\mathbf{b}_{0,n}$, $\mathbf{b^{m,n}}(1,1)=\mathbf{b}_{m,n}$
%interpolacija robnih točk
%vstavi sliko

Odvod Bézierjeve ploskve iz tenzorskega produkta:\\
$$\frac{\partial^{r+s}}{\partial u^r \partial v^s}\mathbf{b}^{m,n}(u,v)=\frac{m!}{(m-r)!}\frac{n!}{(n-s)!}\sum_{i=0}^{m-r} \sum_{j=0}^{n-s} \Delta^{r,s}\mathbf{b}_{i,j}B_i^{m-r}(u)B_j^{n-s}(v),$$
kjer $\Delta^{1,0} \mathbf{b}_{i,j} = \mathbf{b}_{i+1,j}-\mathbf{b}_{i,j}$,\\
$\Delta^{0,1} \mathbf{b}_{i,j} = \mathbf{b}_{i,j+1}-\mathbf{b}_{i,j}$,\\
$\Delta^{r,0} \mathbf{b}_{i,j} = \Delta^{r-1,0} \mathbf{b}_{i+1,j}-\Delta^{r-1,0} \mathbf{b}_{i,j}$,\\
$\Delta^{0,s} \mathbf{b}_{i,j} = \Delta^{0,s-1} \mathbf{b}_{i,j+1}-\Delta^{0,s-1} \mathbf{b}_{i,j}$.
%lahko poveš, da so to v bistvu vektorji
%posebej lahko napišeš (ali dodaš) odvode 1. stopnje
%dodaj pogoje gladkosti za C1 ?
%dodaj trikotne krpe

\section{Primeri konstrukcij $G^1$ ploskev}
\begin{frame}
Dve bikubični Bézierjevi ploskvi iz tenzorskega produkta:
$$R(u,v)=\sum_{i=0}^3 \sum_{j=0}^3 \mathbf{P}_{i,j}B_i^3(u) B_j^3(v)$$ in
$$S(u,v)=\sum_{i=0}^3 \sum_{j=0}^3 \mathbf{Q}_{i,j}B_i^3(u) B_j^3(v),$$
ki se stikata v $C(v)=R(0,v)=S(0,v)$.
\end{frame}
%vstavi sliko

%\section{Tehnični napotki za pisanje}
%
%\subsection{Sklicevanje in citiranje}
%Za sklice uporabljamo \verb|\ref|, za sklice na enačbe \verb|\eqref|, za citate \verb|\cite|. Pri
%sklicevanju in citiranju sklicano številko povežemo s prejšnjo besedo z nedeljivim presledkom
%$\sim$, kot npr.\ \verb|iz trditve~\ref{trd:obstoj-omega} vidimo|.
%
%\begin{primer}
%  Zaporedje~\eqref{eq:zero-kompleks} iz dokaza trditve~\ref{trd:obstoj-omega} na
%  strani~\pageref{trd:obstoj-omega} lahko najdemo tudi v Spletni enciklopediji zaporedij~\cite{oeis}.
%  Citiramo lahko tudi bolj natančno~\cite[trditev 2.1, str.\ 23]{lebedev2009introduction}.
%\end{primer}
%
%\subsection{Okrajšave}
%Pri uporabi okrajšav \LaTeX{} za piko vstavi predolg presledek, kot npr. tukaj. Zato se za vsako
%piko, ki ni konec stavka doda presledek običajne širine z ukazom \verb*|\ |, kot npr.\ tukaj.
%Primerjaj z okrajšavo zgoraj za razliko.
%
%\subsection{Vstavljanje slik}
%Sliko vstavimo v plavajočem okolju \texttt{figure}. Plavajoča okolja \emph{plavajo} po tekstu, in
%jih lahko postavimo na vrh strani z opcijskim parametrom `\texttt{t}', na lokacijo, kjer je v kodi s
%`\texttt{h}', in če to ne deluje, potem pa lahko rečete \LaTeX u, da ga \emph{res} želite tukaj,
%kjer ste napisali, s `\texttt{h!}'. Lepo je da so vstavljene slike vektorske (recimo \texttt{.pdf}
%ali \texttt{.eps} ali \texttt{.svg}) ali pa \texttt{.png} visoke resolucije (več kot
%\unit[300]{dpi}).  Pod vsako sliko je napis in na vsako sliko se skličemo v besedilu. Primer
%vektorske slike je na sliki~\ref{fig:sample}. Vektorsko sliko prepoznate tako, da močno
%zoomate v sliko, in še vedno ostane gladka. Več informacij je na voljo na
%\url{https://en.wikibooks.org/wiki/LaTeX/Floats,_Figures_and_Captions}. Če so slike bitne, kot na
%primer slika~\ref{fig:image}, poskrbite, da so v dovolj visoki resoluciji.

%\begin{figure}[h]
%  \centering
%  \includegraphics[width=0.6\textwidth]{images/sample.pdf}
%% \caption[caption za v kazalo]{Dolg caption pod sliko}
%  \caption[Primer vektorske slike.]{Primer vektorske slike z oznakami v enaki pisavi, kot jo
%     uporablja \LaTeX{}.  Narejena je s programom Inkscape, \LaTeX{} oznake so importane v
%     Inkscape iz pomožnega PDF.}
%  \label{fig:sample}
%\end{figure}

%\begin{figure}[h]
%  \centering
%  \includegraphics[width=0.8\textwidth]{images/image.png}
%  \caption[Primer bitne slike.]{Primer bitne slike, izvožene iz Matlaba. Poskrbite, da so slike v
%  dovolj visoki resoluciji in da ne vsebujejo prosojnih elementov (to zahteva PDF/A-1b format).}
%  \label{fig:image}
%\end{figure}

%\subsection{Kako narediti stvarno kazalo}
%Dodate ukaze \verb|\index{polje}| na besede, kjer je pojavijo, kot tukaj\index{tukaj}.
%Več o stvarnih kazalih je na voljo na \url{https://en.wikibooks.org/wiki/LaTeX/Indexing}.
%
%\subsection{Navajanje literature}
%Članke citiramo z uporabo \verb|\cite{label}|, \verb|\cite[text]{label}| ali pa več naenkrat s
%\verb|\cite\{label1, label2}|. Tudi tukaj predhodno besedo in citat povežemo z nedeljivim presledkom
%$\sim$. Na primer~\cite{chen2006meshless,liu2001point}, ali pa \cite{kibriya2007empirical}, ali pa
%\cite[str.\ 12]{trobec2015parallel}, \cite[enačba (2.3)]{pereira2016convergence}.
%Vnosi iz \verb|.bib| datoteke, ki niso citirani, se ne prikažejo v seznamu literature, zato jih
%tukaj citiram.~\cite{vene2000categorical}, \cite{gregoric2017stopniceni}, \cite{slak2015induktivni},
%\cite{nsphere}, \cite{kearsley1975linearly}, \cite{STtemplate}, \cite{NunbergerTand}.

% Literatura:
% Primer navajanja na http://www.fmf.uni-lj.si/storage/24240/LiteraturaM.pdf,
% ampak bi moral stil poskrbeti za vse. Reference se uredijo po abecedi.
% Če nobena izbira izmed @book, @atricle,... ni ok, potem se lahko vse napiše v
% @misc pod note={} in deluje tako kot normalen LaTeX.
% Komentar v bib datoteki se naredi samo s parom { }
% Za urejanje literature avtor priporoča program Jabref, ki zna tudi avtomatsko
% okrajšati imena revij. Za pravilno sortiranje vnosov brez avtorja, uporabite
% polje key={ }, kot v primeru.
% V primeru napak ustvarite issue na GitHubu ali pišite na jure.slak@fmf.uni-lj.si.
\cleardoublepage                           % na desni strani
\phantomsection                            % da prav delujejo hiperlinki
\addcontentsline{toc}{section}{\bibname}   % dodajmo v kazalo
\bibliographystyle{fmf-sl}                 % uporabljen stil je v datoteki fmf-sl.bst, na voljo tudi angleška verzija
\bibliography{\literatura}                 % literatura je v datoteki, definirani na začetku

% Za stvarno kazalo
\cleardoublepage                           % na desni strani
\phantomsection                            % da prav delujejo hiperlinki
\addcontentsline{toc}{section}{\indexname} % dodajmo v kazalo
\printindex

\end{document}
