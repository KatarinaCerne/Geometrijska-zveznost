\documentclass[11pt]{beamer}

\usetheme{Berlin}
\usecolortheme{rose}

\usepackage[slovene]{babel}
\usepackage[utf8]{inputenc} 
\usepackage[T1]{fontenc}
\usepackage{lmodern}
\usepackage {eurosym}
\usepackage{amsfonts}
\usepackage{graphicx}

\usepackage{amsmath,amssymb,amsthm}
\usepackage{mathrsfs}
\usepackage{amsbsy}

\newcommand{\R}{\mathbb R}
\newcommand{\N}{\mathbb N}
\newcommand{\Z}{\mathbb Z}
\newcommand{\C}{\mathbb C}
\newcommand{\Q}{\mathbb Q}

\setbeamercovered{invisible}

\newtheorem{izrek}{Izrek} 

\newtheorem{definicija}{Definicija} 
\newtheorem{lema}{Lema}

\title[Konstrukcija geometrijsko zveznih ploskev]{Konstrukcija geometrijsko zveznih ploskev}

\author{Katarina Černe}

\date{

 \small{FMF, Ljubljana, 22. 4. 2019}
}

\begin{document}

\section{Geometrijska zveznost}

\begin{frame}
\maketitle
\end{frame}

\begin{frame}
\begin{definicija}
Ploskev pripada razredu $G^n$ oziroma je geometrijsko zvezna z redom $n$, če v okolici vsake njene točke obstaja lokalna regularna parametrizacija razreda $C^n$.
\end{definicija}
\end{frame}

\begin{frame}
\begin{definicija}
Naj bosta $R(u_r,v_r)$ in $S(u_s,v_s)$ regularni $C^n$ parametrizaciji dveh ploskev, ki se stikata v krivulji $C(v)=R(u_{r0},v)=S(u_{s0},v)$. Pravimo, da sta $R$ in $S$ $G^n$-zvezni vzdolž krivulje $C$, če v okolici $C$ obstajata taki $C^n$ reparametrizacijski funkciji $u_r(u_s,v_s)$ in $v_r(u_s,v_s)$, da je 
$$\frac{\partial^{m+k}}{\partial u_s^m \partial v_s^k}\bar{R}(u_s,v_s)\Bigr|_{\substack{C}} =  \frac{\partial^{m+k}}{\partial u_s^m \partial v_s^k}S(u_s,v_s)\Bigr|_{\substack{C}} \textrm{ za } m+k=1,\ldots,n$$
kjer je $\bar{R}(u_s,v_s)=R(u_r(u_s,v_s),v_r(u_s,v_s))$.
\end{definicija}
\end{frame}

\begin{frame}
\begin{izrek}
Naj bosta $R(u_r,v_r)$ in $S(u_s,v_s)$ regularni $C^n$ parametrizaciji dveh ploskev, ki se stikata v krivulji $C(v)=R(u_{r0},v)=S(u_{s0},v)$. Ploskvi $R$ in $S$ sta $G^n$-zvezni vzdolž skupnega roba natanko tedaj ko obstajajo funkcije $p_i(v)$, $q_i(v)$, $i=1,\ldots, n$, da velja
\begin{align*}
\frac{\partial^k S}{\partial u_s^k}\Bigr|_{\substack{C}}=& \sum_{i=1}^k\sum_{|\mathbf{m_i}|=k} A_{\mathbf{m_i}}^k \sum_{h=0}^i {i \choose h} p_{m_1}(v) \cdots p_{m_h}(v) q_{m_{h+1}}(v) \cdots q_{m_i} \\
&\cdot\frac{\partial^i R}{\partial u_r^h \partial v_r^{i-h}},
\end{align*}
kjer $k=1,\ldots,n$ ter\\
$\mathbf{m_i} = (m1,m2,\ldots,m_i)$, $|\mathbf{m_i}|=m_1+m_2+\cdots+m_i$ in $A_{\mathbf{m_i}}^k = \frac{k!}{i!m_1!\cdots m_i!}$.
\end{izrek}
\end{frame}

\begin{frame}
\begin{lema}
Naj bo $f(u,v)$ funkcija razreda $C^n$ in $u(t)$ in $v(t)$ reparametrizaciji razreda $C^n$. Potem
\begin{align*}
\frac{d^k f}{dt^k}=& \sum_{i=1}^k\sum_{|\mathbf{m_i}|=k} A_{\mathbf{m_i}}^k \sum_{h=0}^i {i \choose h} u^{(m_1)}\cdots u^{(m_h)}(v) v^{(m_{h+1})}\cdots v^{(m_i)} \\
&\cdot\frac{\partial^i f}{\partial u_r^h \partial v_r^{i-h}},
\end{align*}
kjer $k=1,\ldots,n$ ter\\
$\mathbf{m_i} = (m1,m2,\ldots,m_i)$, $|\mathbf{m_i}|=m_1+m_2+\cdots+m_i$ in $A_{\mathbf{m_i}}^k = \frac{k!}{i!m_1!\cdots m_i!}$.
\end{lema}
\end{frame}

%kako to izgleda za n=1 (n=2?)
%geometrijska interpretacija

\section{Bézierjeve ploskve}

\begin{frame}
$i$-ti Bernsteinov bazni polinom
$$B_i^n(t)={n \choose i}t^i (1-t)^{n-i}\textrm{, } t\in[0,1]$$
Lastnosti:
\begin{itemize}
\item $B_i^n(0) =\delta_{i,0}$
\item $B_i^n(1)=\delta_{i,n}$
%tvorijo bazo prostora polinomov Pn
\end{itemize}
\end{frame}

\begin{frame}
\begin{definicija}
Naj bodo dane točke $\mathbf{b}_{i,j}\in \R^d$, $i=0,1,\ldots,m$, $j=0,1,\ldots,n$. Bézierjeva ploskev iz tenzorskega produkta je parametrično podana ploskev
$$\mathbf{b}^{m,n} : [0,1]\times[0,1] \rightarrow \R^d$$
s predpisom
$$\mathbf{b}^{m,n}(u,v)=\sum_{i=0}^m \sum_{j=0}^n \mathbf{b_{i,j}} B_i^m(u) B_j^n(v).$$
Točke $\mathbf{b}_{i,j}$ imenujemo kontrolne točke, poligon, ki jih povezuje, pa kontrolni poligon.
\end{definicija}
Velja: $\mathbf{b^{m,n}}(0,0)=\mathbf{b}_{0,0}$, $\mathbf{b^{m,n}}(1,0)=\mathbf{b}_{m,0}$, $\mathbf{b^{m,n}}(0,1)=\mathbf{b}_{0,n}$, $\mathbf{b^{m,n}}(1,1)=\mathbf{b}_{m,n}$
%interpolacija robnih točk
%vstavi sliko
\end{frame}

\begin{frame}
Odvod Bézierjeve ploskve iz tenzorskega produkta:\\
$$\frac{\partial^{r+s}}{\partial u^r \partial v^s}\mathbf{b}^{m,n}(u,v)=\frac{m!}{(m-r)!}\frac{n!}{(n-s)!}\sum_{i=0}^{m-r} \sum_{j=0}^{n-s} \Delta^{r,s}\mathbf{b}_{i,j}B_i^{m-r}(u)B_j^{n-s}(v),$$
kjer $\Delta^{1,0} \mathbf{b}_{i,j} = \mathbf{b}_{i+1,j}-\mathbf{b}_{i,j}$,\\
$\Delta^{0,1} \mathbf{b}_{i,j} = \mathbf{b}_{i,j+1}-\mathbf{b}_{i,j}$,\\
$\Delta^{r,0} \mathbf{b}_{i,j} = \Delta^{r-1,0} \mathbf{b}_{i+1,j}-\Delta^{r-1,0} \mathbf{b}_{i,j}$,\\
$\Delta^{0,s} \mathbf{b}_{i,j} = \Delta^{0,s-1} \mathbf{b}_{i,j+1}-\Delta^{0,s-1} \mathbf{b}_{i,j}$.
%lahko poveš, da so to v bistvu vektorji
%posebej lahko napišeš (ali dodaš) odvode 1. stopnje
%dodaj pogoje gladkosti za C1 ?
%dodaj trikotne krpe??
\end{frame}

\section{Primeri konstrukcij $G^1$ ploskev}
\begin{frame}
Dve bikubični Bézierjevi ploskvi iz tenzorskega produkta:
$$R(u,v)=\sum_{i=0}^3 \sum_{j=0}^3 \mathbf{P}_{i,j}B_i^3(u) B_j^3(v)$$ in
$$S(u,v)=\sum_{i=0}^3 \sum_{j=0}^3 \mathbf{Q}_{i,j}B_i^3(u) B_j^3(v),$$
ki se stikata v $C(v)=R(0,v)=S(0,v)$.
\end{frame}
%vstavi sliko
\begin{frame}
Primer 1:\\

\end{frame}
\end{document}