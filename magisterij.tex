% !TeX spellcheck = sl_SI
% vim: set spell spelllang=sl:
% za preverjanje črkovanja, če se uporablja Texstudio ali vim
\documentclass[12pt,a4paper,twoside]{article}
\usepackage[utf8]{inputenc}  % pravilno razpoznavanje unicode znakov

% NASLEDNJE UKAZE USTREZNO POPRAVI
\newcommand{\program}{Matematika} % ime studijskega programa
\newcommand{\imeavtorja}{Katarina Černe} % ime avtorja
\newcommand{\imementorja}{prof.~dr.~} % akademski naziv in ime mentorja, uporabi poln naziv, prof.~dr.~, doc.~dr., ali izr.~prof.~dr.
\newcommand{\imesomentorja}{} % akademski naziv in ime somentorja, če ga imate
\newcommand{\naslovdela}{Naslov vašega dela}
\newcommand{\letnica}{2020} % letnica magistriranja
\newcommand{\opis}{}  % Opis dela v eni povedi. Ne sme vsebovati matematičnih simbolov v $ $.
\newcommand{\kljucnebesede}{\sep } % ključne besede, ločene z \sep, da se PDF metapodatki prav procesirajo
\newcommand{\keywords}{\sep } % ključne besede v angleščini
\newcommand{\organization}{Univerza v Ljubljani, Fakulteta za matematiko in fiziko} % fakulteta
\newcommand{\literatura}{literatura}  % pot do datoteke z literaturo (brez .bib končnice)
\newcommand{\sep}{, }  % separator med ključnimi besedami v besedilu
% KONEC PODATKOV

\usepackage{bibentry}         % za navajanje literature v programu dela s celim imenom
\nobibliography{\literatura}
\newcommand{\plancite}[1]{\item[\cite{#1}] \bibentry{#1}} % citiranje v programu dela

\usepackage{filecontents}  % za pisanje datoteke s PDF metapodatki
\usepackage{silence} \WarningFilter{latex}{Overwriting file}  % odstrani annoying warning o obstoju datoteke
% datoteka s PDF metapodatki, zgenerira se kot magisterij.xmpdata
\begin{filecontents*}{\jobname.xmpdata}
  \Title{\naslovdela}
  \Author{\imeavtorja}
  \Keywords{\kljucnebesede}
  \Subject{\opis}
  \Org{\organization}
\end{filecontents*}

\usepackage[a-1b]{pdfx}  % zgenerira PDF v tem PDF/A-1b formatu, kot zahteva knjižnica
\hypersetup{bookmarksopen, bookmarksdepth=3, colorlinks=true,
  linkcolor=black, anchorcolor=black, citecolor=black, filecolor=black,
  menucolor=black, runcolor=black, urlcolor=black, pdfencoding=auto,
  breaklinks=true, psdextra}

\usepackage[slovene]{babel}  % slovenščina
\usepackage[T1]{fontenc}     % naprednejše kodiranje fonta
\usepackage{amsmath,amssymb,amsfonts,amsthm} % matematični paketi
%\usepackage[dvipsnames,usenames]{color} % barve
\usepackage{graphicx}     % za slike
\usepackage{emptypage}    % prazne strani so neoštevilčene, ampak so štete
\usepackage{units}        % fizikalne enote kot \unit[12]{kg} s polovico nedeljivega presledka, glej primer v kodi
\usepackage{makeidx}      % za stvarno kazalo, lahko zakomentiraš, če ne rabiš
\makeindex                % za stvarno kazalo, lahko zakomentiraš, če ne rabiš
% oblika strani
\usepackage[
  top=3cm,
  bottom=3cm,
  inner=3.5cm,      % margini za dvostransko tiskanje
  outer=2.5cm,
  footskip=40pt     % pozicija številke strani
]{geometry}

% VEČ ZANIMIVIH PAKETOV
% \usepackage{array}      % več možnosti za tabele
% \usepackage[list=true,listformat=simple]{subcaption}  % več kot ena slika na figure, omogoči slika 1a, slika 1b
% \usepackage[all]{xy}    % diagrami
% \usepackage{doi}        % za clickable DOI entrye v bibliografiji
% \usepackage{enumerate}     % več možnosti za sezname

% Za barvanje source kode
% \usepackage{minted}
% \renewcommand\listingscaption{Program}

% Za pisanje psevdokode
% \usepackage{algpseudocode}  % za psevdokodo
% \usepackage{algorithm}
% \floatname{algorithm}{Algoritem}
% \renewcommand{\listalgorithmname}{Kazalo algoritmov}

% DRUGI TVOJI PAKETI:
% tukaj
\usepackage[utf8]{inputenc}
\usepackage{lmodern}
\usepackage{eurosym}
\usepackage{hyperref}
\usepackage{subfigure}
\usepackage{xcolor}
\usepackage{tcolorbox}
\usepackage{enumitem}

\setlength{\overfullrule}{50pt} % označi predlogo vrstico
\pagestyle{plain}               % samo številka strani na dnu, nobene glave / noge

% ukazi za matematična okolja
\theoremstyle{definition} % tekst napisan pokončno
\newtheorem{definicija}{Definicija}[section]
\newtheorem{primer}[definicija]{Primer}
\newtheorem{opomba}[definicija]{Opomba}
\newtheorem{aksiom}{Aksiom}

\theoremstyle{plain} % tekst napisan poševno
\newtheorem{lema}[definicija]{Lema}
\newtheorem{izrek}[definicija]{Izrek}
\newtheorem{trditev}[definicija]{Trditev}
\newtheorem{posledica}[definicija]{Posledica}

\numberwithin{equation}{section}  % števec za enačbe zgleda kot (2.7) in se resetira v vsakem poglavju

% Matematični ukazi
\newcommand{\R}{\mathbb R}
\newcommand{\N}{\mathbb N}
\newcommand{\Z}{\mathbb Z}
\renewcommand{\C}{\mathbb C}
\newcommand{\Q}{\mathbb Q}

\newcommand{\todo}[1]{{\color{red}{#1}}}
\newcommand{\nevem}[1]{{\color{blue}{#1}}}

\newcommand{\ddd}[1]{\mathbf{#1}}

% \DeclareMathOperator{\tr}{tr}  % morda potrebuješ operator za sled ali kaj drugega?

% bold matematika znotraj \textbf{ }, tudi v naslovih, kot \omega spodaj
\makeatletter \g@addto@macro\bfseries{\boldmath} \makeatother

% Poimenuj kazalo slik kot ``Kazalo slik'' in ne ``Slike''
\addto\captionsslovene{
  \renewcommand{\listfigurename}{Kazalo slik}%
}

% če želiš, da se poglavja začnejo na lihih straneh zgoraj
% \let\oldsection\section
% \def\section{\cleardoublepage\oldsection}

%%%%%%%%%%%%%%%%%%%%%%%%%%%%%%%%%%%%%%%%%%
%%%%%%           DOCUMENT           %%%%%%
%%%%%%%%%%%%%%%%%%%%%%%%%%%%%%%%%%%%%%%%%%

\begin{document}

%\pagenumbering{roman} % začnemo z rimskimi številkami
%\thispagestyle{empty} % ampak na prvi strani ni številke

%\noindent{\large
%UNIVERZA V LJUBLJANI\\[1mm]
%FAKULTETA ZA MATEMATIKO IN FIZIKO\\[5mm]
%\program\ -- 2.~stopnja}
% ustrezno dopolni za IŠRM
%\vfill

%\begin{center}
%  \large
%  \imeavtorja\\[3mm]
%  \Large
%  \textbf{\MakeUppercase{\naslovdela}}\\[10mm]
%  \large
%  Magistrsko delo \\[1cm]
%  Mentor: \imementorja \\[2mm] % ustrezno popravi spol
%   Somentor: \imesomentorja   % dodaj, če potrebno
%\end{center}
%\vfill

%\noindent{\large Ljubljana, \letnica}

%\cleardoublepage

%% IZJAVA O AVTORSTVU
%\pdfbookmark[1]{Izjava o avtorstvu}{izjava} % bookmark v PDF, \pdfbookmark[nivo]{text}{label}
%
%% izjava: po potrebi spremeni v žensko obliko
%\setlength\topsep{0pt}
%\setlength\parskip{0pt}
%\begin{center}
%  \textbf{Univerza v Ljubljani} \\
%  \textbf{Fakulteta za matematiko in fiziko}
%
%  \vfill
%
%  \underline{Izjava o avtorstvu, istovetnosti tiskane in elektronske verzije magistrskega dela in} \\
%  \underline{objavi osebnih podatkov študenta}
%
%  \vfill
%
%  \setlength\topsep{0pt}
%  \setlength\parskip{0pt}
%  \begin{flushleft}
%    Spodaj podpisani študent \imeavtorja{} avtor magistrskega dela (v nadaljevanju: pisnega
%    zaključnega dela študija) z naslovom:
%  \end{flushleft}
%
%  \vfill
%
%  \textbf{\naslovdela}
%
%  \vfill
%
%  IZJAVLJAM
%\end{center}
%
%\begin{enumerate}[1. ]
%  \item \emph{Obkrožite eno od variant a) ali b)}
%  \begin{enumerate}[a)]
%    \item da sem pisno zaključno delo študija izdelal samostojno;
%    \item da je pisno zaključno delo študija rezultat lastnega dela več kandidatov in izpolnjuje
%      pogoje, ki jih Statut UL določa za skupna zaključna dela študija ter je v zahtevanem deležu
%      rezultat mojega samostojnega dela;
%  \end{enumerate}
%  pod mentorstvom IZPOLNI. % dopiši \imementorja v rodilniku
%%   \\ in somentorstvom IZPOLNI. % dopiši \imesomentorja v rodilniku
%  \item da je tiskana oblika pisnega zaključnega dela študija istovetna elektronski obliki
%    pisnega zaključnega dela študija;
%  \item da sem pridobil vsa potrebna dovoljenja za uporabo podatkov in avtorskih del v pisnem
%    zaključnem delu študija in jih v pisnem zaključnem delu študija jasno označil;
%  \item da sem pri pripravi pisnega zaključnega dela študija ravnal v skladu z etičnimi načeli in,
%    kjer je to potrebno, za raziskavo pridobil soglasje etične komisije;
%  \item da soglašam, da se elektronska oblika pisnega zaključnega dela študija uporabi za preverjanje
%    podobnosti vsebine z drugimi deli s programsko  opremo za preverjanje podobnosti
%    vsebine, ki je povezana s študijskim informacijskim sistemom fakultete;
%  \item da na UL neodplačno, neizključno, prostorsko in časovno neomejeno prenašam pravico shranitve
%    avtorskega dela v elektronski obliki, pravico reproduciranja ter pravico dajanja pisnega
%    zaključnega dela študija na voljo javnosti na svetovnem spletu preko Repozitorija UL;
%  \item da dovoljujem objavo svojih osebnih podatkov, ki so navedeni v pisnem zaključnem delu študija
%    in tej izjavi, skupaj z objavo pisnega zaključnega dela študija.
%\end{enumerate}
%
%\vfill
%
%\noindent
%Kraj:  \hfill   Podpis študenta: \phantom{prostor za podpis}
%
%\vfill
%
%\noindent
%Datum:
%
%\cleardoublepage
%% END IZJAVA O AVTORSTVU

% zahvala
%\pdfbookmark[1]{Zahvala}{zahvala} %
%\section*{Zahvala}
%Neobvezno.
%Zahvaljujem se \dots
% end zahvala -- izbriši vse med zahvala in end zahvala, če je ne rabiš

%\cleardoublepage

\pdfbookmark[1]{\contentsname}{kazalo-vsebine}
\tableofcontents

% list of figures
% \cleardoublepage
% \pdfbookmark[1]{\listfigurename}{kazalo-slik}
% \listoffigures
% end list of figures

\cleardoublepage

%\section*{Program dela}
%\addcontentsline{toc}{section}{Program dela} % dodajmo v kazalo
%Mentor naj napiše program dela skupaj z osnovno literaturo. Na literaturo se
%lahko sklicuje kot~\cite{lebedev2009introduction}, \cite{gurtin1982introduction},
%\cite{zienkiewicz2000finite}, \cite{STtemplate}.

%\section*{Osnovna literatura}
%Literatura mora biti tukaj posebej samostojno navedena (po pomembnosti) in ne
%le citirana. V tem razdelku literature ne oštevilčimo po svoje, ampak uporabljamo
%okolje itemize in ukaz plancite, saj je celotna literatura oštevilčena na koncu.
%\begin{itemize}
%  \plancite{lebedev2009introduction}
%  \plancite{gurtin1982introduction}
%  \plancite{zienkiewicz2000finite}
%  \plancite{STtemplate}
%\end{itemize}

%\vspace{2cm}
%\hspace*{\fill} Podpis mentorja: \phantom{prostor za podpis}

% \vspace{2cm}
% \hspace*{\fill} Podpis somentorja: \phantom{prostor za podpis}

%\cleardoublepage
%\pdfbookmark[1]{Povzetek}{abstract}

%\begin{center}
%\textbf{\naslovdela} \\[3mm]
%\textsc{Povzetek} \\[2mm]
%\end{center}
%Tukaj napišemo povzetek vsebine. Sem sodi razlaga vsebine in ne opis tega, kako je delo
%organizirano.

%\vfill
%\begin{center}
%\textbf{English translation of the title} \\[3mm] % prevod slovenskega naslova dela
%\textsc{Abstract}\\[2mm]
%\end{center}

%An abstract of the work is written here. This includes a short description of
%the content and not the structure of your work.

%\vfill\noindent
%\textbf{Math.~Subj.~Class.~(2010):} oznake kot 74B05, 65N99, na voljo so na naslovu
%\url{http://www.ams.org/msc/msc2010.html?t=65Mxx} \\[1mm]
%\textbf{Ključne besede:} \kljucnebesede \\[1mm]
%\textbf{Keywords:} \keywords

%\cleardoublepage

\setcounter{page}{1}    % od sedaj naprej začni zopet z 1
\pagenumbering{arabic}  % in z arabskimi številkami

\section{Uvod}

\todo{začneš s tem, da bi radi različne oblike opisali s čim bolj enostavnimi elementi. 
v ta namen uporabljamo enostavne parametrične ploskve (zelo pogosto polinomske npr. 
bezierove), ki jih nato lepimo skupaj v kompleksnejše oblike. želimo, da bi bil stik 
dveh takih ploskev videti gladek, ploskvi morata biti zato prek skupne meje zvezni. 
predstaviš običajno zveznost, poveš, zakaj ni ustrezna}

\todo{lahko najprej poveš, kaj je c zveznost, potem pa navedeš primer, kjer c zveznost 
ne pride v poštev}

\todo{geometrijska zveznost je zelo uporabna v praksi, ker lahko modeliramo različne situacije, 
kjer c zveznost odpove (npr. zvezda, suitcase corner, house corner)}

\todo{je invarianta za parametrične transformacije, tj neodvisna od parametrizacije}

\todo{geometrijska zveznost je posplošitev c zveznosti. torej vse nedegenerirane 
(kaj to pomeni?) ploskve, ki so c zvezne, so tudi geometrisko zvezne, niso pa vse 
geometrijsko zvezne ploskve tudi c zvezne}

\todo{s čim se to delo ukvarja in kaj bo v kakšnem poglavju}

%It has been shown previously that parametric continuity is sufficient, but not necessary, for geometric smoothness
%geometric continuity as a parametrization independent measure
%Parametric splines a.re piecewise functions, so care must be taken to "stitch" the curve segments or surface patches together in a. "smooth" fashion
%The usual measure of smoothness, known as parametric continuity, requires
%the piecing together of curves a.nd surfaces so that a. given number of parametric
%derivatives match a.t the boundaries between curve segments or surface patches.
%The order of continuity (the number of derivatives that a.re required to match)
%is determined by the particular application. Although this generally results in
%splines that "look" smooth, it is shown in Chapter 2 that parametric continuity
%ca.n be overly restrictive since it depends upon details of the pa.ra.metriza.tions that
%a.re irrelevant for many CAGD applications.
%To remedy this situation, we define a. measure of continuity, known as
%geometric continuity, that is insensitive to changes in these irrelevant details. In
%other words, geometric continuity is a. parametrization independent measure of
%continuity.
%The shape
%parameters are degrees of freedom that are not available when using parametric
%continuity. The shape parameters can be made available to a designer in a CAG D
%environment. If the spline technique is based on the Beta constraints rather
%than the parametric continuity constraints, the shape parameters can be used
%to alter the shape of the curve or surface.

%%%%%%%%%%%%%%%%%%%%%%%%%%%%%%%%%%%%%%%%%%%%%%%%%%%%%%%%%%%%%%%%%%%%%%%%%%%%%%%%%%%%%%%%%
%%%%%%%%%%%%%%%%%%%%%%%%%%%%%%%%%%%%%%%%%%%%%%%%%%%%%%%%%%%%%%%%%%%%%%%%%%%%%%%%%%%%%%%%

\section{Geometrijska zveznost}\label{geom.zv.}

\todo{nek uvodni tekst?}
%obstaja več načinov za modeliranje krivulj in ploskev. eden izmed njih je 
%parametrično podane krivulje in ploskve
%ali: krivulje lahko podajamo na različne načine: eksplicitno, implicitno, parametrično. 
%ogledamo si definicijo parametrične.

%A univariate parametrization is said to be regular if the first derivative vector
%does not vanish. It is well known from differential geometry ( cf. DoCarmo [26])
%that regularity is, in general, essential for the smoothness of the resulting curve
%(see Figure 2.6). We therefore restrict the discussion to parametrizations that
%are regular.

\subsection{Geometriska zveznost krivulj}
\todo{tekst}
\todo{splošna definicija?}

\begin{definicija}
Naj bosta $\ddd{p}:I_1\rightarrow\R^d$ \todo{2? ali d?} in $\ddd{q}:I_2\rightarrow\R^d$ 
parametrično podani krivulji, kjer sta $I_1$ in $I_2$ intervala v $\R$. Krivulji 
$\ddd{p}$ in $\ddd{q}$ naj se stikata v neki točki, torej naj obstajata parametra 
$t_1\in I_1$ in $t_2\in I_2$, da je $\ddd{p}(t_1)=\ddd{q}(t_2)$. 
Pravimo, da se krivulji v skupni točki stikata z \emph{geometrijsko zveznostjo reda n} 
oziroma z \emph{$G^n$-zveznostjo}, če v okolici $I_0$ parametra $t_1$ obstaja regularna 
reparametrizacijska funkcija $f:I_0\rightarrow I_2$, da velja $f(t_1)=t_2$, 
$\frac{df}{dt}>0$ ter 
\begin{align*}
  \frac{d^k}{dt^k}\ddd{p}(t)\big|_{t=t_1}=\frac{d^k}{dt^k}(\ddd{q}\circ f)(t)\big|_{t=t_1},\quad k=0,1,\ldots,n.
\end{align*}
\end{definicija}
\todo{slika}

Da sta krivulji $\ddd{p}$ in $\ddd{q}$ $G^0$-zvezni, mora veljati le, da se ujemata 
v skupni točki, oziroma da je $\ddd{p}(t_1)=\ddd{q}(t_2)$. 
Za $G^1$ zveznost mora poleg pogoja za $G^0$-zveznost veljati še naslednje:
\begin{align*}
\frac{d}{dt}\ddd{p}(t_1)=\frac{d}{dt}(\ddd{q}\circ f)(t_1)=
\frac{d}{dt}(\ddd{q}\circ f)(t_1)\frac{df}{dt}(t_1).
\end{align*}
Za $G^2$ pa mora poleg pogojev za $G^0$-zveznost in $G^1$-zveznost veljati še 
\begin{align*}
\frac{d^2}{dt^2}\ddd{p}(t_1)=\frac{d^2}{dt^2}(\ddd{q}\circ f)(t_1)
\left(\frac{df}{dt}(t_1)\right)^2+\frac{d}{dt}(\ddd{q}\circ f)(t_1)\frac{d^2f}{dt^2}(t_1).
\end{align*}

V splošnem pogoj za $G^n$-zveznost krivulj zapišemo kot:
$$\frac{d^k}{dt^k}\ddd{p}=\sum_{i=1}^k\sum_{|m_i|=k}A_{m_i}^k\frac{d^{m_1}f}{dt^{m_1}}\cdots
\frac{d^{m_i}f}{dt^{m_i}}\frac{d^i}{dt^i}\ddd{q}$$

\subsection{Geometrijska zveznost ploskev}

Najprej si oglejmo povsem splošno definicijo geometrijske zveznosti neke ploskve.

\begin{definicija}
  Ploskev pripada razredu $G^n$ oziroma je \emph{geometrijsko zvezna z redom $n$}, če v okolici vsake njene točke obstaja lokalna regularna parametrizacija razreda $C^n$.
\end{definicija}

\todo{definicija regularne ploskve - lahko poveš definicijo}

\todo{potem lahko poveš, da to v praksi pomeni, da je normala na ploskev v vsaki točki 
neničelna (ali je to res?) ... ampak če hočeš to, moraš verjetno povedati, kako to sledi 
iz definicije (kako?)}

\todo{razloži lokalnost?}

V nadaljevanju se bomo ukvarjali s ploskvami, ki so same po sebi že geometrijsko zvezne, 
zanimalo nas bo le, kakšni pogoji morajo veljati, da je tudi stik dveh takih  
ploskev geometrijsko zvezen, torej da je celotna ploskev, ki jo dobimo, ko zlepimo dve ploskvi, 
geometrijsko zvezna. 

\begin{definicija}\label{def2}
  Naj bosta $\ddd{R}(x,y)$ in $\ddd{S}(u,v)$ regularni $C^n$ parametrizaciji dveh 
  ploskev, ki se stikata v krivulji $\ddd{C}(y)=\ddd{R}(x_0,y)=\ddd{S}(u_0,y)$. 
  Pravimo, da se $\ddd{R}$ in $\ddd{S}$ \emph{stikata z $G^n$-zveznostjo vzdolž 
  krivulje $\ddd{C}$}, če za vsako točko $b_0=\ddd{C}(y_0)$ obstaja lokalna regularna $C^n$ reparametrizacijska funkcija $f(x,y)=(u(x,y),v(x,y))$, da je $f(x_0,y)=(u_0,y)$ za vsak $y \in I_0$ in da velja
  \begin{equation}\label{eq0}
  \frac{\partial^{m+k}}{\partial x^m \partial y^k}\ddd{R}\Bigr|_{\substack{(x_0,y)}}=\frac{\partial^{m+k}}{\partial x^m \partial y^k}(\ddd{S}\circ f)\Bigr|_{\substack{(x_0,y)}}\quad \textrm{za } m+k=1,\ldots,n,
  \end{equation}
  kjer je $I_0$ neka okolica $y_0$. %bolj natančno, kaj je ta okolica
\end{definicija}

\todo{razloži, kaj je regularna reparametrizacijska funkcija? ali je dovolj, da je 
bila prej razložena regularnost?}

Zaradi stikanja ploskev v krivulji $\ddd{C}$ oziroma, ker vzdolž krivulje $\ddd{C}$ velja $y=v$, 
so parcialni odvodi parametrizacij 
po spremenljivki $y$ vzdolž krivulje $\ddd{C}$ enaki, zato je dovolj, da pri obravnavi 
geometrijske zveznosti dveh ploskev opazujemo le parcialne odvode po 
spremenljivki $x$.
\todo{dodati sliko?}
Te parcialne odvode imenujemo \todo{crossboundary derivatives}.

Oglejmo si, kakšni pogoji morajo veljati v primeru, ko želimo, da je stik med dvema 
ploskvama $G^2$ zvezen.
\begin{primer}
Naj bodo dane parametrizaciji ploskev $\ddd{R}$ in $\ddd{S}$, krivulja $\ddd{C}$ in 
reparametrizacijska 
funkcija $f$ kot v definiciji \ref{def2}.
Da bo stik teh dveh ploskev $G^2$ zvezen, mora po definiciji \ref{def2} in ugotovitvi, 
da je dovolj obravnavati le odvode po spremenljivki $x$, veljati
$$\frac{\partial^{k}}{\partial x^k}\ddd{R}\Bigr|_{\substack{(x_0,y)}}=\frac{\partial^{k}}{\partial x^k}(\ddd{S}\circ f)\Bigr|_{\substack{(x_0,y)}}\quad \textrm{za } k=0,1,2,$$
za vsak $y$ v neki okolici točke $y_0$. 
Da dosežemo zgolj geometrijsko zveznost razreda $G^0$, je dovolj, da med 
ploskvama $\ddd{R}$ in $\ddd{S}$ velja pogoj 
$$\ddd{R}(x_0,y)=(\ddd{S} \circ f)(x_0,y)\textrm{, oziroma }\ddd{R}(x_0,y)=\ddd{S}(u(x_0,y),v(x_0,y)).$$ 
Da imamo na stiku geometrijsko zveznost stopnje $G^1$, mora poleg 
pogoja za $G^0$ veljati še 
$$\frac{\partial \ddd{R}}{\partial x}\Bigr|_{\substack{(x_0,y)}}=\frac{\partial}{\partial x}(\ddd{S}\circ f)\Bigr|_{\substack{(x_0,y)}}.$$
Če ustrezno razpišemo parcialni odvod funkcije $\ddd{S} \circ f$, se ta pogoj prepiše v
$$\frac{\partial \ddd{R}}{\partial x}\Bigr|_{\substack{(x_0,y)}}=
\frac{\partial \ddd{S}}{\partial u}\frac{\partial u}{\partial x}\Bigr|_{\substack{(x_0,y)}}+
\frac{\partial \ddd{S}}{\partial v}\frac{\partial v}{\partial x}\Bigr|_{\substack{(x_0,y)}}.$$
Za $G^2$ mora poleg pogojev za $G^0$ in $G^1$ veljati še
\begin{align*} 
\frac{\partial^2 \ddd{R}}{\partial x^2}\bigg|_{\substack{(x_0,y)}} &=
\frac{\partial^2 \ddd{S}}{\partial u^2}\bigg(\frac{\partial u}{\partial x}\bigg)^2\bigg|_{\substack{(x_0,y)}} + 
2\frac{\partial^2 \ddd{S}}{\partial u \partial v}\frac{\partial u}{\partial x}\frac{\partial v}{\partial x}\bigg|_{\substack{(x_0,y)}} + 
\frac{\partial^2 \ddd{S}}{\partial v^2}\bigg(\frac{\partial v}{\partial x}\bigg)^2\bigg|_{\substack{(x_0,y)}} + \\
&+\frac{\partial \ddd{S}}{\partial u}\frac{\partial^2 u}{\partial x^2}\bigg|_{\substack{(x_0,y)}} +
\frac{\partial \ddd{S}}{\partial v}\frac{\partial^2 v}{\partial x^2}\bigg|_{\substack{(x_0,y)}}.
\end{align*}
\end{primer}

V splošnem se pogoj za geometrijsko zveznost stopnje $n$, kjer je $n\in \N_0$, zapiše kot:
\begin{equation}\label{eq1}
\frac{\partial^k \ddd{R}}{\partial x^k}\bigg|_{\substack{C}}=
\sum_{i=1}^k \sum_{|m_i|=k}A_{m_i}^k
\sum_{h=0}^i{i \choose h}\frac{\partial^{m_1}u}{\partial x^{m_1}}\cdots 
\frac{\partial^{m_h}u}{\partial x^{m_h}}
\frac{\partial^{m_{h+1}}v}{\partial x^{m_{h+1}}}\cdots \frac{\partial^{m_i}v}{\partial x^{m_i}}
\frac{\partial^i \ddd{S}}{\partial u^h \partial v^{i-h}}\bigg|_{\substack{C}}
\end{equation}
za vsak $k=0,1,\ldots,n$. Tu z $A_{m_i}^k$ iznačujemo koeficient
$$A_{m_i}^k = \frac{k!}{i!m_1!\cdots m_i!}.$$
Z $u_x^{m_i}$ je označen $m_i$-ti delni odvod funkcije $u$ po $x$, z oznako $|m_i|$ pa 
označimo vsoto $|m_i|=m_1+m_2+\cdots + m_i$, kjer velja $m_j>0$ za vsak $j=1,\ldots, i$.

\todo{dokaz z indukcijo?}

Taka definicija geometrijske zveznosti med dvema ploskvama sama po sebi pri konstrukciji geometrijsko 
zveznih ploskev ni najbolj koristna. V nadaljevanju bo veliko uporabnejši naslednji izrek.

\begin{izrek}\label{izrek1}
  Naj bosta $\ddd{R}(x,y)$ in $\ddd{S}(u,v)$ regularni $C^n$ parametrizaciji dveh 
  ploskev, ki se stikata v krivulji $\ddd{C}(y)=\ddd{R}(x_0,y)=\ddd{S}(u_0,y)$. 
  Ploskvi $\ddd{R}$ in $\ddd{S}$ sta vzdolž skupnega roba $G^n$-zvezni natanko 
  tedaj, ko obstajajo 
  $C^n$ funkcije $\alpha_1(y), \ldots, \alpha_n(y)$ in $\beta_1(y) \ldots, \beta_n(y)$, 
  da velja 
  \begin{equation}\label{eq2}
  \frac{\partial^k \ddd{R}}{\partial x^k}\bigg|_{\substack{\ddd{C}}}=\sum_{i=1}^k \sum_{|m_i|=k}A_{m_i}^k
  \sum_{h=0}^i{i \choose h}\alpha_{m_1}\cdots \alpha_{m_h}\beta_{m_{h+1}}\cdots \beta_{m_i}
  \frac{\partial^i \ddd{S}}{\partial u^h \partial v^{i-h}}\bigg|_{\substack{\ddd{C}}},
  \end{equation} 
  za $k=1,\ldots,n$, 
  kjer je $A_{m_i}^k = \frac{k!}{i!m_1!\cdots m_i!}$.
  Veljati mora tudi, da je $\alpha_1(y)\neq 0$ \todo{in predznak}
\end{izrek}

\begin{opomba}
  Funkcije $\alpha_1(y), \ldots, \alpha_n(y)$ in $\beta_1(y) \ldots, \beta_n(y)$ 
  imenujemo \emph{stične funkcije} \todo{junction/connection functions}
\end{opomba}

\begin{proof}
  \todo{kaj je z lokalnostjo in b0?}
  Najprej prodpostavimo, da obstajajo $C^n$ funkcije $\alpha_1(y), \ldots, \alpha_n(y)$ 
  in $\beta_1(y) \ldots, \beta_n(y)$, za katere velja enakost \eqref{eq2} v izreku, in 
  da je $\alpha_1(y)\neq 0$. 
  Dokazati želimo, da od tod 
  sledi $G^n$-zveznost stika ploskev $\ddd{R}(x,y)$ in $\ddd{S}(u,v)$. 

  Definirajmo reparametrizacijsko funkcijo $f(x,y)=(u(x,y),v(x,y))$ na naslednji način: 
  $$u(x,y) =  u_0 + \sum_{i=0}^n \frac{1}{i!}\alpha_i(y)(x-x_0)^i,$$
  $$v(x,y) = y + \sum_{i=0}^n \frac{1}{i!}\beta_i(y)(x-x_0)^i.$$
  
  Ker so po predpostavki funkcije $\alpha_i$ in $\beta_i$, $i=1,\ldots,n$ razreda $C^n$, 
  tudi funkcija $f$ pripada temu razredu. 
  Opazimo še, da za $i=1, \ldots, k$ velja
  \begin{align*}
  \frac{\partial^i u}{\partial x^i}(x_0,y)=\alpha_i(y),&&
  \frac{\partial^i v}{\partial x^i}(x_0,y)=\beta_i(y).
  \end{align*}
  Če dobljeno vstavimo v enačbo \eqref{eq2}, dobimo ravno enačbo \eqref{eq1}, od koder zaradi 
  ujemanja ploskev $\ddd{R}$ in $\ddd{S}$ v krivulji $\ddd{C}$ sledi tudi enakost \eqref{eq0}. Pokazati moramo 
  le še, da je $f$ lokalno regularna.
  
  Vemo, da je reparametrizacijska funkcija $f$ regularna vzdolž $\ddd{C}$, če sta oba njena parcialna odvoda prvega 
  reda linearno neodvisna, torej če velja 
  $$\frac{\partial f}{\partial x}(x_0,y)\times \frac{\partial f}{\partial y}(x_0,y) \neq 0.$$

  Razpišimo oba odvoda reparametrizacijske funkcije $f(x,y)=(u(x,y),v(x,y))$ vzdolž krivulje $\ddd{C}$ 
  in ju skušajmo zapisati s pomočjo stičnih funkcij. Za odvod po spremenljivki $x$ velja
  $$\frac{\partial f}{\partial x}(x_0,y)=\left(\frac{\partial u}{\partial x}(x_0,y),\frac{\partial v}{\partial x}(x_0,y)\right)=(\alpha_1(y),\beta_1(y)).$$
  Če razpišemo odvod po spremenljivki $y$, pa dobimo
  $$\frac{\partial f}{\partial y}(x_0,y)=\left(\frac{\partial u}{\partial y}(x_0,y),\frac{\partial v}{\partial y}(x_0,y)\right)=(0,1).$$

  Vektorski produkt $\frac{\partial f}{\partial x}(x_0,y)\times \frac{\partial f}{\partial y}(x_0,y)$
  je torej enak 
  $$\frac{\partial f}{\partial x}(x_0,y)\times \frac{\partial f}{\partial y}(x_0,y) = 
  (\alpha_1(y),\beta_1(y))\times (0,1) = \alpha_1(y).$$
  Vektorski produkt obeh parcialnih odvodov prvega reda je torej različen od $0$ 
  natanko tedaj, ko je $\alpha_1(y)\neq 0$, kar pa velja po začetni predpostavki.
  Sledi, da je reparametrizacijska funkcija $f$  regularna. 

  Pokazali smo torej, da obstaja lokalno regularna reparametrizacijska funkcija $f$, ki 
  ustreza pogojem iz definicije \ref{def2}, od koder sledi, da se ploskvi $\ddd{R}$ in $\ddd{S}$ stikata 
  z $G^n$-zveznostjo.

  Dokažimo izrek še v drugo smer. 
  Če predpostavimo, da sta ploskvi $R$ in $S$ na stiku $G^n$-zvezni, 
  obstoj funkcij $\alpha_1(y), \ldots, \alpha_n(y)$ in $\beta_1(y) \ldots, \beta_n(y)$
  in enačba \eqref{eq2} sledjo neposredno iz definicije \ref{def2} in enačbe \eqref{eq1}, 
  če definiramo $\alpha_i(y)=\frac{\partial^iu}{\partial{x^i}}(x_0,y)$ in 
  $\beta_i(y)=\frac{\partial^iv}{\partial {x^i}}(x_0,y)$ za $i=1,\ldots,n$. 
  
  Ker je stik ploskev $\ddd{R}$ in $\ddd{S}$ $G^n$-zvezen, po definiciji \ref{def2} obstaja 
  lokalno regularna reparametrizacijska funkcija $f(x,y)=(u(x,y),v(x,y))$, ki ustreza 
  pogojem iz definicije \ref{def2}. Videli smo že, da je funkcija $f$ regularna 
  natanko tedaj, ko je $\frac{\partial u}{\partial x}(x_0,y)=\alpha_1(y)\neq 0$. 
  Torej smo okazali še neničelnost funkcije $\alpha_1$, s čimer je dokaz končan. 
\end{proof}

\todo{Drugo, na kar moramo paziti pri izbiri funkcije $\alpha_1$ pa je njen predznak. 
Pri izbiri napačnega predznaka namreč lahko pride do stika v obliki "špice".} 

\todo{nekaj za zaključek poglavja in napeljavo na novo poglavje}

%%%%%%%%%%%%%%%%%%%%%%%%%%%%%%%%%%%%%%%%%%%%%%%%%%%%%%%%%%%%%%%%%%%%%%%%%%%%%%%%%%%%%%%%%%%%%%55

\subsection{Geometrijska interpretacija $G^1$-zveznosti}
\todo{ali naredim še za $G^2$ zveznost}

V nadaljevanju se bomo nekoliko natančneje ukvarjali z $G^1$-zveznostjo med ploskvami. 
V tem podpoglavju si oglejmo geometrijsko interpretacijo $G^1$-zveznosti, ki bo 
kasneje služila tudi v praktičnih primerih.

Imejmo regularni $C^1$ parametrizaciji ploskev $\ddd{R}(x,y)$ in $\ddd{S}(u,v)$, ki 
se v krivulji $\ddd{C}(y) = \ddd{R}(x_0,y) = \ddd{S}(u_0,y)$ 
stikata z geometrijsko zveznostjo $G^1$.  
Sledi, da je $\ddd{R}_y(x_0,y) = \ddd{S}_y(x_0,y) = \ddd{S}_v(x_0,y)$. Kot smo že videli v poglavju \ref{geom.zv.}, 
nam je zato potrebno opazovati zgolj odvode v smeri $x$.

Ker je stik obeh ploskev v $\ddd{C}$ $G^1$-zvezen, po izreku \ref{izrek1} obstajata 
funkciji $\alpha_1$ in $\beta_1$, kjer je $\alpha_1(y) \neq 0$ za vsak $y$ 
in ima ustrezen predznak, da velja:

\begin{equation}\label{eq3}
\ddd{R}_x(x_0,y)=\alpha_1(y) \ddd{S}_u(u_0,y)+\beta_1(y)\ddd{S}_v(u_0,y).
\end{equation}

Zgornja enačba nam pove, da parcialni odvodi $\ddd{R}_x(x_0,y)$, $\ddd{S}_u(u_0,y)$ in 
$\ddd{S}_v(u_0,y)$ v vsaki točki $y$ ležijo 
v isti tangentni ravnini na krivuljo $\ddd{C}$. Tangentna ravnina na ploskev $\ddd{R}$ 
se na krivulji $\ddd{C}$ ujema s tangentno ravnino na ploskev $\ddd{S}$, kar pomeni, 
da se tangentna ravnina zvezno spreminja vzdolž zlepka obeh ploskev. 
Zato torej $G^1$-zveznost imenujemo tudi 
\emph{zveznost tangentnih ravnin}.

\todo{slika}

Še eno ime za $G^1$-zveznost je \emph{zveznost enotskih normal}.
Oglejmo si, od kod pride to poimenovanje. 
Znova opazujemo enačbo \eqref{eq3}. 
Enačbo sedaj z obeh strani vektorsko pomnožimo z $\ddd{R}_y(x_0,y)$:
$$\ddd{R}_x(x_0,y)\times \ddd{R}_y(x_0,y)=\alpha_1(y) \ddd{S}_u(u_0,y)\times \ddd{R}_y(x_0,y)+\beta_1(y)\ddd{S}_v(u_0,y)\times \ddd{R}_y(x_0,y).$$
Upoštevamo, da je $\ddd{R}_y(x_0,y)=\ddd{S}_v(u_0,y)$ in dobimo:
\begin{equation}\label{eq30}
\ddd{R}_x(x_0,y) \times \ddd{R}_y(x_0,y) = \alpha_1(y) \ddd{S}_u(u_0,y) \times \ddd{S}_v(u_0,y).
\end{equation}
Vektorski produkt $\ddd{R}_x(x_0,y) \times \ddd{R}_y(x_0,y)$ predstavlja normalo na 
ploskev $\ddd{R}$ v točki $(x_0,y)$ na krivulji $\ddd{C}$, 
$\ddd{S}_u(u_0,y) \times \ddd{S}_v(u_0,y)$ pa normalo na ploskev $\ddd{S}$ na 
krivulji $\ddd{C}$. 
Enačba \eqref{eq30} pove, da sta normali na ploskvi $\ddd{R}$ in $\ddd{S}$ na njunu skupni  
krivulji $\ddd{C}$ vzporedni. Na skupnem robu imata torej obe ploskvi enaki 
enotski normali:
$$\frac{\ddd{R}_x(x_0,y) \times \ddd{R}_y(x_0,y)}{||\ddd{R}_x(x_0,y) \times \ddd{R}_y(x_0,y)||}=\frac{\ddd{S}_u(u_0,y) \times \ddd{S}_v(u_0,y)}{||\ddd{S}_u(u_0,y) \times \ddd{S}_v(u_0,y)||}.$$
Enotska normala se zato zvezno spreminja vzdolž zlepka obeh ploskev, od koder poimenovanje 
\emph{zveznost enotskih normal}. 

Pogoj, da parcialni odvodi $\ddd{R}_x(x_0,y)$, $\ddd{S}_u(u_0,y)$ in $\ddd{S}_v(u_0,y)$ ležijo na isti 
tangentni ravnini, je možno izraziti tudi na naslednja dva načina. 
Zgornji parcialni odvodi so del iste tangentne ravnine natanko tedaj ko velja
$$\det(\ddd{R}_x(x_0,y), \ddd{S}_u(u_0,y), \ddd{S}_v(u_0,y))=0.$$

Ta pogoj pa je ekvivalenten pogoju, da obstajajo skalarne funkcije $\lambda$, 
$\mu$ in $\gamma$, da velja
$$\lambda(y)\ddd{R}_x(x_0,y)=\mu(y)\ddd{S}_u(u_0,y)+\gamma(y)\ddd{S}_v(u_0,y).$$

Če predpostavimo, da sta ploskvi $\ddd{R}$ in $\ddd{S}$ polinomski, lahko tudi za $\lambda$, $\mu$ in $\gamma$ 
izberemo polinome ali racionalne funkcije, kar nam zelo olajša 
konstrukcijo geometrijsko zveznih ploskev.

V nadaljevanju tega dela se bomo ukvarjali z vrsto parametričnih ploskev, 
imenovano Bézierjeve ploskve, in s pogoji, ki morajo veljati zanje, da je stik 
med njimi geometrijsko zvezen. 

%%%%%%%%%%%%%%%%%%%%%%%%%%%%%%%%%%%%%%%%%%%%%%%%%%%%%%%%%%%%%%%%%%%%%%%%%%%%%%%%%%%%%%%%%
%%%%%%%%%%%%%%%%%%%%%%%%%%%%%%%%%%%%%%%%%%%%%%%%%%%%%%%%%%%%%%%%%%%%%%%%%%%%%%%%%%%%%%%%%

\section{Konstrukcija geometrijsko zveznih Bezierjevih ploskev 
iz tenzorskega produkta}

\subsection{Bézierjeve ploskve iz tenzorskega produkta}

\todo{povej, v katerem viru so dokazi itd.}

V tem podpoglavju bodo predstavljene Bézierjeve krivulje in Bézierjeve ploskve iz 
tenzorskega produkta. Te podajamo s pomočjo Bernsteinovih baznih polinomov.

\begin{definicija}
Za $i=0,1,\ldots,n$ je $i$-ti \emph{Bérnsteinov bazni polinom} stopnje $n$ definiran 
kot 
$$B_i^n(t)={n \choose i}t^i (1-t)^{n-i},\quad t\in[0,1].$$
\end{definicija}


\begin{izrek}
Bernsteinovi bazni polinomi sestavljajo bazo prostora polinomov
$\mathbb{P}_n=\mbox{Lin}\{1,t,\\t^2,\ldots,t^n\}$.
\end{izrek}

Bazo prostora polinomov dveh spremenljivk 
$\mathbb{P}_m \otimes \mathbb{P}_n = \mbox{Lin}\{u^iv^j,\\i=0,\ldots,m, j=0,\ldots,n\}$, 
torej tenzorskega produkta prostorov $\mathbb{P}_m$ in $\mathbb{P}_n$, pa sestavljajo 
produkti po dveh Bernsteinovih polinomov, kjer prvi pripada $\mathbb{P}_m$ in 
drugi $\mathbb{P}_n$, oziroma tenzorski produkti dveh Bernsteinovih polinomov. 

O ploskvah iz tenzorskega produkta govorimo, kadar je njihova parametrizacija 
podana v bazi iz tenzorskih produktov. Bezierove ploskve iz tenzorskega 
produkta so definirane na naslednji način.

\begin{definicija}
  Naj bodo dane točke $\mathbf{b}_{i,j}\in \R^d$, $i=0,1,\ldots,m$, $j=0,1,\ldots,n$. 
  \emph{Bézierjeva ploskev iz tenzorskega produkta} je parametrično podana ploskev
  $$\mathbf{B}_{m,n} : [0,1]\times[0,1] \rightarrow \R^d$$
  s predpisom
  $$\mathbf{B}_{m,n}(u,v)=\sum_{i=0}^m \sum_{j=0}^n \mathbf{b_{i,j}} B_i^m(u) B_j^n(v).$$
  Točke $\mathbf{b}_{i,j}$ imenujemo \emph{kontrolne točke}, mrežo, ki jih povezuje, 
  pa \emph{kontrolna mreža} ploskve $\ddd{B}_{m,n}$. 
\end{definicija}

\todo{slika?}

Opazimo lahko, da za Bézierovo ploskev iz tenzorskega produkta velja naslednje: 
$\mathbf{B_{m,n}}(0,0)=\mathbf{b}_{0,0}$, $\mathbf{B_{m,n}}(1,0)=\mathbf{b}_{m,0}$, 
$\mathbf{B_{m,n}}(0,1)=\mathbf{b}_{0,n}$ in $\mathbf{B_{m,n}}(1,1)=\mathbf{b}_{m,n}$. 
Parametrizacija torej interpolira vogalne kontrolne točke. 

\todo{ali rabim decasteljaujev algoritem?}

Prerez ploskve pri neki vrednosti $u=u_0$ ali $v=v_0$ je Bézierova krivulja. 
Njihove parametrizacije so podane v bazi Bernsteinovih polinomov, definirane 
pa so na naslednji način.

\begin{definicija}
Naj bodo dane točke $\ddd{b}_i\in \R^d, i=0,1,\ldots,n$. \emph{Bézierova krivulja} 
je parametrično podana krivulja 
$$\ddd{B}_n:[0,1] \rightarrow \R^d$$
s predpisom
$$\ddd{B}_n(t)=\sum_{i=0}^n \ddd{b}_i B_i^n(t).$$
Točke $\ddd{b}_i$ imenujemo \emph{kontrolne točke krivulje}.
\end{definicija}
Tudi za Bézierjeve krivulje velja, da interpolirajo prvo in zadnjo kontrolno točko. 

V nekaterih primerih v nadaljevanju bo, da dosežemo ujemanje stopenj Bézierjevih 
krivulj v enačbah, potrebno zvišati oziroma znižati stopnjo Bézierjeve krivulje, 
torej opisati oziroma aproksimirati dano krivuljo s krivuljo višje oziroma nižje 
stopnje. Naj bodo $\ddd{b}_0,\ddd{b}_1,\ldots,\ddd{b}_n$ kontrolne točke Bézierjeve 
krivulje stopnje $n$. To krivuljo lahko opišemo s krivuljo stopnje $n+k$, kjer je 
$k\in \N$, njene kontrolne točke pa se izražajo na naslednji način:
\begin{align}\label{eq42}
\ddd{b}_i^{(k)}=\sum_{j=0}^n\ddd{b}_j{n \choose j}\frac{{k \choose i-j}}{{n+k \choose i}}.
\end{align}

V nadaljevanju, ko bomo preučevali geometrijsko zveznost Bézierovih ploskev 
iz tenzorskega produkta, bomo imeli opravka predvsem z odvodi Bézierovih krivulj 
in ploskev. 
Imejmo Bézierovo krivuljo $\ddd{B}_n(t)=\sum_{i=0}^n\ddd{b}_iB_i^n(t)$ in naj 
bo $r\in \N$, $r\leq n$. Potem se $n$-ti odvod te krivulje izračuna 
kot 
$$\frac{d^r}{dt^r}\ddd{B}_n(t)=\frac{n!}{(n-r)!}\sum_{i=0}^{n-r}\Delta^r\ddd{b}_iB_i^{n-r}(t).$$ 
Z $\Delta$ označujemo operator, ki deluje na kontrolnih točkah in se izraža 
rekurzivno: \\
$\Delta^0\ddd{b}_i=\ddd{b}_i$,\\
$\Delta^1\ddd{b}_i=\ddd{b}_{i+1}-\ddd{b}_i$,\\
$\Delta^k\ddd{b}_i=\Delta(\Delta^{k-1}\ddd{b}_i)$, kjer je $k\in\N$, $k>1$.\\

Odvod Bézierjeve ploskve iz tenzorskega produkta \\
$\ddd{B}_{m,n}(u,v)=\sum_{i=0}^m\sum_{j=0}^n\ddd{b}_{i,j}B_i^m(u)B_j^n(v)$ se izraža na podoben način:\\
\begin{align}\label{eq28}
\frac{\partial^{r+s}}{\partial u^r \partial v^s}\mathbf{b}^{m,n}(u,v)=\frac{m!}{(m-r)!}\frac{n!}{(n-s)!}\sum_{i=0}^{m-r} \sum_{j=0}^{n-s} \Delta^{r,s}\mathbf{b}_{i,j}B_i^{m-r}(u)B_j^{n-s}(v),
\end{align}
Tu z $\Delta$ ponovno označujemo rekurzivno definirani operator na kontrolnih točkah:
$\Delta^{1,0} \mathbf{b}_{i,j} = \mathbf{b}_{i+1,j}-\mathbf{b}_{i,j}$,\\
$\Delta^{0,1} \mathbf{b}_{i,j} = \mathbf{b}_{i,j+1}-\mathbf{b}_{i,j}$,\\
$\Delta^{r,0} \mathbf{b}_{i,j} = \Delta^{r-1,0} \mathbf{b}_{i+1,j}-\Delta^{r-1,0} \mathbf{b}_{i,j}$,\\
$\Delta^{0,s} \mathbf{b}_{i,j} = \Delta^{0,s-1} \mathbf{b}_{i,j+1}-\Delta^{0,s-1} \mathbf{b}_{i,j}$.
%lahko poveš, da so to v bistvu vektorji
%posebej lahko napišeš (ali dodaš) odvode 1. stopnje
%dodaj pogoje gladkosti za C1 ?

%%%%%%%%%%%%%%%%%%%%%%%%%%%%%%%%%%%%%%%%%%%%%%%%%%%%%%%%%%%%%%%%%%%%%%%%%%%%%%%%%%%%%%%%%%%%%%%%%

\subsection{$C^n$-zveznost med dvema Bézierjevima ploskvama 
iz tenzorskega produkta}\label{cn.zv.}

Najprej si oglejmo, kakšni pogoji morajo veljati za dve Bézierjevi ploskvi, da 
je njun stik $C^n$-zvezen. Dobljeni rezultat bomo v nadaljevanju primerjali s 
pogoji za $G^n$-zveznost. 

Imejmo dve Bézierjevi ploskvi 
$$\ddd{R}(u,v)=\sum_{i=0}^m\sum_{j=0}^n\ddd{r}_{i,j}B_i^m\left(\frac{u-u_1}{u_0-u_1}\right)B_j^n\left(\frac{v-v_0}{v_1-v_0}\right)\textrm{, }\quad
u\in[u_0,u_1]\textrm{, }v\in[v_0,v_1],$$
in 
$$\ddd{S}(u,v)=\sum_{i=0}^m\sum_{j=0}^n\ddd{s}_{i,j}B_i^m\left(\frac{u-u_1}{u_2-u_1}\right)B_j^n\left(\frac{v-v_0}{v_1-v_0}\right)\textrm{, }\quad
u\in[u_1,u_2]\textrm{, }v\in[v_0,v_1].$$

Ploskvi naj se stikata v krivulji  $\ddd{C}(v)=\ddd{R}(u_1,v)=\ddd{S}(u_1,v)$, 
torej naj velja $\ddd{r}_{0,j}=\ddd{s}_{0,j}$ za 
$j=0,\ldots,n$. S tem dosežemo $C^0$-zveznost. Da se bosta ploskvi stikali s 
$C^r$-zveznostjo, se morajo ujemati njuni odvodi pri $u=u_1$. \todo{!?!?} 
do odvoda stopnje $r$. Zaradi stikanja v krivulji $\ddd{C}$ je dovolj obravnavati le 
odvode po spremenljivki $u$. Veljati mora
$$\frac{\partial^k}{\partial u^k}\ddd{R}(u,v)|_{u=u_0}=\frac{\partial^k}{\partial u^k}\ddd{S}(u,v)|_{u=u_0},
\textrm{ }k=1,\ldots,r.$$
Uporabimo formulo \eqref{eq28} za odvod Bézierjeve ploskve iz tenzorskega produkta: 
\begin{align*}
\frac{n!}{(n-k)!}&\frac{1}{(u_0-u_1)^k}\sum_{i=0}^{m-k}\sum_{j=0}^n\Delta^{k,0}
\ddd{r}_{i,j}B_i^{m-k}\left(\frac{u-u_1}{u_0-u_1}\right)B_i^n\left(\frac{v-v_0}{v_1-v_0}\right)|_{u=u_1}=\\
&=\frac{n!}{(n-k)!}\frac{1}{(u_2-u_1)^k}\sum_{i=0}^{m-k}\sum_{j=0}^n\Delta^{k,0}
\ddd{s}_{i,j}B_i^{m-k}\left(\frac{u-u_1}{u_2-u_1}\right)B_i^n\left(\frac{v-v_0}{v_1-v_0}\right)|_{u=u_1},\\
k=1,&\ldots,r.
\end{align*}\todo{zamiki}
Sledi
\begin{align*}
&\frac{1}{(u_0-u_1)^k}\sum_{j=0}^n\Delta^{k,0}
\ddd{r}_{0,j}B_i^n\left(\frac{v-v_0}{v_1-v_0}\right)=
\frac{1}{(u_2-u_1)^k}\sum_{j=0}^n\Delta^{k,0}
\ddd{s}_{0,j}B_i^n\left(\frac{v-v_0}{v_1-v_0}\right),\\
&k=1,\ldots,r.
\end{align*}
Za vsak $j=1,\ldots,n$ primerjajmo koeficiente pred baznim polinomom $B_j^n(\frac{v-v_0}{v_1-v_0})$ 
in dobimo pogoje, ki morajo veljati med kontrolnimi točkami dveh ploskev, da sta na 
stiku $C^r$-zvezni:
\begin{align}\label{eq29}
  \frac{1}{(u_0-u_1)^k}\Delta^{k,0}\ddd{r}_{0,j}=\frac{1}{(u_2-u_1)^k}\sum_{j=0}^n\Delta^{k,0}
  \ddd{s}_{0,j},\textrm{ }k=1,\ldots,r.
\end{align}

%%%%%%%%%%%%%%%%%%%%%%%%%%%%%%%%%%%%%%%%%%%%%%%%%%%%%%%%%%%%%%%%%%%%%%%%%%%%%%%%%%%%%%%%%%%%%%%5

\subsection{$G^n$-zveznost med dvema Bézierovima\\ ploskvama iz tenzorskega produkta}

V tem podpoglavju si bomo ogledali, kako se splošni pogoji za geometrijsko 
zveznost med dvema ploskvama, ki smo jih izpeljali v poglavju \ref{geom.zv.}, 
odražajo na Bézierjevih ploskvah iz tenzorskega produkta. 

Ker so Bézierjeve ploskve polinomske, se lahko pri izbiri stičnih funkcij 
omejimo na racionalne funkcije, kar močno olajša konstrukcijo v praktičnih primerih. 
O tem bo govoril izrek, predstavljen v tem podpoglavju.

Imejmo dve polinomski Bézierjevi ploskvi $\textbf{R}$ in $\textbf{S}$, podani na 
naslednji način: 
$$\textbf{R}(x,y)=\sum_{i=0}^{m_r}\sum_{j=0}^{n_r}\textbf{p}_{ij}B_i^{m_r}(x)B_j^{n_r}(y)$$
$$\textbf{S}(u,v)=\sum_{i=0}^{m_s}\sum_{j=0}^{n_s}\textbf{p}_{ij}B_i^{m_s}(u)B_j^{n_s}(v),$$
kjer so $\{\textbf{p}_{i,j}, i=1,\ldots, m_r, j=1, \ldots, n_r\}$ in 
$\{\textbf{q}_{i,j}, i=1, \ldots, m_s, j=1, \ldots, n_s\}$ kontrolne točke ploskev 
$\textbf{R}$ in $\textbf{S}$, $x, y, u$ in $v$ pa parametri z vrednostmi 
na intervalu $[0,1]$.

Ploskvi $\textbf{R}$ in $\textbf{S}$ naj se stikata v skupni robni krivulji
$\textbf{C}(v)=\textbf{R}(0,v)=\textbf{S}(0,v)$
\todo{pojasni, zakaj lahko to predpostavimo. ker lahko parametriziramo? poglej.} 
\todo{pojasni še, kako je s tem, da sta ploskvi prav obrnjeni, da ni špice} 
Robno krivuljo $\textbf{C}$ zapišemo kot Bézierjevo krivuljo na naslednji način:
$$\textbf{C}=\sum_{i=0}^{n_c}\textbf{z}_i B_i^{n_c},$$
kjer so $\{\textbf{z}_i, i=1,\ldots n_c\}$ njene kontrolne točke. Stopnja $n_c$ krivulje 
$\textbf{C}$ ni nujno enaka stopnjama $n_r$ ali $n_s$, velja pa, da je $n_c \leq \min(n_r,n_s)$. 

Naj bosta ploskvi $\textbf{R}$ in $\textbf{S}$ regularni vzdolž krivulje $\textbf{C}$, 
torej naj bodo normale na ploskvi vzdolž krivulje $\textbf{C}$ neničelne:
$$N_R=\bigg(\frac{\partial \textbf{R}}{\partial x}\times \frac{\partial \textbf{R}}{\partial y}\bigg)\bigg|_{\textbf{C}}\neq 0$$
$$N_S=\bigg(\frac{\partial \textbf{S}}{\partial u}\times \frac{\partial \textbf{S}}{\partial v}\bigg)\bigg|_{\textbf{C}}\neq 0$$

O pogojih za geometrijsko zveznost teh dveh ploskev govori naslednji izrek. 

\begin{izrek}\label{izrek2}
  Naj bosta $\textbf{R}$ in $\textbf{S}$ zgoraj definirani Bézierjevi 
  ploskvi, ki se stikata v robni krivulji $\textbf{C}$. Stik ploskev je 
  $G^n$-zvezen natanko tedaj, ko obstajajo polinomi $D(y)$, $E_i(y)$ in $F_i(y)$, da 
  velja
  \begin{equation}\label{eq4}
    \begin{split}
      D^{2k-1}(y)\frac{\partial^k\textbf{S}}{\partial u^k}(0,y) = &
      \sum_{i=0}^k \sum_{|\ddd{m_i}|=k} A_{\ddd{m_i}}^k \sum_{h=0}^i {i\choose h} D^{i-1}(y)E_{m_1}(v)
      \cdots E_{m_h}(y) \\
      &\cdot F_{m_{h+1}}(y)\cdots F_{m_i}(y) \frac{\partial^i \textbf{R}}{\partial x^h \partial y^{i-h}}(0,y),
    \end{split}  
  \end{equation}
  kjer je $i=1, \ldots n$ in $k=1, \ldots n$. Z $A_{m_i}^k$ zopet označujemo 
  $A_{\mathbf{m_i}}^k = \frac{k!}{i!m_1!\cdots m_i!}$ in $|\mathbf{m_i}|=m_1+m_2+\cdots+m_i$.
  Velja še $D(y)E_1(y)\neq 0$ za $y\in [0,1]$, za stopnje polinomov pa velja\\
  \begin{align*}
  \mbox{deg}(D)&\leq n_r + n_c -1,\\
  \mbox{deg}(E_i)&\leq (2i-2)n_r + in_s +in_c -2i +1,\\
  \mbox{deg}(F_i)&\leq (2i-1)n_r +in_s + (i-1)n_c -2i +2.
  \end{align*}
\end{izrek}

\begin{proof}
  Najprej predpostavimo, da obstajajo polinomi $D$, $E_i$ in $F_i$, $i=1, \ldots n$, 
  ki ustrezajo enačbi \eqref{eq4} in ostalim pogojem v izreku. Pokazati hočemo, da 
  od tod sledi geometrijska zveznost ploskev $\textbf{R}$ in $\textbf{S}$. V ta namen 
  bomo uporabili izrek \ref{izrek1}. 
  
  Preoblikujmo enačbo \eqref{eq4}. 
  Predpostavka, da je $D(y)E_1(y)\neq 0$ na $[0,1]$, zagotavlja neničelnost polinoma 
  $D$ na $[0,1]$, zato lahko celotno enačbo \eqref{eq4} delimo z $D^{2k-1}(y)$ in dobimo 
  \begin{equation}\label{eq5}
    \begin{split}
    \frac{\partial^k\textbf{S}}{\partial u^k}(0,y) = &
    \sum_{i=0}^k \sum_{|\ddd{m_i}|=k} A_{\ddd{m_i}}^k \sum_{h=0}^i {i\choose h} D^{i-2k}(y)E_{m_1}(v)
    \cdots E_{m_h}(y) \\
    &\cdot F_{m_{h+1}}(y)\cdots F_{m_i}(y) \frac{\partial^i \textbf{R}}{\partial x^h \partial y^{i-h}}(0,y).
    \end{split}
  \end{equation}
  Funkcijo $D^{2k-i}$ lahko zapišemo kot produkt \\
  $D^{2k-i}(y)=D^{2m_1-1}(y)D^{2m_2-1}(y)\cdots D^{2m_h-1}(y) D^{2m_{h+1}}(y)\cdots D^{2m_i-1}(y)$,
  saj je $|\ddd{m_i}|=k$.
  Dobljeno vstavimo v enačbo \eqref{eq5}:
  \begin{equation}\label{eq6}
    \begin{split}
    \frac{\partial^k\textbf{S}}{\partial u^k}(0,y) = &
    \sum_{i=0}^k \sum_{|m_i|=k} A_{m_i}^k \sum_{h=0}^i {i\choose h} \frac{E_{m_1}(v)}{D^{2m_1-1}(y)}
    \cdots \frac{E_{m_h}(y)}{D^{2m_h-1}(y)} \\
    &\cdot \frac{F_{m_{h+1}}(y)}{D^{2m_{h+1}-1}(y)}\cdots \frac{F_{m_i}(y)}{D^{2m_i-1}(y)} \frac{\partial^i \textbf{R}}{\partial x^h \partial y^{i-h}}(0,y).
    \end{split}
  \end{equation}

  Definirajmo
  $$\alpha_i(y)=\frac{E_i(y)}{D^{2i-1}(y)}\textrm{ in }
  \beta_i(y) = \frac{F_i(y)}{D^{2i-1}(y)},\quad i=1, \ldots, n.$$
  Potem enačba \eqref{eq6} dobi enako obliko kot  
  enačba \eqref{eq2} v izreku \ref{izrek1}. Iz izreka \ref{izrek1} torej sledi, da se 
  ploskvi $\textbf{R}$ in $\textbf{S}$ stikata z geometrijsko zveznostjo $G^n$. S tem smo 
  dokazali, da je pogoj \eqref{eq4} zadosten za $G^n$-zveznost \todo{stika??zlepka} ploskev 
  $\ddd{R}$ in $\ddd{S}$. 

  Sedaj dokažimo še, da je pogoj \eqref{eq4} tudi potreben. Predpostavimo, da se ploskvi 
  $\textbf{R}$ in $\textbf{S}$, definirani kot zgoraj, stikata v robni krivulji 
  $\textbf{C}$ z geometrijsko zveznostjo $G^n$. Pokazati hočemo, da od tod sledi obstoj 
  polinomov $D$, $E_i$ in $F_i$ z lastnostmi kot v izreku.
  Dokaza se lotimo z indukcijo po $k$.

  Naj bo najprej $k=1$. Ker je slik ploskev $\textbf{R}$ in $\textbf{S}$ $G^n$-zvezen, 
  torej vsaj $G^1$-zvezen,
  po izreku \ref{izrek1} obstajata $C^n$ funkciji $\alpha_1(y)$ in $\beta_1(y)$, ki zadoščata 
  enačbi
  \begin{equation}\label{eq7}
  \frac{\partial \textbf{S}}{\partial u}(0,y) = 
  \alpha_1(y)\frac{\partial \textbf{R}}{\partial x}(0,y) + 
  \beta_1(y) \frac{\partial \textbf{R}}{\partial y}(0,y).
  \end{equation}
  Dobljeno enačbo z desne vektorsko pomnožimo z $\frac{\partial \textbf{R}}{\partial y}$. 
  in dobimo:
  \begin{equation}\label{eq8}
  \frac{\partial \textbf{S}}{\partial u}(0,y) \times \frac{\partial \textbf{R}}{\partial y}(0,y) = 
  \frac{\partial \textbf{S}}{\partial u}(0,y) \times \frac{\partial \textbf{S}}{\partial v}(0,y) = 
  \alpha_1(y)\frac{\partial \textbf{R}}{\partial x}(0,y)\times \frac{\partial \textbf{R}}{\partial y}(0,y).
  \end{equation}
  V poglavju \ref{geom.zv.} smo namreč že videli, da je 
  $\frac{\partial \textbf{R}}{\partial y}(0,y)=
  \frac{\partial \textbf{S}}{\partial v}(0,y)=\textbf{C}^\prime(y)$.

  Enačbo \eqref{eq7} sedaj z desne vektorsko množimo še z $\frac{\partial \ddd{R}}{\partial x}$ 
  in dobimo
  \begin{equation}\label{eq9}
  \frac{\partial \ddd{S}}{\partial u}(0,y)\times 
  \frac{\partial \ddd{R}}{\partial x}(0,y) = 
  \beta_1(y)\frac{\partial \ddd{R}}{\partial y}(0,y)
  \times \frac{\partial \ddd{R}}{\partial x}(0,y).
  \end{equation}

  Z $\ddd{W}(y)$ označimo vektorsko funkcijo
  $\ddd{W}(y)=\frac{\partial \ddd{R}}{\partial x}(0,y)\times \frac{\partial \ddd{S}}{\partial u}(0,y)$, 
  z $\ddd{N}_R$ in $\ddd{N}_S$ pa normalo na ploskev $\ddd{R}$ oziroma $\ddd{S}$ v neki točki 
  na mejni krivulji $\ddd{C}$.
  %, torej 
  %$\ddd{N}_r = \big(\frac{\partial \ddd{R}}{\partial x} \times \frac{\partial \ddd{R}}{\partial y}\big)\big|_{\ddd{C}}$ 
  %in $\ddd{N}_s = \big(\frac{\partial \ddd{S}}{\partial u} \times \frac{\partial \ddd{S}}{\partial v}\big)\big|_{\ddd{C}}$.

  Prej dobljeni enačbi \eqref{eq8} in \eqref{eq9} torej zapišemo na naslednji način:
  \begin{align}\label{eq10} 
  \ddd{N}_S(y)=\alpha_1(y)\ddd{N}_R(y),&&\ddd{W}(y)=\beta_1(y)\ddd{N}_R(y).
  \end{align}

  \todo{|C??}Stopnja $\frac{\partial\ddd{R}}{\partial y}|_{\ddd{C}}$ je največ $n_c-1$, 
  saj je $\frac{\partial\ddd{R}}{\partial y}|_{\ddd{C}}=\ddd{C}'$. Enako velja za 
  $\frac{\partial \ddd{S}}{\partial v}|_{\ddd{C}}$. Stopnja $\frac{\partial\ddd{R}}{\partial x}|_{\ddd{C}}$ 
  je manjša ali enaka $n_r$, stopnja $\frac{\partial\ddd{S}}{\partial u}|_{\ddd{C}}$ 
  pa manjša ali enaka $n_s$. 
  Od tod in iz definicij funkcij $\ddd{N}_R$, $\ddd{N}_S$ in $\ddd{W}$ sledi
  $\mbox{deg}(\ddd{N}_R)\leq n_r + n_s -1$, $\mbox{deg}(\ddd{N}_S)\leq n_s + n_c -1$ in 
  $\mbox{deg}(\ddd{W})\leq n_r + n_s$.

  Videli smo že, da sta zaradi predpostavke o regularnosti ploskev 
  $\ddd{R}$ in $\ddd{S}$ funkciji $\ddd{N}_R(y)$ in $\ddd{N}_S(y)$ za vsak $y\in[0,1]$
  različni od $0$.
  Ker je $\ddd{N}_R(y)$ neničelna, mora biti vsaj ena izmed njenih koordinatnih funkcij  
  neničeln polinom. Brez škode za splošnost predpostavimo, da je neničelna $x$-koordinata, 
  torej polinom $N_{R,1}(y)$. 
  Če enačbi iz \eqref{eq10} razpišemo po koordinatah, za $x$-koordinato dobimo
  \begin{align*}
  N_{S,1}(y)=\alpha_1(y)N_{R,1}(y),&&W_1(y)=\beta_1(y)N_{R,1}(y),
  \end{align*}
  kjer je $N_{S,1}$ $x$-koordinata funkcije $\ddd{N}_S$, $W_1$ pa $x$-koordinata 
  funkcije $\ddd{W}$.

  Iz zgornjih enačb lahko vidimo, da so vse realne ničle polinoma $N_{R,1}(y)$ na intervalu $[0,1]$ 
  tudi ničle polinomov $N_{S,1}(y)$ in $W_1(y)$, torej da polinom $U(y)$, ki je zgrajen 
  kot produkt vseh linearnih faktorjev v polinomskem razcepu polinoma $N_{R,1}(y)$, deli 
  polinoma $N_{S,1}(y)$ in $W_1(y)$. Da to res drži, lahko vidimo na naslednji način.
  Zapišimo $N_{R,1}(y) = U(y)D(y)$, kjer je $U(y)$ produkt vseh linearnih faktorjev, 
  $D(y)$ pa produkt vseh nelinearnih faktorjev v polinomskem razcepu polinoma $N_{R,1}(y)$. 
  Predpostavimo, da $U(y)$ ne deli polinoma $N_{S,1}(y)$. Ker je $N_{S,1}(y)=\alpha_1(y)U(y)D(y)$, 
  je to mogoče le, če je $\alpha_1(y)$ racionalna funkcija, katere imenovalec deli polinom $U(y)$. 
  Funkcija $\alpha_1(y)$ ima torej na intervalu $[0,1]$ pol. Ker velja $\ddd{N}_S(y)=\alpha_1(y)\ddd{N}_R(y)$
  in so vse koordinatne funkcije funkcij $\ddd{N}_S(y)$ in $\ddd{N}_R(y)$ polinomi, 
  mora veljati, da imenovalec funkcije $\alpha_1(y)$ deli $N_{R,1}(y)$, $N_{R,2}(y)$ in 
  $N_{R,3}(y)$. Funkcija $\alpha_1(y)$ ima pol, označimo ga z $y_0$. Sledi, da je $y_0$ 
  ničla polinomov $N_{R,1}(y)$, $N_{R,2}(y)$ in $N_{R,3}(y)$, in zato je $\ddd{N}_R(y_0)=0$, 
  kar pa je v nasprotju s predpostavko o regularnosti ploskve $\ddd{R}$. Torej mora polinom 
  $U(y)$ deliti polinom $N_{S,1}(y)$. Z enakimi sklepi trditev pokažemo še za polinom $W_1(y)$.


  %\todo{vprašanje: ali U vsebuje vse linearne faktorje ali samo tiste, ki imajo zvezo 
  %z ničlami na [0,1]???}


  Polinom $N_{R,1}$ sedaj znova zapišimo kot produkt $N_{R,1}(y) = U(y)D(y)$, kjer sta 
  polinoma $U(y)$ in $D(y)$ definirana kot zgoraj. Torej velja
  \begin{align*} 
  N_{S,1}(y)=U(y)\alpha_1(y)D(y),&&W_1(y)=U(y)\beta_1(y)D(y).
  \end{align*}
  Naj bo $E_1(y)=\alpha_1(y)D(y)$ in $F_1(y)=\beta_1(y)D(y)$. Pokazati moramo, da sta 
  dobljeni funkciji $E_1$ in $F_1$ polinoma. Ker sta funkciji $N_{S,1}(y)$ in $W_1(y)$ 
  polinoma, morata imenovalca funkcij $\alpha_1$ in $\beta_1$ deliti ali polinom $U$ ali 
  polinom $D$. Videli smo že, da $\alpha_1$ in $\beta_1$ nimata polov na intervalu 
  $[0,1]$, torej njuna imenovalca ne delita polinoma $U$. Sledi, da morata njuna imenovalca 
  deliti polinom $D$, s čimer smo pokazali, da sta $E_1$ in $F_1$ res polinoma.

  Videti želimo še, da je $D(y)E_1(y)\neq 0$ na intervalu $[0,1]$. Polinom $D(y)$ po definiciji 
  vsebuje vse nelinearne faktorje v polinomskem razcepu polinoma $N_{rx}(y)$, torej 
  na intervalu $[0,1]$ nima ničel. Polinom $E_1(y)$ je enak $E_1(y)=\alpha_1(y)D(y)$. 
  Ker je stik ploskev $\ddd{R}$ in $\ddd{S}$ $G^n$-zvezen, funkcija $\alpha_1(y)$ 
  po izreku \ref{izrek1} na 
  intervalu $[0,1]$ ni enaka nič, zato tudi $E_1(y)$ na tem intervalu 
  nima ničel. 

  Oglejmo si še stopnje polinomov $D(y)$, $E_1(y)$ in $F_1(y)$. 
  Očitno velja
  \begin{align*}
  &\mbox{deg}(D(y))\leq \mbox{deg}(N_{rx}(y))\leq \mbox{deg}(\ddd{N}_r(v))\leq n_r+n_c-1\\
  &\mbox{deg}(E_1(y))\leq \mbox{deg}(N_{sx}(y))\leq \mbox{deg}(\ddd{N}_s(v))\leq n_s+n_c-1\\
  &\mbox{deg}(F_1(y))\leq \mbox{deg}(W_x(y))\leq st(\ddd{W}(v))\leq n_r+n_s,\\
  \end{align*}
  s čimer dokažemo izrek za $k=1$.

  Lotimo se še dokaza za $k>1$. Prepodstavimo, da izrek velja za vse $k\leq m$, kjer je $m\in \N$, $m<n$. 
  Torej obstajajo polinomi $D(y)$, $E_1(y), \ldots, E_m(y)$, $F_1(y), \ldots, F_m(y)$ 
  z ustreznimi stopnjami, da velja enačba \eqref{eq4} za $k=1,2,\ldots,m$.
  
  Izhajamo iz predpostavke, da je stik ploskev $\ddd{R}$ in $\ddd{S}$ $G^n$-zvezen. 
  Iz izreka \ref{izrek1} sledi, da obstajajo funkcije 
  $\alpha_1, \ldots, \alpha_{m+1}$ in $\beta_1 \ldots, \beta_{m+1}$, da velja
  \begin{align*}
    \frac{\partial^{m+1} \ddd{S}}{\partial u^{m+1}}(0,v)&= 
    \sum_{i=1}^{m+1}\sum_{|\ddd{m}_i|=m+1}A_{m_i}^{m+1}
    \sum_{h=0}^i {i \choose h}\alpha_{m_1}(v)\cdots \alpha_{m_h}(v)\cdot\\
    &\cdot\beta_{m_{h+1}}(v)\beta_{m_i}(v) 
    \frac{\partial^i \ddd{R}}{\partial x^h \partial y^{i-h}}(0,v)=\\
    & = \alpha_{m+1}(v) \frac{\partial \ddd{R}}{\partial u}(0,v) + 
    \beta_{m+1}(v) \frac{\partial \ddd{R}}{\partial v}(0,v)+\\
    & + \sum_{i=2}^{m+1} \sum_{|\ddd{m}_i|=m+1}A_{m_i}^{m+1} \sum_{h=0}^i {i \choose h}
    \alpha_{m_1}(v) \cdots \alpha_{m_h}(v)\cdot\\ 
    &\cdot\beta_{m_{h+1}}(v) \cdots \beta_{m_i}(v)
    \frac{\partial^i \ddd{R}}{\partial x^h \partial y^{i-h}}(0,v).
  \end{align*}
  Po indukcijski predpostavki je $\alpha_i(y) = \frac{E_i(y)}{D^{2i-1}(y)}$ in 
  $\beta_i(y)=\frac{F_i(y)}{D^{2i-1}(y)}$ za $i=1, \ldots, m$. 
  Uporabimo to v zgornji 
  enačbi in dobimo:
  \begin{equation}\label{eq12}
    \begin{split}
    \frac{\partial^{m+1} \ddd{S}}{\partial u^{m+1}}\Bigr|_{\substack{\ddd{C}}}&= 
    \alpha_{m+1} \frac{\partial \ddd{R}}{\partial u}|_{\ddd{C}} + 
    \beta_{m+1} \frac{\partial \ddd{R}}{\partial v}|_{\ddd{C}} + \\
    &+ \sum_{i=2}^{m+1} \sum_{|\ddd{m}_i|}A_{m_i}^{m+1} \sum_{h=0}^i {i \choose h}
    \frac{E_{m_1}(y)}{D^{2m_1-1}(y)} \cdots \frac{E_{m_h}(y)}{D^{2m_h-1}(y)}\\ 
    &\cdot \frac{F_{m_{h+1}}(y)}{D^{2m_{h+1}-1}(y)} \cdots \frac{F_{m_i}(y)}{D^{2m_i-1}(y)}
    \frac{\partial^i \ddd{R}}{\partial x^h \partial y^{i-h}}\Bigr|_{\substack{C}} = \\
    &= \alpha_{m+1} \frac{\partial \ddd{R}}{\partial u}|_{\ddd{C}} + 
    \beta_{m+1} \frac{\partial \ddd{R}}{\partial v}|_{\ddd{C}} + \\
    &+ \sum_{i=2}^{m+1} \sum_{|\ddd{m}_i|}A_{m_i}^{m+1} \sum_{h=0}^i {i \choose h}
    D^{i-2}(y)D^{-2m}E_{m_1}(y) \cdots E_{m_h}(y) \\
    & \cdot F_{m_{h+1}}(y) \cdots F_{m_i}(y)
    \frac{\partial^i \ddd{R}}{\partial x^h \partial y^{i-h}}\Bigr|_{\substack{C}},
    \end{split}
  \end{equation}
  saj je $|\ddd{m}_i|=m_1+m_2+\cdots + m_i = m+1$ in zato je \\
  $$D^{-2m_1+1}(y) D^{-2m_2+1}(y) \cdots D^{-2m_i+1}(y)=D^{-2(m+1)}(y)D^i(y).$$

  Sedaj definirajmo vektorsko polinomsko funkcijo $\ddd{S}_{m+1}$:
  \begin{align*}\ddd{S}_{m+1}(y)=&D^{2m}(y)\frac{\partial^{m+1} \ddd{S}}{\partial u^{m+1}}(0,y) - 
  \sum_{i=2}^{m+1} \sum_{|\ddd{m}_i|=m+1} A_{m_i}^{m+1} \sum_{h=0}^i {i \choose h}
  D^{i-1}(y)\cdot \\ &\cdot E_{m_1}(y)\cdots E_{m_h}(y) F_{m_{h+1}}(y)\ldots F_{m_i}(y)
  \frac{\partial^i \ddd{R}}{\partial x^h \partial y^{i-h}}(0,y).
  \end{align*}
  Če enačbo \eqref{eq12} pomnožimo z $D^{2m}(y)$ in jo nekoliko preoblikujemo, dobimo
  \begin{equation}\label{eq13} 
  \ddd{S}_{m+1}(y)=D^{2m}(y)\alpha_{m+1}(y)\frac{\partial \ddd{R}}{\partial x}(0,y) + 
  D^{2m}(y)\beta_{m+1}(y)\frac{\partial \ddd{R}}{\partial y}(0,y).
  \end{equation}

  Na dobljeni enačbi sedaj uporabimo podoben postopek, kot smo ga uporabili pri dokazu 
  za $k=1$. Enačbo \eqref{eq13} z leve vektorsko množimo z $\frac{\partial \ddd{R}}{\partial x}(0,y)$ 
  in dobimo 
  $$\frac{\partial \ddd{R}}{\partial x}(0,y) \times \ddd{S}_{m+1}(y) = 
  D^{2m}(y)\beta_{m+1}(y) \frac{\partial \ddd{R}}{\partial x}(0,y) \times \frac{\partial \ddd{R}}{\partial y}(0,y).$$
  Če pa enačbo \eqref{eq13} z desne pomnožimo z $\frac{\partial \ddd{R}}{\partial y}(0,y)$, 
  dobimo 
  $$\ddd{S}_{m+1}(y)\times\frac{\partial \ddd{R}}{\partial x}(0,y) = 
  D^{2m}(y)\alpha_{m+1}(y)\frac{\partial \ddd{R}}{\partial x}(0,y) \times \frac{\partial \ddd{R}}{\partial y}(0,y).$$
  Označimo
  \begin{equation}\label{eq31} 
  \ddd{\overline{W}}(y) = \ddd{S}_{m+1}(y)\times\frac{\partial \ddd{R}}{\partial y}(0,y) 
  \end{equation}
  in
  \begin{equation}\label{eq32} 
  \ddd{\overline{\overline{W}}}(y) = \frac{\partial \ddd{R}}{\partial x}(0,y)\times \ddd{S}_{m+1}(y)
  \end{equation}
  ter kakor prej 
  $N_R = \frac{\partial \ddd{R}}{\partial x}(0,y) \times \frac{\partial \ddd{R}}{\partial y}(0,y)$. 
  Kot v primeru za $k=1$, spet lahko predpostavimo, da je polinom $N_{R,1}(y)$ neničeln in 
  ga zapišemo kot $N_{R,1}(y)=U(y)D(y)$. Velja 
  $\overline{W_{1}} = D^{2m+1}(y)U(y)\alpha_{m+1}(y)$ in $\overline{\overline{W_{1}}} = D^{2m+1}(y)U(y)\beta_{m+1}(y)$ 
  in enaki argumenti kot v primeru za $k=1$ nas pripeljejo do razulatata, da sta 
  $$E_{m+1}(y) = D^{2m+1}(y)\alpha_{m+1}(y), \quad F_{m+1}(y) = D^{2m+1}(y)\beta_{m+1}(y)$$ 
  res polinoma. 

  Pokazati moramo še, da je $\mbox{deg}(E_{m+1})\leq 2mn_r + (m+1)n_s+(m+1)n_c-2m-1$ in 
  $\mbox{deg}(F_{m+1})\leq (2m+1)n_r + (m+1)n_s+mn_c-2m$. Tega se lotimo tako, da si najprej 
  ogledamo stopnjo $\ddd{S}_{m+1}$.

  Očitno je
  \begin{align*} 
    \mbox{deg}\left(D^{2m}(y)\frac{\partial^{m+1} \ddd{S}}{\partial u^{m+1}}\bigg|_{\ddd{C}}\right)&\leq 
    2m(n_r+n_c-1)+n_s \\
    &\leq 2mn_r + (m+1)n_s + mn_c -2m, 
  \end{align*}
  kjer v prvi neenakosti uporabimo dejstvo, da je $\mbox{deg}(D(y))\leq n_r + n_c-1$ in \\
  $\mbox{deg}(\frac{\partial^{m+1} \ddd{S}}{\partial u^{m+1}})\leq n_s$, v drugi neenakosti pa, 
  da je $n_c\leq n_s$.

  Oglejmo si še, kakšna je
  $$\mbox{deg}\left(D^{i-1}E_{m_1}\cdots E_{m_h}F_{m_{h+1}}\ldots F_{m_i}
  \frac{\partial^i \ddd{R}}{\partial x^h \partial y^{i-h}}\bigg|_{\ddd{C}}\right).$$
  Najprej si jo oglejmo za $h=0$:
  \begin{align*}
    &\mbox{deg}\left(D^{i-2}F_{m_1}\cdots F_{m_i}\frac{\partial^i \ddd{R}}{\partial y^i}\bigg|_{\ddd{C}}\right)\leq\\
    &\leq(i-2) \mbox{deg}(D) + \sum_{j=1}^i \mbox{deg}(F_{m_j}) + \mbox{deg}\left(\frac{\partial^i \ddd{R}}{\partial y^i}\bigg|_{\ddd{C}}\right) 
  \end{align*}
  Vemo, da je $\mbox{deg}(D)\leq n_r+n_c-1$ in $\mbox{deg}(\frac{\partial^i \ddd{R}}{\partial y^i}|_{\ddd{C}})\leq n_c-i$, 
  po indukcijski predpostavki pa velja še
  \begin{align*} 
  &\mbox{deg}(E_i)\leq (2i-2)n_r+in_s+in_c-2i+1),\\
  &\mbox{deg}(F_i)\leq (2i-1)n_r+in_s+(i-1)n_c-2i+2),\quad i=1, \ldots, m.
  \end{align*} 
  Torej je 
  \begin{align*}
    &\mbox{deg}\left(D^{i-2}F_{m_1}\cdots F_{m_i}\frac{\partial^i \ddd{R}}{\partial y^i}\bigg|_{\ddd{C}}\right)\leq
    (i-2)(n_r+n_c-1)+\\
    &+\sum_{j=1}^i((2m_j-1)n_r+m_jn_s+(m_j-1)n_c-2m_j+2) + (n_c -i)= \\
    &= (i-2)(n_r + n_c -1) + 2(m+1)n_r - in_r + (m+1)n_s + (m+1)n_c -in_c -\\
    &- 2(m+1) +2i + n_c -i = \\
    &= 2mn_r + (m+1)n_s + mn_c -2m -in_r \leq \\
    &\leq 2mn_r +(m+1)n_s + mn_c -2m.
  \end{align*}
  Tu smo uporabili, da je $\sum_{j=1}^{i}m_j = m+1$.

  Sedaj obravnavajmo še primer, ko je $h>1$.
  \begin{align*}
    &\mbox{deg}\left(D^{i-2}E_{m_1}\cdots E_{m_h}F_{m_{h+1}}\cdots F_{m_i}
    \frac{\partial^i\ddd{R}}{\partial x^h \partial y^{i-h}}\bigg|_{\ddd{C}}\right)\leq \\
    &\leq (i-2)\mbox{deg}(D) + \sum_{j=1}^h \mbox{deg}(E_{m_j}) + \sum_{j=h+1}^i \mbox{deg}(F_{m_j}) + \\
    &+ \mbox{deg}\left(\frac{\partial^i \ddd{R}}{\partial x^h \partial y^{i-h}}\bigg|_{\ddd{C}}\right).
  \end{align*}
  Zopet uporabimo indukcijsko predpostavko za stopnje polinomov $E_i$ in $F_i$, 
  kjer je $i=1,\ldots, m$, ter dejstvo, da je 
  $\mbox{deg}\left(\frac{\partial^i \ddd{R}}{\partial x^h \partial y^{i-h}}\bigg|_{\ddd{C}}\right)=n_r-i+h$, in 
  dobimo
  \begin{align*}
    &\mbox{deg}\left(D^{i-2}E_{m_1}\cdots E_{m_h}F_{m_{h+1}}\cdots F_{m_i}
    \frac{\partial^i\ddd{R}}{\partial x^h \partial y^{i-h}}\bigg|_{\ddd{C}}\right)\leq \\
    &\leq (i-2)(n_r+n_c-1)+2n_r\sum_{j=1}^h m_j - 2n_rh + n_s\sum_{j=1}^hm_j +
    + n_c\sum_{j=1}^hm_j-\\ &- 2\sum_{j=1}^hm_j + h + 2n_r\sum_{j=h+1}^im_j - (i-h)n_r
    + n_s\sum_{j=h+1}^im_j + n_c\sum_{j=h+1}^i - (i-h)n_c-\\ &
    - 2\sum_{j=h+1}^i + 2(i-h) +n_r-i+h=\\
    &=(i-2)(n_r+n_c-1)+2n_r(m+1)+n_s(m+1)+n_c(m+1)-2(m+1)-\\
    &-2n_rh+h-(i-h)n_r-(i-h)n_c+2(i-h)+n_r-i-h.
  \end{align*}
  V zadnji enakosti smo uprabili, da je $\sum_{j=1}^im_j=m+1$. Nadaljujmo z računom:
  \begin{align*}
    &\mbox{deg}\left(D^{i-2}E_{m_1}\cdots E_{m_h}F_{m_{h+1}}\cdots F_{m_i}
    \frac{\partial^i\ddd{R}}{\partial x^h \partial y^{i-h}}\bigg|_{\ddd{C}}\right)\leq \\
    &\leq(i-2)(n_r+n_c-1)+2n_r(m+1)+n_s(m+1)+n_c(m+1)-2(m+1)-\\
    &-2n_rh+h-(i-h)n_r-(i-h)n_c+2(i-h)+n_r-i-h \leq \\
    &\leq 2mn_r+(m+1)n_s+mn_c-2m-n_rh+n_ch-n_c+n_r = \\
    &=2mn_r + (m+1)n_s + mn_c-2m+(h-1)(n_c-n_r)\leq \\
    &\leq 2mn_r + (m+1)n_s +mn_c-2m.
  \end{align*}
  V zadnji neenakosti smo uporabili, da je $n_c\leq n_r$, torej je $n_c-n_r\leq 0$.
  S tem smo torej pokazali, da je $$\mbox{deg}(\ddd{S}_{m+1})\leq 2mn_r + (m+1)n_s+(m+1)n_c-2m.$$

  Iz enačbe \eqref{eq31} in definicije polinoma $E_{m+1}$ je razvidno naslednje:
  \begin{align*}
    &\mbox{deg}(E_{m+1})\leq \mbox{deg}(\overline{W_{1}})\leq \mbox{deg}(\overline{\ddd{W}})\leq \mbox{deg}(\ddd{S}_{m+1})+
    \mbox{deg}\left(\frac{\partial \ddd{R}}{\partial y}\bigg|_{\ddd{C}}\right) \leq \\
    &\leq (2mn_r+(m+1)n_s+mn_c-2m)+(n_c-1)=\\
    &=2mn_r + (m+1)n_s + (m+1)n_c -2m -1.
  \end{align*}

  Podobno dobimo iz enačbe \eqref{eq32} in definicije polinoma $F_{m+1}$ oceno za 
  stopnjo polinoma $F_{m+1}$:
  \begin{align*}
    &\mbox{deg}(F_{m+1})\leq \mbox{deg}(\overline{\overline{W_{1}}})\leq \mbox{deg}(\overline{\overline{\ddd{W}}})\leq \mbox{deg}(\ddd{S}_{m+1})+
    \mbox{deg}\left(\frac{\partial \ddd{R}}{\partial x}\bigg|_{\ddd{C}}\right) \leq \\
    &\leq (2mn_r+(m+1)n_s+mn_c-2m)+n_r=\\
    &=(2m+1)n_r + (m+1)n_s + mn_c -2m.
  \end{align*}
  S tem smo dokazali izrek še za $k>1$.
\end{proof}

%%%%%%%%%%%%%%%%%%%%%%%%%%%%%%%%%%%%%%%%%%%%%%%%%%%%%%%%%%%%%%%%%%%%%%%%%%%%%%%%%%%%%%%%%%%%%%%%%%%%%
%%%%%%%%%%%%%%%%%%%%%%%%%%%%%%%%%%%%%%%%%%%%%%%%%%%%%%%%%%%%%%%%%%%%%%%%%%%%%%%%%%%%%%%%%%%%%%%%%%

%\subsection{Kompatibilnostni pogoji}
%
%enačba \eqref{eq13}
%
%vstavimo $\alpha_{m+1}(y)=\frac{E_{m+1}(y)}{D^{2m+1}(y)}$ in 
%$\beta_{m+1}(y)=\frac{F_{m+1}(y)}{D^{2m+1}(y)}$
%
%ker je bil $m$ v dokazu izreka poljuben, lahko zamenjamo $m+1$ s $k$, kjer je 
%$k=1,\ldots,n$
%
%$$D(y)\ddd{S}_{k}(y)=E_{k}(y)\frac{\partial\ddd{R}}{\partial x}|_{\ddd{C}}+
%F_{k}(y)\frac{\partial \ddd{R}}{\partial y}|_{C}$$
%
%ta enačba je ekvivalentna enačbi v izreku
%
%rabili jo bomo za preučevanje pogojev, ki omejujejo izbiro koeficientnih funkcij
%
%ker ??, koeficientne funkcije ne morejo biti povsem poljubne, 
%zadoščati morajo nekaterim pogojem - kompatibilnostni pogoji 
%
%\todo{v praksi: nizke stopnje koeficientnih polinomov - kompatibilnostni pogoji 
%izginejo - to daj na konec, povezava s primerom}
%
%\todo{kompatibilnostni pogoji določijo nekatere kontrolne točke pri konstrukciji}
%
%zaradi lažje notacije označimo 
%$\ddd{A}(y)=\frac{\partial \ddd{R}}{\partial x}|_{\ddd{C}}(y)$, 
%$\ddd{B}(y)=\frac{\partial \ddd{R}}{\partial y}|_{\ddd{C}}(y)$, 
%$\ddd{W}(y)=\ddd{S}_k(y)$, $E(y)=E_k(y)$, $F(y)=F_k(y)$
%
%$$D(y)\ddd{W}(y)=E(y)\ddd{A}(y)+
%F(y)\ddd{B}(y)$$
%
%$st(D)+st(\ddd{W})=st(E)+st(\ddd{A})=st(F)+st(\ddd{B})=n$
%
%$st(\ddd{W})=n_w$, $st(\ddd{A})=n_a$, $st(\ddd{B})=n_b$
%
%potem $st(D)=n_d=n-n_w$, $st(E)=n_e=n-n_a$, $st(F)=n_f=n-n_b$
%
%\todo{nekako povej, da imajo W, A, B kontrolne vektorje}
%
%$\{\ddd{w}_i; i=0,1,\ldots,n_w\}$, $\{\ddd{a}_i; i=0,1,\ldots,n_a\}$, 
%$\{\ddd{b}_i; i=0,1,\ldots, n_b\}$ 
%kontrolni vektorji od $\ddd{W}$, $\ddd{A}$ in $\ddd{B}$. 
%če so $D$, $E$ in $F$ polinomi v Bernsteinovi bazi, naj bodo 
%$\{d_i, i=0,1,\ldots, n_d\}$, $\{e_i; i=0,1,\ldots, n_e\}$ in 
%$\{f_i; i=0,1,\ldots, n_f\}$ 
%kontrolni koeficienti teh polinomov.
%
%označimo 
%$\ddd{D}_i = [\underbrace{0,0,\ldots,0}_i,d_0,n_dd_0,\ldots {n_d \choose j}d_j,
%\ldots, n_dd_{n_d-1},d_{n_d},\underbrace{0,0,\ldots,0}_{n_w-i}]^T$
%
%$\ddd{E}_i = [\underbrace{0,0,\ldots,0}_i,e_0,n_ee_0,\ldots {n_e \choose j}e_j,
%\ldots, n_ee_{n_e-1},e_{n_e},\underbrace{0,0,\ldots,0}_{n_a-i}]^T$
%
%$\ddd{F}_i = [\underbrace{0,0,\ldots,0}_i,f_0,n_ff_0,\ldots {n_f \choose j}f_j,
%\ldots, n_ff_{n_f-1},f_{n_f},\underbrace{0,0,\ldots,0}_{n_b-i}]^T$
%
%$\overline{\ddd{W}}=[\ddd{w}_0,n_w\ddd{w}_1,\ldots,{n_w \choose i}\ddd{w}_i,
%\ldots, n_w\ddd{w}_{n_w-1},\ddd{w}_{n_w}]^T$
%
%$\overline{\ddd{A}}=[\ddd{a}_0,n_a\ddd{a}_1,\ldots,{n_a \choose i}\ddd{a}_i,
%\ldots, n_a\ddd{a}_{n_a-1},\ddd{a}_{n_a}]^T$
%
%$\overline{\ddd{B}}=[\ddd{b}_0,n_w\ddd{b}_1,\ldots,{n_b \choose i}\ddd{b}_i,
%\ldots, n_b\ddd{b}_{n_b-1},\ddd{b}_{n_b}]^T$
%
%sestavimo matrike
%$M_d = [\ddd{D}_0, \ddd{D}_1, \ldots, \ddd{D}_{n_w}]$
%
%$M_e = [\ddd{E}_0, \ddd{E}_1, \ldots, \ddd{E}_{n_a}]$
%
%$M_f = [\ddd{F}_0, \ddd{F}_1, \ldots, \ddd{F}_{n_b}]$
%
%dimenzij $(n+1)\times (n_w+1)$, $(n+1)\times(n_a+1)$, $(n+1)\times(n_b+1)$
%
%\begin{align*}
%  M_d=
%\begin{bmatrix}
%  M_d^1\\
%  M_d^2
%\end{bmatrix} 
%\end{align*}, 
%\begin{align*}
%  M_e&=
%\begin{bmatrix}
%  M_e^1\\
%  M_e^2
%\end{bmatrix} 
%\end{align*}, 
%\begin{align*}
%  M_f&=
%\begin{bmatrix}
%  M_f^1\\
%  M_f^2
%\end{bmatrix} 
%\end{align*}, 
%
%$M_d^1$, $M_e^1$ in $M_f^1$ so dimenzij $(n_w+1)\times(n_w+1)$, $(n_a+1)\times(n_a+1)$ 
%in $(n_b+1)\times (n_b+1)$. 
%
%$M_d^2$, $M_e^2$ in $M_f^2$ so dimenzij $n_d\times(n_w+1)$, $n_e\times(n_a+1)$ 
%in $n_f \times (n_b+1)$
%
%$M_d^1\overline{\ddd{W}}=M_e^1\overline{\ddd{A}}+M_f^1\overline{\ddd{B}}$
%
%$M_d^2\overline{\ddd{W}}=M_e^2\overline{\ddd{A}}+M_f^2\overline{\ddd{B}}$
%
%$M_d^1$ je obrnljiva
%\todo{blabla}
%
%$\overline{\ddd{W}}=(M_d^1)^{-1}M_e^1\overline{\ddd{A}}+(M_d^1)^{-1}M_f^1\overline{\ddd{B}}$
%
%$(M_e^2-M_d^2(M_d^1)^{-1}M_e^1)\overline{\ddd{A}} + 
%(M_f^2-M_d^2(M_d^1)^{-1}M_f^1)\overline{\ddd{B}} = 0$

\section{Primeri konstrukcij geometrijsko zveznih ploskev iz tenzorskega produkta}

\todo{nek uvod, navezava na prejšnje poglavje}

\subsection{Konstrukcija $G^1$-zveznih Bézierjevih ploskev iz tenzorskega produkta}
\label{g1primeri}

V tem podpoglavju si bomo ogledali, kako na različne načine konstruirati ploskvi, ki 
sta na stiku $G^1$-zvezni, torej kakšne pogoje prinesejo različni načini konstrukcije 
na njune kontrolne točke. Pri tem bomo predpostavljali, 
da so robovi obeh ploskev vnaprej določeni. Pogoje, ki jih prinese zahteva $G^1$-zveznosti 
bomo primerjali s $C^1$-zveznostjo. 

Imejmo dve bi-kubični Bézierjevi ploskvi iz tenzorskega produkta:
$$\ddd{R}(u,v)=\sum_{i=0}^3 \sum_{j=0}^3 \mathbf{P}_{i,j}B_i^3(u) B_j^3(v)$$ in
$$\ddd{S}(u,v)=\sum_{i=0}^3 \sum_{j=0}^3 \mathbf{Q}_{i,j}B_i^3(u) B_j^3(v),$$
kjer velja $u,v\in[0,1]$.

Ploskvi $\ddd{R}$ in $\ddd{S}$ naj se stikata v $\ddd{C}(v)=\ddd{R}(0,v)=\ddd{S}(0,v)$, torej naj velja
$$\ddd{C}(v)=\sum_{i=0}^{n_c} \ddd{Z}_i B_i^{n_c}(v),$$
kjer so $\ddd{Z}_i = \ddd{P}_{0,i} = \ddd{Q}_{0,i}$ kontrolne točke krivulje $\ddd{C}$. 
%Stopnja $n_c$ krivulje $\ddd{C}$ ni nujno enaka $3$, veljati pa mora $n_c\leq 3$.\todo{?}
Predpostavili bomo, da so robne krivulje ploskev $\ddd{R}$ in $\ddd{S}$ 
že določene na tak način, da bomo imeli na robu zahtevano zveznost. 
Zanimalo nas bo,  
kakšne zveze v teh primerih veljajo za notranje kontrolne točke, torej za 
$\ddd{P}_{1,1}$, $\ddd{P}_{1,2}$, $\ddd{Q}_{1,1}$ in $\ddd{Q}_{1,2}$., 
da bo stik ploskev $G^1$-zvezen. 
%\todo{razlaga s ploščinami trikotnikov?} 
V nadaljevanju bomo uporabljali še naslednje oznake za kontrolne vektorje obeh 
ploskev in robne krivulje:
$\ddd{p}_{i,j}=\ddd{P}_{i+1,j}-\ddd{P}_{i,j}$, 
$\ddd{q}_{i,j}=\ddd{Q}_{i+1,j}-\ddd{Q}_{i,j}$ in 
$\ddd{z}_i=\ddd{Z}_{i+1}-\ddd{Z}_i$.

Najprej si oglejmo, kakšne pogoje in omejitve nam da zahteva $C^1$-zveznosti 
na stiku teh dveh ploskev. To bomo nato primerjali s pogoji, ki nam jih da 
$G^1$-zveznost.

\begin{primer}\label{primer1}
  Domena ploskev $\ddd{R}$ in $\ddd{S}$ je kvadrat $[0,1]\times[0,1]$. Da lahko 
  obravnavamo $C^1$-zveznost stika ploskev, moramo najprej reparametrizirati 
  ploskev $\ddd{R}$ tako, da bo njena domena $[-1,0]\times[0,1]$ in bosta obe domeni 
  skupaj po stiku tvorili pravokotnik $[-1,1]\times[0,1]$. Da to dosežemo, 
  moramo ploskev $\ddd{R}$ zapisati na naslednji način:
  $$\ddd{R}(u,v) = \sum_{i=0}^3 \sum_{j=0}^3 \mathbf{P}_{i,j}B_i^3(-u) B_j^3(v),$$
  kjer je $u\in[-1,0]$ in $v\in[0,1]$.

  Da je stik ploskev $\ddd{R}$ in $\ddd{S}$ $C^0$-zvezen, se morata krivulji 
  ujemati v kontrolnih točkah, ki določajo stično krivuljo: 
  $\ddd{P}_{0,j}=\ddd{Q}_{0,j}$ za $j=0,\ldots,3$. V podpoglavju \ref{cn.zv.} smo 
  videli, da se morata za dosego $C^1$-zveznosti 
  poleg tega ujemati še odvoda obeh ploskev v $u$-smeri v robnih 
  točkah: 
  $\frac{\partial}{\partial u}\ddd{R}(u,v)|_{u=0}=\frac{\partial}{\partial u}\ddd{S}(u,v)|_{u=0}$.
  Če razpišemo oba parcialna odvoda oziroma se sklicujemo na enakost \eqref{eq29}, 
  dobimo naslednje pogoje za $C^1$-zveznost med ploskvama:
  $$-(\ddd{P}_{1,j}-\ddd{P}_{0,j})=\ddd{Q}_{1,j}-\ddd{Q}_{0,j}$$
  oziroma 
  $$\ddd{q}_{0,j}=-\ddd{p}_{0,j}$$
  za $j=0,\ldots,3$.

  Vidimo torej, da morata biti za dosego $C^1$-zveznosti zlepka ploskev vektorja 
  $\ddd{p}_{0,j}$ in $\ddd{q}_{0,j}$ kolinearna za vsak $j=0,\ldots,3$, poleg tega 
  pa morata biti njuni dolžini v razmerju, ki ga določata parametrizaciji obeh 
  ploskev. Kar se tiče oblike ploskve, ki jo na ta način lahko konstruiramo kot 
  zlepek ploskev $\ddd{R}$ in $\ddd{S}$, torej nimamo ravno veliko izbire. 
  Nekoliko več svobode imamo le pri izbiri notranjih kontrolnih točk. 
  Kontrolni točki $\ddd{Q}_{1,1}$ in $\ddd{Q}_{1,2}$ sta točno določeni 
  z izbiro kontrolnih točk $\ddd{P}_{1,1}$ in $\ddd{P}_{1,2}$, medtem ko sta 
  $\ddd{P}_{1,1}$ in $\ddd{P}_{1,2}$ prosti. Ker zahtevamo zgolj zveznost stopnje 
  1, so proste tudi kontrolne točke $\ddd{Q}_{2,1}$, $\ddd{Q}_{2,2}$, 
  $\ddd{P}_{2,1}$ in $\ddd{P}_{2,2}$.

\end{primer}

Sedaj si oglejmo nekaj primerov konstrukcij $G^1$-zveznih ploskev in jih primerjajmo 
z rezultatom, dobljenim v primeru \ref{primer1}. 
Izrek \ref{izrek1} pravi, da je stik obeh ploskev $G^1$-zvezen, natanko tedaj ko 
obstajata $C^1$ funkciji 
$\alpha_1(v)$ in $\beta_1(v)$, da velja 
\begin{align}\label{eq33}
\frac{\partial \ddd{S}}{\partial u}(0,v) = \alpha_1(v)\frac{\partial \ddd{R}}{\partial u}(0,v) + 
\beta_1(v)\frac{\partial \ddd{R}}{\partial v}(0,v),
\end{align}
kjer je $\alpha_1(v)\neq 0$ na intervalu $[0,1]$ in ima ustrezen predznak.
V našem primeru gre za polinomske ploskve, zato lahko uporabimo izrek \ref{izrek2},
ki pove, da to velja natanko tedaj, 
ko obstajajo polinomi $D(v)$, $E_1(v)$ 
in $F_1(v)$, kjer sta polinoma $D$ in $E_1$ stopnje največ 5, polinom $F_1$ pa stopnje 
največ 6, da velja $D(v)E_1(v)\neq 0$ na $[0,1]$ ter
\begin{align}\label{eq14}
D(v)\frac{\partial \ddd{S}}{\partial u}(0,v) = E_1(v)\frac{\partial \ddd{R}}{\partial u}(0,v) + 
F_1(v)\frac{\partial \ddd{R}}{\partial v}(0,v).
\end{align}

Razpišimo prve odvode parametrizacij ploskev $\ddd{R}$ in $\ddd{S}$: %\todo{sklic na formulo?}
$$\frac{\partial \ddd{S}}{\partial u}(0,v)=3\sum_{i=0}^2\sum_{j=0}^3
(\ddd{P}_{i+1,j}-\ddd{P}_{i,j})B_i^2(0)B_j^3(v)=
3\sum_{j=0}^3(\ddd{P}_{1,j}-\ddd{P}_{0,j})B_j^3(v),$$
$$\frac{\partial \ddd{R}}{\partial u}(0,v)=3\sum_{i=0}^2\sum_{j=0}^3
(\ddd{Q}_{i+1,j}-\ddd{Q}_{i,j})B_i^2(0)B_j^3(v)=
3\sum_{j=0}^3(\ddd{Q}_{1,j}-\ddd{Q}_{0,j})B_j^3(v),$$
in 
$$\frac{\partial \ddd{R}}{\partial v}(0,v)=3\sum_{i=0}^3\sum_{j=0}^2
(\ddd{Q}_{i,j+1}-\ddd{Q}_{i,j})B_i^2(0)B_j^3(v)=
3\sum_{j=0}^2(\ddd{Q}_{0,j}-\ddd{Q}_{0,j+1})B_j^2(v).$$
Dobljeno vstavimo v enačbo \eqref{eq14}. Vidimo, da mora veljati:
\begin{equation}\label{eq15}
  \begin{split} 
&D(v)\sum_{j=0}^3(\ddd{P}_{1,j}-\ddd{P}_{0,j})B_j^3(v) = \\
&=E_1(v)\sum_{j=0}^3(\ddd{Q}_{1,j}-\ddd{Q}_{0,j})B_j^3(v) + 
F_1(v)\sum_{j=0}^2(\ddd{Q}_{0,j}-\ddd{Q}_{0,j+1})B_j^2(v).
  \end{split}
\end{equation}

Najprej si oglejmo, kakšni pogoji v primeru $G^1$ zveznosti veljajo za robne 
kontrolne točke ploskev $\ddd{R}$ in $\ddd{S}$. V enačbo \eqref{eq15} vstavimo 
vrednosti $v=0$ in $v=1$.
Pri vrednosti $v=0$ dobimo
$$d_0(\ddd{Q}_{1,0}-\ddd{Q}_{0,0}) = e_0(\ddd{P}_{1,0}-\ddd{P}_{0,0})+f_0(\ddd{P}_{0,1}-\ddd{P}_{0,0}),$$
oziroma
\begin{align}\label{eq18}
d_0\ddd{q}_{0,0} = e_0 \ddd{p}_{0,0} + f_0 \ddd{z}_0.
\end{align}  
Tu smo z $d_0$ označili vrednost $D(0)$, z $e_0$ vrednost $E_1(0)$, z $f_0$ 
pa vrednost $F_1(0)$.
Pri vrednosti $v=1$ pa dobimo
$$d_1\ddd{Q}_{1,3}= e_1(\ddd{P}_{1,3}-\ddd{P}_{0,3})+f_1(\ddd{P}_{0,3}-\ddd{P}_{0,2}),$$
oziroma
\begin{align}\label{eq19}
d_1\ddd{q}_{0,3} = e_1 \ddd{p}_{0,3} + f_1 \ddd{z}_2.
\end{align}
Tu smo z $d_1$ označili vrednost $D(1)$, z $e_1$ vrednost $E_1(1)$ in z 
$f_1$ vrednost $F_1(1)$.

Pogoji, ki veljajo za robne kontrolne točke, so enaki neglede na način konstrukcije 
$G^1$-zveznega zlepka ploskev. Pogoji, ki veljajo za notranje kontrolne točke, 
število svobodnih parametrov, ki določajo obliko dobljene ploskve, in število 
prostih kontrolnih točk, pa so odvisni od izbire načina konstrukcije, natančneje, 
od izbire stopnje koeficientnih polinomskih funkcij in stopnje odvodov parametrizacij 
ploskev $\ddd{R}$ in $\ddd{S}$.

Izbira stopenj koeficientnih funkcij ni povsem poljubna, temveč je odvisna od 
stopnje geometrijske zveznosti, ki jo zahtevamo, pa tudi od
stopenj odvodov $\frac{\partial \ddd{S}}{\partial u}|_{u=0}$, 
$\frac{\partial \ddd{R}}{\partial u}|_{u=0}$ in $\frac{\partial \ddd{R}}{\partial v}|_{u=0}$ 
oziroma $\ddd{C}'(v)$. 

V praksi se običajno uporabljajo koeficientne funkcije čim nižje stopnje, saj s 
tem dobimo manj pogojev za kontrolne točke. V primeru, da sta funkciji $D(v)$ in 
$E(v)$ konstantni, funkcija $F(v)$ pa kvečjemu linearna, dobimo pogoje le za dve 
notranji kontrolni točki, vse ostale pa so proste, podobno kot v primeru 
$C^1$-zveznosti (primer \ref{primer1}). Če za koeficientne funkcije izberemo polinome 
višjih stopenj, se lahko zgodi, da dobimo pogoje za tri ali štiri kontrolne točke. 

Najprej si oglejmo situacijo, v kateri za koeficientne funkcije izberemo polinome 
minimalne stopnje.

\begin{primer}\label{primer2}
  Da zagotovimo $G^1$-zveznost na stiku ploskev $\ddd{R}$ in $\ddd{S}$, mora 
  poleg pogoja, da se ploskvi stikata v robni krivulji, veljati enakost \eqref{eq15}, 
  oziroma
  \begin{align}\label{eq16} 
    D(v)\sum_{j=0}^3 \ddd{q}_j B_j^3(v) = E_1(v) \sum_{j=0}^3 \ddd{p}_j B_j^3(v) + 
    F_1(v) \sum_{j=0}^2 \ddd{z}_j B_j^2(v).
  \end{align}

  Ker je stopnja krivulj $\frac{\partial \ddd{S}}{\partial u}|_{u=0}$ in 
  $\frac{\partial \ddd{R}}{\partial u}|_{u=0}$ enaka 3, stopnja krivulje 
  $\frac{\partial \ddd{R}}{\partial v}|_{u=0}$ pa 2 in če njihovih stopenj 
  ne nižamo oziroma višamo, bodo stopnje polinomov $D(v)$, $E_1(v)$ in $F_1(v)$ 
  minimalne, če bosta $D(v)$ in $E_1(v)$ konstantna polinoma, $F_1(v)$ pa linearen. 
  V tem primeru namreč obe strani enačbe predstavljata Bézierjevo krivuljo stopnje 3.

  Brez škode za splošnost lahko izberemo, da je $D(v)\equiv 1$. Potem je 
  $E_1(v)\equiv e_0 = e_1$. Ker predpostavimo, da je $F_1(v)$ linearen in da 
  velja $F_1(0)=f_0$ ter $F_1(1)=f_1$, mora za $F_1$ veljati
  $$F_1(v) = f_0(1-v) + f_1v.$$

Vstavimo polinome $D$, $E_1$ in $F_1$ v enačbo \eqref{eq16} in dobimo:
\begin{align*}
  &\sum_{j=0}^3\ddd{q}_jB_j^3(v) = e_0\sum_{j=0}^3\ddd{p}_jB_j^3(v) + 
  (f_0(1-v)+f_1v)\sum_{j=0}^2\ddd{z}_jB_j^2(v)=\\
  &=e_0\sum_{j=0}^3\ddd{p}_jB_j^3(v) + \sum_{j=0}^2\ddd{z}_jf_0{2 \choose j}v^j(1-v)^{3-j} + 
  \sum_{j=0}^2\ddd{z}_jf_1{2 \choose j}v^{j+1}(1-v)^{2-j}=\\
  &=\sum_{j=0}^3e_0\ddd{p}_jB_j^3(v) + \sum_{j=0}^3\ddd{z}_jf_0{2 \choose j}v^j(1-v)^{3-j} + 
  \sum_{j=1}^3f_1\ddd{z}_{j-1}{2 \choose j-1}v^j(1-v)^{3-j}=\\
  &=\sum_{j=0}^3e_0\ddd{p}_jB_j^3(v) + \sum_{j=0}^3f_0\ddd{z}_j\frac{3-j}{3}{3 \choose j}v^j(1-v)^{3-j} + 
  \sum_{j=1}^3f_1\ddd{z}_{j-1}\frac{j}{3}{3 \choose j}v^3(1-v)^{3-j}=\\
  &=\sum_{j=0}^3e_0\ddd{p}_jB_j^3(v) + \sum_{j=0}^3f_0\ddd{z}_j\frac{3-j}{3}B_j^3(v) + 
  \sum_{j=0}^3f_1\ddd{z}_{j-1}\frac{j}{3}B_j^3(v)
\end{align*}
V tretji vrstici zgornjega računa smo uporabili dejstvo, da je v vsoti 
$$\sum_{j=0}^3\ddd{z}_jf_0{2 \choose j}v^j(1-v)^{3-j}$$ 
člen pri $j=3$ enak 0 
in je zato 
$$\sum_{j=0}^2\ddd{z}_jf_0{2 \choose j}v^j(1-v)^{3-j}=\sum_{j=0}^3\ddd{z}_jf_0{2 \choose j}v^j(1-v)^{3-j}.$$
Podobno smo v zadnji vrstici upoštevali, da je v vsoti 
$$\sum_{j=0}^3b_1\ddd{z}_{j-1}\frac{j}{3}B_j^3(v)$$ 
člen pri $j=0$ enak 0 in je zato 
$$\sum_{j=1}^3f_1\ddd{z}_{j-1}\frac{j}{3}{3 \choose j}v^3(1-v)^{3-j}=\sum_{j=0}^3b_1\ddd{z}_{j-1}\frac{j}{3}B_j^3(v).$$

Od tod dobimo pogoje za kontrolna vektorja $\ddd{q}_{1,1}$ in $\ddd{q}_{1,2}$:
\begin{align*}
\ddd{q}_{1,1}=e_0\ddd{p}_{1,1}+\frac{1}{3}f_1\ddd{z}_0+\frac{2}{3}f_0\ddd{z}_1,&&
\ddd{q}_{1,2}=e_0\ddd{p}_{1,2}+\frac{2}{3}f_1\ddd{z}_1+\frac{1}{3}f_0\ddd{z}_2.
\end{align*}
Najprej opazimo, da za razliko od primera \ref{primer1}, tu ni več potrebe po 
kolinearnosti vektorjev $\ddd{q}_{1,1}$ in $\ddd{p}_{1,1}$ oziroma 
vektorjev $\ddd{q}_{1,2}$ in $\ddd{p}_{1,2}$, zahtevamo le še koplanarnost. 
Ena izmed omejitev, ki veljajo za parametre $e_0$, $f_0$ in $f_1$, je, da 
mora biti $e_0<0$,
saj bi imel v nasprotnem primeru stik ploskev obliko špice. 
Parametri so določeni z enačbama \eqref{eq18} in \eqref{eq19} na naslednji način. 
Zapišimo enačbo \eqref{eq18} kot 
\begin{align*}
\ddd{q}_{0,0}=e_1\ddd{p}_{0,0}+f_1\ddd{z}_0+g\ddd{n},
\end{align*}
kjer $\ddd{n}$ označuje normalo na ravnino, ki jo določajo vektorji $\ddd{q}_{0,0}$, 
$\ddd{p}_{0,0}$ in $\ddd{z}_0$, v točki $(0,0)$. Matrika $[\ddd{p}_{0,0},\ddd{z}_0,\ddd{n}]$ 
je nesingularna, saj sta vektorja $\ddd{p}_{0,0}$ in $\ddd{z}_0$ nekolinearna in pravokotna 
na $\ddd{n}$, zato je mogoče na enoličen način izraziti neznanke $e_1$, $f_1$ in $g$. 
ker je vektor $\ddd{q}_{0,0}$ del iste ravnine kot $\ddd{p}_{0,0}$ in $\ddd{z}_0$, 
mora biti $g=0$, vrednosti $e_0$ in $f_0$ pa izračunamo s pomočjo Cramerjevih formul: 
\begin{align*}
e_0=\frac{\det[\ddd{q}_{0,0},\ddd{z}_0,\ddd{n}]}{\det[\ddd{p}_{0,0},\ddd{z}_0,\ddd{n}]},&&
f_0=\frac{\det[\ddd{p}_{0,0},\ddd{q}_{0,0},\ddd{n}]}{\det[\ddd{p}_{0,0},\ddd{z}_0,\ddd{n}]}.
\end{align*}
Na enak način iz enačbe \eqref{eq19} dobimo še parametra $e_1$ in $f_1$
\begin{align*}
  e_1=\frac{\det[\ddd{q}_{0,3},\ddd{z}_2,\ddd{n}]}{\det[\ddd{p}_{0,3},\ddd{z}_2,\ddd{n}]},&&
  f_1=\frac{\det[\ddd{p}_{0,3},\ddd{q}_{0,3},\ddd{n}]}{\det[\ddd{p}_{0,3},\ddd{z}_2,\ddd{n}]}.
\end{align*}

Če bi imeli robne krivulje ploskev izbrane na tak način, da bi veljalo 
$e_0=-1$, $f_0=0$ in $f_1=0$, bi dobili enak rezultat kot v primeru \ref{primer1}, 
v katerem smo iskali pogoje za $C^1$-zveznost. Z drugačno izbiro robnih krivulj in 
posledično drugačnimi vrednostmi parametrov pa 
lahko dosežemo poljuben kot med vektorjema $q_{1,1}$ in $p_{1,1}$, biti morata samo 
del iste ravnine.  
Vidimo torej, da je zahteva $G^1$-zveznosti kar se tiče oblike dobljenega zlepka 
ploskev prinese veliko več možnosti kot zahteva $C^1$-zveznosti.

V danem primeru, kjer so stopnje koeficientnih polinomov minimalne, je tudi število 
prostih kontrolnih točk enako kakor v primeru $C^1$-zveznosti. Kontrolni točki 
$\ddd{Q}_{1,1}$ in $\ddd{Q}_{1,2}$ sta točno določeni z izbiro točk $\ddd{P}_{1,1}$ 
in $\ddd{P}_{1,2}$, z robnimi kontrolnimi točkami ter izbiro parametrov $e_0$, 
$f_0$ in $f_1$, kontrolni točki $\ddd{P}_{1,1}$ in $\ddd{P}_{1,2}$ pa sta 
prosti. Enako velja za vse ostale notranje kontrolne točke. 
\end{primer}

\begin{primer}\label{primer3}
  Oglejmo si še nekoliko drugačen primer konstrukcije $G^1$-zveznih zlepkov dveh ploskev. 
  Stopnja polinoma $D(v)$ naj bo znova 0, stopnja $F_1(v)$ pa 1, medtem ko naj bo 
  polinom $E_1(v)$ 
  stopnje 1. Da bomo v tem primeru na obeh straneh enačbe \eqref{eq15} dobili 
  Bézierjevo krivuljo stopnje 3, moramo znižati stopnjo krivulje 
  $\frac{\partial \ddd{R}}{\partial u}|_{u=0}$. 
  Videli bomo, da v tem primeru sicer dobimo drugačne 
  možnosti, kar se tiče oblike, kakor v primeru \ref{primer2} \todo{več možnosti?}, 
  vendar se pri tem pojavi dodatna omejitev za notranje kontrolne točke.

  Naj krivuljo $\frac{\partial \ddd{R}}{\partial u}|_{u=0}$ 
  določajo kontrolni vektorji $\ddd{p}_0$, $\ddd{p}_m$ in $\ddd{p}_3$:

$$\frac{1}{3}\frac{\partial \ddd{R}}{\partial u}|_{u=0} = 
\sum_{i=0}^3\ddd{p}_{1,i}B_i^3(v)=(1-v)^2\ddd{p}_{1,0} + 2(1-v)v\ddd{p}_m + v^2\ddd{p}_{1,3}$$

Po formulah za višanje stopnje krivulje velja
$$\ddd{p}_{1,1}=\frac{2}{3}\ddd{p}_m+\frac{1}{3}\ddd{p}_{1,0}$$
$$\ddd{p}_{1,2}=\frac{2}{3}\ddd{p}_m+\frac{1}{3}\ddd{p}_{1,3},$$
oziroma
\begin{align}\label{eq17}
\ddd{p}_m=\frac{3}{2}\ddd{p}_{1,1}-\frac{1}{2}\ddd{p}_{1,0}=
\frac{3}{2}\ddd{p}_{1,2}-\frac{1}{2}\ddd{p}_{1,3}.
\end{align}

Enačba \eqref{eq15} se torej v tem primeru preoblikuje v
\begin{align}\label{eq16}
D(v)\sum_{j=0}^3\ddd{q}_{1,j}B_j^3(v)=E_1(v)\sum_{j=0}^2\tilde{\ddd{p}}_jB_j^2(v)+
F_1(v)\sum_{j=0}^2\ddd{s}_jB_j^2(v),
\end{align}
kjer smo označili $\tilde{\ddd{p}}_0=\ddd{p}_{1,0}$, $\tilde{\ddd{p}}_1=\ddd{p}_m$ in $\tilde{\ddd{p}}_2=\ddd{p}_{1,3}$.

Polinom $D(v)$ naj bo konstanten, znova lahko predpostavimo $D(v)\equiv 1$. Polinoma 
$E_1(v)$ in $F_1(v)$ naj bosta linearna, zanju naj velja še $E_1(0)=e_0$, 
$E_1(1)=e_1$, $F_1(0)=f_0$, $F_1(1)=f_1$. Torej mora veljati
$$E_1(v)=e_0(1-v)+e_1v$$
$$F_1(v)=f_0(1-v)+f_1v.$$

Vstavimo polinoma v enačbo \eqref{eq16} in na podoben način kot v primeru \ref{primer2} dobimo:
\begin{align*}
  &\sum_{j=0}^3\ddd{q}_{1,j}B_j^3(v)=(a_0(1-v)+a_1v)\sum_{j=0}^2\tilde{\ddd{p}}_jB_j^2(v)
  + (b_0(1-v)+b_1v)\sum_{j=0}^2\ddd{s}_jB_j^2(v) =\\
  &=a_0\sum_{j=0}^2\tilde{\ddd{p}}_j{2 \choose j}v^j(1-v)^{3-j} + 
  a_1\sum_{j=0}^2\tilde{\ddd{p}}_j{2 \choose j}v^{j+1}(1-v)^{2-j} +\\ 
  &+b_0\sum_{j=0}^2\tilde{\ddd{z}}_j{2 \choose j}v^j(1-v)^{3-j} + 
  b_1\sum_{j=0}^2\tilde{\ddd{z}}_j{2 \choose j}v^{j+1}(1-v)^{2-j}=\\
  &=\sum_{j=0}^3(a_1\tilde{\ddd{p}}_{j-1}\frac{j}{3}+a_0\tilde{\ddd{p}}_j\frac{3-j}{3}+
  b_1\ddd{z}_{j-1}\frac{j}{3}+b_0\ddd{z}_j\frac{3-j}{3})B_j^3(v).
\end{align*}
Od tod sledijo pogoji za vektorja $\ddd{q}_{1,1}$ in $\ddd{q}_{1,2}$. 
$$\ddd{q}_{1,1}=\frac{1}{3}(e_1\ddd{p}_{1,0}+2e_0\ddd{p}_m+f_1\ddd{z}_0+2f_0\ddd{z}_1)$$
$$\ddd{q}_{1,2}=\frac{1}{3}(e_0\ddd{p}_{1,3}+2e_1\ddd{p}_m+2f_1\ddd{z}_1+f_0\ddd{z}_2).$$
Če še izrazimo vektor $\ddd{p}_m$ z vektorjema $\ddd{p}_{1,0}$ in $\ddd{p}_{1,1}$ 
oziroma vektorjema $\ddd{p}_{1,2}$ in $\ddd{p}_{1,3}$, dobimo naslednji enačbi:

$$\ddd{q}_{1,1}=e_0\ddd{p}_{1,1}+\frac{1}{3}(e_1-e_0)\ddd{p}_{1,0}+\frac{1}{3}f_1\ddd{z}_0+\frac{2}{3}f_0\ddd{z}_1$$
$$\ddd{q}_{1,2}=e_1\ddd{p}_{1,2}+\frac{1}{3}(e_0-e_1)\ddd{p}_{1,3}+\frac{2}{3}f_1\ddd{z}_1+\frac{1}{3}f_0\ddd{z}_2.$$

%Vidimo, da za razliko od primera \ref{primer2}, kjer smo imeli zgolj 3 parametre 
%za določanje oblike in od teh le enega prostega, tu dobimo 4 parametre: $e_0$, $e_1$, $f_0$ in $f_1$. 
%Dva izmed njih sta določena z enačbama \eqref{eq18} in \eqref{eq19}, druga dva pa sta prosta. 
%S tem dobimo nove možnosti za obliko dobljene ploskve.

Da se izognemo možnosti, kjer ima dobljeni zlepek obliko špice, mora veljati 
omejitev $E_1(v)<0$ za $v\in [0,1]$, s čimer dobimo nekaj omejitev za izbiro 
parametrov $e_0$ in $e_1$. Polinom $E_1(v)$ zapišimo kot $E_1(v)=(e_1-e_0)v+e_0$. 
Na intervalu $[0,1]$ bo $E_1(v)<0$, če bo njegov maksimum na tem intervalu manjši od 0.  
Obravnavajmo dve možnosti. Prva možnost je, da je $e_1-e_0<0$ oziroma $e_1<e_0$. 
V tem primeru je $E_1(v)$ padajoča funkcija, zato ima na $[0,1]$ maksimum v $v=0$. 
Torej bo v tem primeru $E_1(v)<0$, če bo $E_1(0)=e_0<0$. 
Druga možnost je, da je $e_1-e_0>0$. V tem primeru je $E_1(v)$ naraščajoča funkcija 
in ima maksimum v $v=1$. Torej bo v tem primeru $E_1(v)<0$, če bo $E_1(1)=e_1<0$. 
Omejitev za parametra $e_0$ in $e_1$ je torej, da sta oba negativna. 

Parametri $e_0$, $e_1$, $f_0$ in $f_1$ so tudi v tem primeru določeni iz enačb \eqref{eq18} 
in \eqref{eq19} na enak način kot v primeru \ref{primer2}.

V trenutnem primeru imamo nekoliko manj svobode, kar se tiče izbire kontrolnih točk, 
kot v primerih \ref{primer1} in \ref{primer2}.  
Kontrolni točki $\ddd{Q}_{1,1}$ in $\ddd{Q}_{1,2}$ sta kot v primeru \ref{primer2} 
določeni s točkama $\ddd{P}_{1,1}$ in $\ddd{P}_{1,2}$ ter robnimi kontrolnimi 
točkami, kontrolni točki $\ddd{P}_{1,1}$ in $\ddd{P}_{1,2}$ pa nista več obe 
prosti. Prosta je le še ena izmed njiju, druga pa je določena z enačbo \eqref{eq17}.
\todo{to pomeni, da v splošnem ne moremo konstruirati ploskve S poljubne stopnje, 
če je R podana, saj dobimo pogoje za kontrolne točke R}
\end{primer}

\todo{primer s stopnjo 3 in 2? ali pa samo omeni, da je v tem primeru 
veliko več parametrov, ampak da so vse kontrolne točke odvisne od robnih,
polinomi višje stopnje se v bistvu ne splačajo? uporabni so samo, kadar 
imamo stik ploskev različnih stopenj}

%\todo{lahko narediš primer s stopnjami 3 in 2 in vidimo, da kontrolne točke 
%niso proste in je zato v splošnem nemogoče konstruirati bezierjevo ploskev 
%vnaprej določene stopnje, ki je g zvezna z neko vnaprej določeno bez ploskvijo, 
%ali kaj?}

%%%%%%%%%%%%%%%%%%%%%%%%%%%%%%%%%%%%%%%%%%%%%%%%%%%%%%%%%%%%%%%%%%%%%%%%%%%%%%%%%%%%%%%%%%%%%%%%%%%%%

\subsection{Konstrukcija $G^2$-zveznih Bézierovih ploskev iz tenzorskega produkta}
\todo{naredim tudi $C^2$?}

V tem podpoglavju si oglejmo še primer konstrukcije dveh ploskev, ki sta na 
skupnem robu $G^2$-zvezni. %\todo{še kaj?} 

Naj bosta $\ddd{R}$ in $\ddd{S}$ Bézierovi ploskvi iz tenzorskega produkta stopnje $(5,5)$: 

$$\ddd{R}(u,v)=\sum_{i=0}^5\sum_{j=0}^5\ddd{P}_{i,j}B_i^5(u)B_j^5(v)$$
in 
$$\ddd{S}(u,v)=\sum_{i=0}^5\sum_{j=0}^5\ddd{Q}_{i,j}B_i^5(u)B_j^5(v),$$
kjer je $u,v\in[0,1]$. Stikata naj se v krivulji $\ddd{C}(v)=\ddd{R}(0,v)=
\ddd{S}(0,v)$ s kontrolnimi točkami $\{\ddd{Z}_i; i=0,\ldots,5\}$, kjer je 
$\ddd{Z}_i=\ddd{P}_{0,i}=\ddd{Q}_{0,i}$. Tako kot v podpoglavju \ref{g1primeri} predpostavljajmo, da imamo 
že vnaprej določene robne krivulje obeh ploskev, znova nas zanima, kakšne pogoje 
prinese zahteva $G^2$-zveznosti za notranje kontrolne točke. Ker gre sedaj za zveznost 
stopnje 2, bomo poleg kontrolnih točk $\ddd{P}_{1,i}$ in $\ddd{Q}_{i,1}$, $i=1,2$, 
opazovali tudi kontrolne točke $\ddd{P}_{2,i}$ in $\ddd{Q}_{2,i}$, $i=1,2$. 
Poleg 
oznak za kontrolne vektorje $\ddd{p}_{1,j}$, $\ddd{q}_{1,j}$ in $\ddd{z}_{i}$, 
kjer je $j=0,\ldots,5$ in $i=0,\ldots,4$, že predstavljenih v podpoglavju \ref{g1primeri}, 
vpeljimo še oznake za kontrolne vektorje, ki nastopajo v drugih odvodih obeh ploskev. 
Naj bo 
\begin{align*}
&\ddd{p}_{2,j}=\ddd{P}_{2,j}-2\ddd{P}_{1,j}+\ddd{P}_{0,j},\\
&\ddd{q}_{2,j}=\ddd{Q}_{2,j}-2\ddd{Q}_{1,j}+\ddd{Q}_{0,j}\textrm{ za }j=0,\ldots,5,\\
&\ddd{s}_k=\ddd{P}_{k+2}-2\ddd{P}_{k+1}+\ddd{P}_{k}\textrm{ za }k=0,\ldots,3\\
&\textrm{ter}\\
&\ddd{v}_j=\ddd{P}_{1,j+1}-\ddd{P}_{0,j+1}-\ddd{P}_{1,j}+\ddd{P}_{0,j}\textrm{ za }j=0,\ldots,4.
\end{align*}

Po izreku \ref{izrek1} bosta ploskvi $\ddd{R}$ in $\ddd{S}$ $G^2$-zvezni na skupnem 
robu, natanko tedaj, ko bodo obstajale 
$C^2$-funkcije $\alpha_1(v)$, $\beta_1(v)$, $\alpha_2(v)$ in $\beta_2(v)$, 
kjer je $\alpha_1(v)\neq 0$ in ustreznega predznaka na intervalu $[0,1]$, da bo 
veljala enakost \eqref{eq33} ter 
\begin{equation}\label{eq34}
  \begin{split}
\frac{\partial^2\ddd{S}}{\partial u^2}(0,v)=&\alpha_2(v)\frac{\partial \ddd{R}}{\partial u}(0,v)
+\beta_2(v)\frac{\partial\ddd{R}}{\partial v}(0,v)+\alpha_1^2(v)\frac{\partial^2\ddd{R}}{\partial u^2}(0,v)+\\
&+2\alpha_1(v)\beta_1(v)\frac{\partial^2\ddd{R}}{\partial u \partial v}(0,v)+\beta_1^2(v)
\frac{\partial^2\ddd{R}}{\partial v^2}(0,v).
  \end{split}
\end{equation}
Ker gre znova za Bézierjeve ploskve iz tenzorskega produkta, lahko uporabimo 
izrek \ref{izrek2}, po katerem sta ploskvi $G^2$-zvezni na skupnem robu natanko tedaj, 
ko obstajajo polinomi $D(v)$, $E_1(v)$, $F_1(v)$, $E_2(v)$ in $E_2(v)$, kjer je 
$D(v)E_1(v)\neq 0$ in polinom $E_1(v)$ ustreznega predznaka na intervalu $[0,1]$, da 
velja enakost \eqref{eq14} ter 
\begin{equation}\label{eq35}
  \begin{split}
D^3(v)\frac{\partial^2\ddd{S}}{\partial u^2}(0,v)=&E_2(v)\frac{\partial \ddd{R}}{\partial u}(0,v)
+F_2(v)\frac{\partial\ddd{R}}{\partial v}(0,v)+D(v)E_1^2(v)\frac{\partial^2\ddd{R}}{\partial u^2}(0,v)+\\
&+2D(v)E_1(v)F_1(v)\frac{\partial^2\ddd{R}}{\partial u \partial v}(0,v)+D(v)F_1^2(v)
\frac{\partial^2\ddd{R}}{\partial v^2}(0,v).
  \end{split}
\end{equation}
Pri tem mora za stopnje polinomov veljati naslednje: $st(D)\leq 9$, $st(E_1)\leq 9$, 
$st(F_1)\leq 10$, $st(E_2)\leq 27$ in $st(F_2)\leq 28$. 

Zaradi enostavnosti naj bo stopnja polinoma $D(v)$ enaka 0. Torej lahko brez škode za 
splošnost predpostavimo, da je $D(v)\equiv 1$. Stopnje preostalih polinomov bomo 
izbrali v nadaljevanju.

Po enakosti \eqref{eq28} dobimo naslednje rezultate za parcialne odvode 
parametrizacij ploskev $\ddd{R}$ in $\ddd{S}$ v $u=0$: 

\begin{align*}
&\frac{\partial \ddd{R}}{\partial u}(0,v)=5\sum_{j=0}^5\ddd{p}_{1,j}B_j^5(v)&&
\frac{\partial \ddd{R}}{\partial v}(0,v)=5\sum_{j=0}^4\ddd{z}_jB_j^4(v)\\
&\frac{\partial^2\ddd{S}}{\partial u^2}(0,v)=20\sum_{j=0}^5\ddd{q}_{2,j}B_j^5(v)&&
\frac{\partial^2\ddd{R}}{\partial u^2}(0,v)=20\sum_{j=0}^5\ddd{p}_{2,j}B_j^5(v)\\
&\frac{\partial^2\ddd{R}}{\partial v^2}(0,v)=20\sum_{j=0}^3\ddd{s}_jB_j^3(v)&&
\frac{\partial^2\ddd{R}}{\partial u \partial v}(0,v)=25\sum_{j=0}^4\ddd{v}_jB_j^4(v).
\end{align*}

Dobljene izraze za odvode vstavimo v enačbi \eqref{eq14} in \eqref{eq35}. Oglejmo si, kakšne 
pogoje nam ti dve enačbi data pri vrednostih $v=0$ in $v=1$, torej kakšni pogoji 
morajo veljati na robovih obeh ploskev, da bo njun stik $G^2$-zvezen. 
Pri vrednosti $v=0$ dobimo
\begin{align}\label{eq36} 
\ddd{q}_{1,0}=e_{10}\ddd{p}_{1,0}+f_{10}\ddd{z}_0
\end{align}
in 
\begin{align}\label{eq37}
\ddd{q}_{2,0}=\frac{1}{4}e_{20}\ddd{p}_{1,0}+\frac{1}{4}f_{20}\ddd{z}_0+e_{10}^2
\ddd{p}_{2,0}+\frac{5}{2}e_{10}f_{10}\ddd{v}_0+f_{10}^2\ddd{s}_0.
\end{align}
Tu smo vpeljali oznake $e_{10}=E_1(0)$, $f_{10}=F_1(0)$, $e_{20}=E_2(0)$ in $f_{20}=F_2(0)$. 
Pri vrednosti $v=1$ pa dobimo 
\begin{align}\label{eq38} 
  \ddd{q}_{1,5}=e_{11}\ddd{p}_{1,5}+f_{11}\ddd{z}_4
\end{align}
in 
\begin{align}\label{eq39}
  \ddd{q}_{2,5}=\frac{1}{4}e_{21}\ddd{p}_{1,5}+\frac{1}{4}f_{21}\ddd{z}_4+e_{11}^2
  \ddd{p}_{2,5}+\frac{5}{2}e_{11}f_{11}\ddd{v}_4+f_{11}^2\ddd{s}_3,
\end{align}
kjer smo vpeljali oznake $e_{11}=E_1(1)$, $f_{11}=F_1(1)$, $e_{21}=E_2(1)$ in $f_{21}=F_2(1)$. 

Po izreku \ref{izrek2} mora za polinom $E_1(v)$ veljati, da je za $v\in [0,1]$ različen 
od 0, da zagotovimo, da ploskvi ne bosta imeli stika v obliki špice, pa mora veljati 
še $E_1(v)<0$ na intervalu $[0,1]$. Sledi, da mora veljati $e_{10}<0$ in $e_{11}<0$. 

Ker smo predpostavili, da imamo vnaprej določene robove obeh ploskev, so vsi kontrolni 
vektorji, ki nastopajo v zgornjih enačbah(\eqref{eq36}-\eqref{eq39}), razen vektorjev $\ddd{v}_0$ in $\ddd{v}_4$ 
znani. Enačbi \eqref{eq36} in \eqref{eq38} enolično določata vrednosti parametrov 
$e_{10}$, $f_{10}$, $e_{11}$ in $f_{11}$, izračunamo jih s pomočjo 
Cramerjevih formul, kot v primeru \ref{primer2}. Proste parametre oziroma 
kontrolne točke nam bosta prinesli le enačbi \eqref{eq37} in \eqref{eq39}. 

Najprej si oglejmo enačbo \eqref{eq37} in obravnavajmo dva primera. V prvem primeru 
naj bodo robovi krivulj izbrani na tak način, da 
je koeficient pred vektorjem $\ddd{v}_0$ enak 0, torej da je $f_{10}=0$,
v drugem primeru pa je različen od 0. 
Z $\ddd{n}_1$ bomo označili enotsko normalo na ploskev v točki $\ddd{P}_{0,0}$. 
Če je $f_{10}=0$, izgubimo člen, ki vsebuje $\ddd{v}_0$. To 
pomeni, da so vsi vektorji, ki nastopajo v enačbi, že določeni. V enačbi se nam pojavita 
le dva še neznana parametra $e_{20}$ in $f_{20}$, ki ju lahko določimo iz enačbe. %\todo{skalarni produkt z normalo?} 
V tem primeru nam enačba 
\eqref{eq27} ne da nobenih prostih parametrov ali kontrolnih točk. Če pa je $f_{10}\neq 0$, 
imamo na voljo dva različna scenarija. Prva možnost je, da projekcijo vektorja 
$\ddd{v}_0$ na normalo $\ddd{n}_1$ ter parametra $e_{20}$ in $f_{20}$ določimo iz 
enačbe \eqref{eq37}, projekcija vektorja $\ddd{v}_0$ na tangentno ravnino na ploskev v 
točki $\ddd{P}_{0,0}$ pa je prosta. 
Druga možnost pa je, da sta parametra $e_{20}$ in $f_{20}$ prosta, vektor $\ddd{v}_0$ 
pa je določen z enačbo \eqref{eq37}. 

Na enak način lahko analiziramo enačbo \eqref{eq39}. V prvem primeru, kjer je $f_{11}=0$, 
člen, ki vsebuje vektor $\ddd{v}_4$, izgine, parametra $e_{21}$ in $f_{21}$ pa sta 
določena z enačbo \eqref{eq39}. Če je $f_{11}\neq 0$ pa znova dobimo dve situaciji. Lahko 
določimo projekcijo vektorja $\ddd{v}_4$ na normalo $\ddd{n}_2$ na ploskev v točki 
$\ddd{P}_{0,5}$ ter parametra $e_{21}$ in $f_{2,1}$ iz enačbe \eqref{eq39}, projekcija 
vektorja $\ddd{v}_4$ na tangentno ravnino na ploskev v točki $\ddd{P}_{0,5}$ pa je 
prosta, ali pa sta prosta parametra $e_{21}$ in $f_{21}$, vektor $\ddd{v}_4$ pa je 
določen z enačbo \eqref{eq39}. 

Sedaj moramo izbrati polinome $E_1$, $E_2$, $F_1$ in $F_2$. Izbira njihovih stopenj je 
v primeru $G^2$-zveznosti nekoliko težja kot v primeru $G^1$-zveznosti, saj so 
odvisni od stopenj Bézierjevih krivulj tako v enačbi \eqref{eq14} kot v enačbi \eqref{eq35}. 
Če bi želeli minimalne stopnje koeficientnih polinomov, bi lahko za $E_1(v)$ in 
$E_2(v)$ izbrali konstanti, polinoma $F_1$ in $F_2$ pa bi izbrali linearna. V tem primeru 
bi za prvih nekaj kontrolnih točk ploskve $\ddd{S}$ dobili enake pogoje, kot v 
primeru \ref{primer2}, kontrolne točke ploskve $\ddd{R}$ pa bi bile proste. V nadaljevanju 
pa si bomo raje ogledali, kako poteka konstrukcija, če sta polinoma $E_1(v)$ in 
$E_2(v)$ stopnje 2, polinoma $F_1(v)$ in $F_2(v)$ pa linearna. Naj torej velja: 

\begin{align*}
  &E_1(v)=e_{10}(1-v)^2+2e_{1m}v(1-v)+e_{11}v^2\\
  &F_1(v)=f_{10}(1-v)+f_{11}v\\
  &E_2(v)=e_{20}(1-v)^2+2e_{1m}v(1-v)+e_{21}v^2\\
  &F_2(v)=f_{20}(1-v)+f_{21}v.\\
\end{align*}

Da se bodo pri taki izbiri koeficientnih polinomov stopnje Bézierjevih krivulj v 
enačbah \eqref{eq14} in \eqref{eq35} ujemale, moramo znižati stopnji parcialnih odvodov 
$\frac{\partial \ddd{R}}{\partial u}$ in $\frac{\partial^2 \ddd{R}}{\partial u^2}$, 
stopnja krivulje $\frac{\partial \ddd{R}}{\partial u}$ naj bo 3, stopnja krivulje 
$\frac{\partial^2 \ddd{R}}{\partial u^2}$ pa naj bo 1. 

\begin{align*}
  \frac{1}{5}\frac{\partial \ddd{R}}{\partial u}|_{u=0}=\sum_{i=0}^5\ddd{p}_{1,i}B_i^5(v)=
  \sum_{i=0}^3\ddd{\tilde{p}}_iB_i^3(v),
\end{align*}
\begin{align*}
\frac{1}{20}\frac{\partial^2 \ddd{R}}{\partial u^2}|_{u=0}=\sum_{i=0}^5\ddd{p}_{2,i}
=(1-v)\ddd{p}_{2,0}+v\ddd{p}_{2,5}.
\end{align*}

Kontrolne vektorje $\ddd{p}_{1,i}$, $i=0,\ldots,5$ lahko izrazimo z vektorji 
$\ddd{\tilde{p}}_i$ s pomočjo enačb za višanje stopnje Bézierjeve krivulje  
na naslednji način:
\begin{align}\label{eq40}
&\ddd{p}_{1,1}=\frac{1}{5}(3\ddd{\tilde{p}}_1+2\ddd{p}_{1,0}),&&
\ddd{p}_{1,2}=\frac{1}{10}(6\ddd{\tilde{p}}_1+3\ddd{\tilde{p}}_2+\ddd{p}_{1,0}),
\end{align}
\begin{align}\label{eq41}
&\ddd{p}_{1,3}=\frac{1}{10}(3\ddd{\tilde{p}}_1+6\ddd{\tilde{p}}_2+\ddd{p}_{1,5}),&&
\ddd{p}_{1,4}=\frac{1}{5}(3\ddd{\tilde{p}}_2+2\ddd{p}_1,5).
\end{align}
Seveda velja $\ddd{\tilde{p}}_0=\ddd{p}_{1,0}$, $\ddd{\tilde{p}_3}=\ddd{p}_{1,5}$. 
Iz enačb \eqref{eq40} lahko izrazimo še vektorja $\ddd{\tilde{p}}_1$ in $\ddd{\tilde{p}}_2$ kot 
\begin{align*}
\ddd{\tilde{p}}_1=\ddd{p}_{1,0}+\frac{5}{3}\ddd{v}_0,&&
\ddd{\tilde{p}}_2=\ddd{p}_{1,5}-\frac{5}{3}\ddd{v}_4.
\end{align*}
Tu smo uporabili dejstvo, da je $\ddd{v}_0=\ddd{p}_{1,1}-\ddd{p}_{10}$ in 
$\ddd{v}_4=\ddd{p}_{1,5}-\ddd{p}_{1,4}$. 

Enako lahko s pomočjo enačb za višanje stopnje kontrolne vektorje 
$\ddd{p}_{2,i}$, $i=0,\ldots,5$ izrazimo z vektorjema $\ddd{p}_{2,0}$ in $\ddd{p}_{2,5}$: 
\begin{align}
&\ddd{p}_{2,1}=\frac{1}{5}(4\ddd{p}_{2,0}+\ddd{p}_{2,5}),&&
\ddd{p}_{2,2}=\frac{1}{10}(6\ddd{p}_{2,0}+4\ddd{p}_{2,5}),\\
&\ddd{p}_{2,3}=\frac{1}{10}(4\ddd{p}_{2,0}+6\ddd{p}_{2,5}),&&
\ddd{p}_{2,4}=\frac{1}{5}(\ddd{p}_{2,0}+4\ddd{p}_{2,5}).
\end{align}
S temi enačbami so torej natančno določeni kontrolni vektorji $\ddd{p}_{i,j}$, 
$i=1,2$, $j=0,\ldots,5$ ploskve $\ddd{R}$, torej tudi njene kontrolne točke 
$\ddd{P}_{i,j}$, $i=0,1,2$, $j=0,\ldots,5$. Ostale notranje kontrolne točke so proste 
saj zahtevamo le geometrijsko zveznost stopnje 2. 

Za določitev pogojev, ki morajo veljati za kontrolne vektorje ploskve $\ddd{S}$ moramo 
uporabiti enačbi \eqref{eq14} in \eqref{eq35}, ki predstavljata pogoja za $G^1$ in $G^2$-zveznost. 
Ko v enačbo \eqref{eq14} vstavimo polinoma $E_1(v)$ in $F_1(v)$, preoblikujemo dobljeno 
enačbo, da na obeh straneh dobimo Bézierjevo krivuljo stopnje 5 ter primerjamo koeficiente 
pred baznimi polinomi, dobimo:

\begin{align*}
&\ddd{q}_{1,1}=\frac{1}{5}(3e_{10}\ddd{\tilde{p}}_1+2e_{1m}\ddd{p}_{1,0}+
f_{11}\ddd{z}_0+4f_{10}\ddd{z}_1),\\
&\ddd{q}_{1,2}=\frac{1}{10}(6e_{1m}\ddd{\tilde{p}}_1+3e_{10}\ddd{\tilde{p}}_2+
e_{11}\ddd{p}_{1,0}+4f_{11}\ddd{z}_1+6f_{10}\ddd{z}_2),\\
&\ddd{q}_{1,3}=\frac{1}{10}(3e_{11}\ddd{\tilde{p}}_1+6e_{1m}\ddd{\tilde{p}}_2+
e_{10}\ddd{p}_{1,5}+6f_{11}\ddd{z}_2+4f_{10}\ddd{z}_3),\\
&\ddd{q}_{1,4}=\frac{1}{5}(3e_{11}\ddd{\tilde{p}}_2+2e_{1m}\ddd{p}_{1,5}+
4f_{11}\ddd{z}_3+f_{10}\ddd{z}_4).
\end{align*}
Ko enak postopek ponovimo še z enačbo \eqref{eq35}, pa dobimo:

\begin{align*}
5\ddd{q}_{2,1}=&4e_{10}e_{1m}\ddd{p}_{2,0}+e_{10}^2\ddd{p}_{2,5}+e_{10}f_{10}(\ddd{\tilde{p}}_2-\ddd{\tilde{p}}_1)
+(2e_{1m}f_{10}+e_{10}f_{11})\ddd{v}_0+3f_{10}^2\ddd{s}_1+\\
&+2f_{10}f_{11}\ddd{s}_0+\frac{1}{4}(3e_{20}\ddd{\tilde{p}}_1+2e_{2m}\ddd{p}_{1,0}
+f_{21}\ddd{z}_0+4f_{20}\ddd{z}_1),
\end{align*}

\begin{align*}
10\ddd{q}_{2,2}=&(2e_{10}e_{11}+4e_{1m}^2)\ddd{p}_{2,0}+4e_{10}e_{1m}\ddd{p}_{2,5}+
+(e_{10}f_{11}+2e_{1m}f_{10})(\ddd{\tilde{p}}_2-\ddd{\tilde{p}}_1)+\\
&+(2e_{1m}f_{11}+e_{11}f_{10})\ddd{v}_0+e_{10}f_{10}\ddd{v}_4+3f_{10}^2\ddd{s}_2+
6f_{10}f_{11}\ddd{s}_1+f_{11}^2\ddd{s}_0+\\
&+\frac{1}{4}(6e_{2m}\ddd{\tilde{p}}_1+3e_{20}\ddd{\tilde{p}}_2+e_{21}\ddd{p}_{1,0}+
4f_{21}\ddd{z}_1+6f_{20}\ddd{z}_2),
\end{align*}

\begin{align*}
  10\ddd{q}_{2,3}=&(2p_{10}p_{11}+4p_{1m}^2)\ddd{p}_{2,5}+4p_{10}p_{1m}\ddd{p}_{2,0}+
  +(p_{11}q_{10}+2p_{1m}q_{11})(\ddd{\tilde{p}}_2-\ddd{\tilde{p}}_1)+\\
  &+(2p_{1m}q_{10}+p_{10}q_{11})\ddd{v}_4+p_{11}q_{11}\ddd{v}_0+3q_{11}^2\ddd{s}_1+
  6q_{10}q_{11}\ddd{s}_2+q_{10}^2\ddd{s}_3+\\
  &+\frac{1}{4}(6p_{2m}\ddd{\tilde{p}}_2+3p_{21}\ddd{\tilde{p}}_1+p_{20}\ddd{p}_{1,5}+
  4q_{21}\ddd{z}_3+6q_{21}\ddd{z}_2),
  \end{align*}

\begin{align*}
  5\ddd{q}_{2,4}=&4p_{10}p_{1m}\ddd{p}_{2,5}+p_{11}^2\ddd{p}_{2,5}+p_{11}q_{11}(\ddd{\tilde{p}}_2-\ddd{\tilde{p}}_1)
  +(2p_{1m}q_{11}+p_{11}q_{10})\ddd{v}_4+3q_{11}^2\ddd{s}_2+\\
  &+2q_{10}q_{11}\ddd{s}_3+\frac{1}{4}(3p_{21}\ddd{\tilde{p}}_2+2p_{2m}\ddd{p}_{1,5}
  +q_{21}\ddd{z}_3+4q_{20}\ddd{z}_4).
\end{align*}

Parametra $p_{1m}$ in $p_{2m}$ sta prosta. Po izbiri parametrov so točno določeni še 
kontrolni vektorji $\ddd{q}_{i,j}$, $i=1,2$, $j=0,\ldots,5$ ploskve $\ddd{S}$, 
oziroma kontrolne točke $\ddd{Q}_{i,j}$, $i=0,1,2$, $j=0,\ldots,5$. 

\todo{drugi možen pristop? če nista prosta p1m in p2m ampak projekcija enega od vektorjev?}

%%%%%%%%%%%%%%%%%%%%%%%%%%%%%%%%%%%%%%%%%%%%%%%%%%%%%%%%%%%%%%%%%%%%%%%%%%%%%%%%%%%%%%%%%%%%%%%%%%%%%%%%%%%%%%%
%%%%%%%%%%%%%%%%%%%%%%%%%%%%%%%%%%%%%%%%%%%%%%%%%%%%%%%%%%%%%%%%%%%%%%%%%%%%%%%%%%%%%%%%%%%%%%%%%%%%

\section{Konsrtrukcija $G^1$-zveznih trikotnih Bézierjevih ploskev}

\subsection{Trikotne Bézierjeve ploskve}\label{trikotne1}

\todo{povej, kje je možno najti dokaze trditev v tem podpoglavju}

Trikotne Bézierjeve ploskve so tip Bézierjevih ploskev s parametrizacijo, definirano 
nad domeno, sestavljeno iz trikotnikov. Ko imamo opravka s trikotnimi domenami, uporaba običajnega 
kartezičnega koordinatnega sistema ni najbolj praktična, zato bomo 
točke znotraj takih domen raje zapisovali z baricentričnimi koordinatami.

\begin{definicija}
Naj bo $T$ trikotnik z oglišči $\ddd{p}_1=(x_1,y_1)$, $\ddd{p}_2=(x_2,y_2)$ 
in $\ddd{p}_3=(x_3,y_3)$, oziroma $T=\langle\ddd{p}_1,\ddd{p}_2,\ddd{p}_3\rangle$. Velja, 
da je vsako točko $\ddd{p}=(x,y)$ mogoče enolično zapisati kot 
$\ddd{p}=u\ddd{p}_1+v\ddd{p}_2+w\ddd{p}_3$, kjer je $u+v+w=1$. Trojico
$\ddd{u}=(u,v,w)$ imenujemo \emph{baricentrične koordinate} točke $\ddd{p}$ glede 
na trikotnik $T$. Krajše to zapišemo kot $\ddd{u}=\mbox{Bar}(\ddd{p},T)$.
\end{definicija}

Trikotne Bézierove ploskve definiramo na naslednji način. 
\begin{definicija}
  \emph{Trikotna Bézierova ploskev stopnje n nad trikotnikom $T=\\\langle\ddd{p}_1,\ddd{p}_2,\ddd{p}_3\rangle$}\todo{lomljenje?} 
  je podana s parametrizacijo
  $$\ddd{R}_n(\ddd{p})=\sum_{i+j+k=n}\ddd{b}_{i,j,k}B_{i,j,k}^n(\ddd{u}),\quad \ddd{u}=\mbox{Bar}(\ddd{p},T)$$
  kjer so $\ddd{b}_{i,j,k}$, $i+j+k=n$, kontrolne točke te ploskve. Trikotnik $T$ je domena te parametrizacije. Z 
  $B_{i,j,k}^n$ označujemo Bernsteinove bazne polinome v dveh spremenljivkah, 
  ki so definirani na naslednji način:
  $$B_{i,j,k}^n(\ddd{u})=\frac{n!}{i!j!k!}u^iv^jw^k,$$
  kjer velja $i+j+k=n$ ter $i\geq0$, $j\geq0$ in $k\geq0$.
\end{definicija}

Koordinate poljubne točke na trikotni Bézierovi ploskvi lahko izračunamo s pomočjo 
de Casteljaujevega algoritma.

\todo{decasteljaujev algoritem? ali samo sklic na vir?}

V nadaljevanju si bomo ogledali še definicijo smernega odvoda trikotne Bézierjeve 
ploskve in pogoje za $C^n$-zveznost med dvema trikotnima Bézierjevima ploskvama, 
v ta namen pa moramo najprej definirati pojem trikotniškega razcveta. 
\begin{definicija}
  izberemo točke $\ddd{p}_1, \ldots, \ddd{p}_n$ z baricentričnimi koordinatami 
  $\ddd{u}_i=Bar(\ddd{p}_i;T)$, $i=1,\ldots,n$. Če v de Casteljaujevem algoritmu 
  na $i$-tem koraku namesto z baricentričnimi koordinatami $\ddd{u}$ računamo 
  z $\ddd{u}_i$, kot rezultat algoritma dobimo polinom več spremenljivk, ki ga 
  imenujemo \emph{trikotniški razcvet} in ga označujemo z 
  $\ddd{b}[\ddd{u}_1,\ldots,\ddd{u}_n;T]$ oziroma $\ddd{b}[\ddd{p}_1,\ldots,\ddd{p}_n]$. 
\end{definicija}

Kontrolne točke neke ploskve nad trikotnikom $T$ lahko izrazimo s trikotniškim 
razcvetom na naslednji način: 
$\ddd{b}_{i,j,k}=\ddd{b}[\ddd{e}_1^{\langle i\rangle},\ddd{e}_2^{\langle j\rangle},\ddd{e}_3^{\langle k\rangle}]$.

S pomočjo trikotniškega razcveta lahko izrazimo kontrolne točke ploskve glede na 
različne trikotnike v domeni. Imejmo trikotnika $T_1=\langle\ddd{p}_1,\ddd{p}_2,\ddd{p}_3\rangle$ in 
$T_2 = <\ddd{r}_1,\ddd{r}_2,\ddd{r}_3>$. 
Naj bodo $\ddd{b}_{i,j,k}$, $i+j+k=n$ kontrolne točke ploskve $\ddd{P}$ glede 
na trikotnik $T_1$ in naj bo $\ddd{s}_1=Bar(r_1;T_1)$, $\ddd{s}_2=Bar(r_2;T_1)$ ter 
$\ddd{s}_3=Bar(r_3;T_1)$. Potem se kontrolne točke ploskve $\ddd{P}$ glede na 
trikotnik $T_2$ izražajo na naslednji način: 
$c_{i,j,k} = \ddd{b}[\ddd{s}_1^{\langle i\rangle},\ddd{s}_2^{\langle j\rangle},\ddd{s}_3^{\langle k\rangle}]$, $i+j+k=n$. 
Tu s $\ddd{s}_1^{\langle i\rangle}$ označujemo $i$-kratno ponovitev $\ddd{s}_1$.

Če imamo domeno, sestavljeno iz dveh trikotnikov, ki se na enem robu stikata, 
torej če imamo trikotnika $T_1=\langle\ddd{p}_1,\ddd{p}_2,\ddd{p}_3\rangle$ in 
$T_2 = \langle\ddd{p}_2,\ddd{p}_3,\ddd{p}_4\rangle$, ter ploskev, podano s kontrolnimi točkami 
glede na trikotnik $T_1$: $b_{i,j,k},$ $i+j+k=n$,   
lahko kontrolne točke ploskve  
glede na trikotnik $T_2$ zapišemo tako: 
$\ddd{c}_{i,j,k}=\ddd{b}[\ddd{e}_1^{\langle i\rangle},\ddd{e}_2^{\langle j\rangle},\mathbf{\alpha}^{\langle k\rangle};T]$. 
Tu je $\mathbf{\alpha}=Bar(\ddd{p}_4;T_1)$, $\ddd{e}_1=(1,0,0)$ in $\ddd{e}_2=(0,1,0)$. 
%Ta ugotovitev bo uporabna pri iskanju pogojev za $C^1$-zveznost na stiku dveh 
%trikotnih ploskev. 

Oglejmo si še formulo za smerni odvod parametrizacije trikotne Bézierjeve ploskve. 
Imejmo ploskev $\ddd{P}(\ddd{p})=\sum_{i+j+k}\ddd{b}_{i,j,k}B_{i,j,k}^n(\ddd{u})$ 
nad trikotnikom $T$, kjer je $\ddd{u}=Bar(\ddd{p};T)$.  
Recimo, da odvajamo v smeri vektorja $\ddd{d}\in \R^3$. Naj bo $\ddd{d} = \ddd{a}-\ddd{b}$ ter 
$\ddd{\alpha}=Bar(\ddd{a};T)$ in $\ddd{\beta}=Bar(\ddd{b};T)$. Potem so 
baricentrične koordinate vektorja $\ddd{d}$ enake 
$\ddd{\mu}=(\mu_1,\mu_2,\mu_3)=\ddd{\beta}-\ddd{\alpha}$. Velja $\mu_1+\mu_2+\mu_3=0$. 
Odvod parametrizacije $\ddd{P}$ v smeri vektorja $\ddd{p}$ izračunamo na naslednji način:
\begin{align*}
D_{\ddd{p}}\ddd{P}(\ddd{p})&=n\sum_{i+j+k=n-1}(\mu_1\ddd{b}_{i+1,j,k}+
\mu_2\ddd{b}_{i,j+1,k}+\mu_3\ddd{b}_{i,j,k+1})B_{i,j,k}^{n-1}(\ddd{u})\\
&=n\ddd{b}[\mu,\ddd{u}^{\langle n-1\rangle};T].
\end{align*}

Pokazati je mogoče tudi, da za smerni odvod trikotne ploskve velja:
$$D_{\ddd{p}}\ddd{P}(\ddd{p})=n(\mu_1\ddd{b}_{1,0,0}^{n-1}(\ddd{u})+
\mu_2\ddd{b}_{0,1,0}^{n-1}(\ddd{u})+\mu_3\ddd{b}_{0,0,1}^{n-1}(\ddd{u})),$$
kjer so $\ddd{b}_{1,0,0}^{n-1}(\ddd{u})$, $\ddd{b}_{0,1,0}^{n-1}(\ddd{u})$ in 
$\ddd{b}_{0,0,1}^{n-1}(\ddd{u})$ točke, dobljene po $n-1$ korakih de Casteljaujevega 
algoritma, ki razpenjajo tangentno ravnino na točko $\ddd{P}(\ddd{p})$. 
Ta ugotovitev bo pomembna pri konstrukciji $G^1$-zveznih trikotnih ploskev. 

%%%%%%%%%%%%%%%%%%%%%%%%%%%%%%%%%%%%%%%%%%%%%%%%%%%%%%%%%%%%%%%%%%%%%%%%%%%%%%%%%%%%%%%%5

\subsection{Konstrukcija $C^1$-zveznih trikotnih Bézierovih ploskev}

Imejmo dve trikotni Bézierovi ploskvi $\ddd{R}$ in $\ddd{S}$ stopnje $n$ nad 
trikotnima domenama $T_1=\langle\ddd{p_1},\ddd{p_2},\ddd{p}_3\rangle$ in $T_2=\langle\ddd{p}_4,\ddd{p}_2,\ddd{p}_3\rangle$:
$$\ddd{R}(\ddd{p})=\sum_{i+j+k=n}\ddd{r}_{i,j,k}B_{i,j,k}^n(\ddd{u});\quad
\ddd{u}=\mbox{Bar}(\ddd{p};T_1),$$
$$\ddd{S}(\ddd{p})=\sum_{i+j+k=n}\ddd{s}_{i,j,k}B_{i,j,k}^n(\ddd{v});
\quad\ddd{v}=\mbox{Bar}(\ddd{p};T_2).$$
Zapišimo točki $\ddd{u}$ in $\ddd{v}$ kot $\ddd{u}=(u_1,u_2,u_3)$ in $\ddd{v}=(v_1,v_2,v_3)$. 
Ploskvi se stikata na robu nad daljico $\overline{\ddd{p}_2\ddd{p}_3}$, oziroma na 
robu, določenem z $u_1=v_1=0$.  
Torej velja 
$\ddd{r}_{0,j,k}=\ddd{s}_{0,j,k}$ za $j\geq 0$, $k\geq 0$, $j+k=0$. 

Da bo stik ploskev $\ddd{R}$ in $\ddd{S}$ $C^1$-zvezen, se morata v $u_1=0$ 
oziroma $v_1=0$ ujemati odvoda parametrizacij ploskev v katerikoli smeri. 
Opazujmo odvoda ploskev v smeri $\ddd{d}=\ddd{p_4}-\ddd{p}_2$. 
Naj bodo $\ddd{\mu}=(\mu_1,\mu_2,\mu_3)=\mbox{Bar}(\ddd{d};T_1)=\ddd{\alpha}-\ddd{e}_2$ baricentrične 
koordinate vektorja $\ddd{d}$ glede na trikotnik $T_1$, kjer je $\ddd{\alpha}=
\mbox{Bar}(\ddd{p}_4;T_1)$, in 
$\ddd{\eta}=(\eta_1,\eta_2,\eta_3)=\mbox{Bar}(\ddd{d};T_2)=\ddd{e}_1-\ddd{e}_2$ baricentrične 
koordinate vektorja $\ddd{d}$ glede na trikotnik $T_2$. Veljati mora:
\begin{align*}
&\sum_{i+j+k=n-1}(\mu_1\ddd{r}_{i+1,j,k}+
\mu_2\ddd{r}_{i,j+1,k}+\mu_3\ddd{r}_{i,j,k+1})B_{i,j,k}^{n-1}(\ddd{u})|_{u_1=0}=\\
&\sum_{i+j+k=n-1}(\eta_1\ddd{s}_{i+1,j,k}+
\eta_2\ddd{s}_{i,j+1,k}+\eta_3\ddd{s}_{i,j,k+1})B_{i,j,k}^{n-1}(\ddd{v})|_{v_1=0}
\end{align*}\todo{ok |u1=0?}
oziroma
\begin{align}\label{eq20}
\mu_1\ddd{r}_{1,j,k}+\mu_2\ddd{r}_{0,j+1,k}+\mu_3\ddd{r}_{0,j,k+1}=
\eta_1\ddd{s}_{1,j,k}+\eta_2\ddd{s}_{0,j+1,k}+\eta_3\ddd{s}_{0,j,k+1}
\end{align}
za $j+k=n-1$.
Točki $\ddd{u}$ in $\ddd{v}$ se namreč na stiku obeh ploskev ujemata. Če izrazimo 
baricentrične koordinate vektorjev s pomočjo baricentričnih koordinat točk, 
torej če pišemo $\ddd{\mu}=(\alpha_1,\alpha_2-1,\alpha_3)$ in $\ddd{\eta}=(1,-1,0)$ 
ter dobljeno vstavimo v enačbo \eqref{eq20}, ob tem pa upoštevamo še, da se 
kontrolne točke obeh ploskev na skupnem robu ujemajo, dobimo
\begin{align}\label{eq21}
  \ddd{s}_{1,j,k} = \alpha_1\ddd{r}_{1,j,k}+\alpha_2\ddd{r}_{0,j+1,k}+\alpha_3\ddd{r}_{0,j,k+1}
\end{align}
za $j+k=n-1$. 
Enačba \eqref{eq21} predstavlja pogoj, ki mora veljati za kontrolne točke obeh 
ploskev, da se stikata s $C^1$-zveznostjo. Do enakega rezultata bi na podoben 
način prišli, če bi namesto vektorja $\ddd{d}$ uporabili vektor $\ddd{p}_4-\ddd{p}_3$ 
katerikoli linearno kombinacijo obeh vektorjev.

%Alternativni način za izpeljavo pogojev \eqref{eq21}
Dobljeni rezultat sedaj uporabimo na primeru stika dveh ploskev stopnje 3.

\begin{primer}\label{primerT1}
Imejmo dve trikotni Bézierovi ploskvi $\ddd{R}$ in $\ddd{S}$ stopnje 3 nad 
trikotnima domenama $T_1=\langle\ddd{p_1},\ddd{p_2},\ddd{p}_3\rangle$ in $T_2=\langle\ddd{p}_4,\ddd{p}_2,\ddd{p}_3\rangle$:
$$\ddd{R}(\ddd{p})=\sum_{i+j+k=3}\ddd{r}_{i,j,k}B_{i,j,k}^3(\ddd{u});
\textrm{ }\ddd{u}=\mbox{Bar}(\ddd{p};T_1),$$
$$\ddd{S}(\ddd{p})=\sum_{i+j+k=3}\ddd{s}_{i,j,k}B_{i,j,k}^3(\ddd{v});
\textrm{ }\ddd{v}=\mbox{Bar}(\ddd{p};T_2).$$
Naj bo $\ddd{u}=(u_1,u_2,u_3)$ in $\ddd{v}=(v_1,v_2,v_3)$ ter 
$\ddd{\alpha}=(\alpha_1,\alpha_2,\alpha_3)=\mbox{Bar}(\ddd{p}_4;T_1)$.
Ploskvi se kakor prej stikata v $u_1=v_1=0$, torej za kontrolne točke na skupnem 
robu velja $\ddd{s}_{0,0,2}=\ddd{r}_{0,0,2}$, $\ddd{s}_{0,1,1}=\ddd{r}_{0,1,1}$ in 
$\ddd{s}_{0,2,0}=\ddd{r}_{0,2,0}$. 

Da bo stik obeh ploskev še $C^1$ zvezen mora po enačbi \eqref{eq21} veljati:
\begin{align*}
&\ddd{s}_{1,0,2}= \alpha_1\ddd{r}_{1,0,2}+\alpha_2\ddd{r}_{0,1,2}+\alpha_3\ddd{r}_{0,0,3}\\
&\ddd{s}_{1,1,1}= \alpha_1\ddd{r}_{1,1,1}+\alpha_2\ddd{r}_{0,2,1}+\alpha_3\ddd{r}_{0,1,2}\\
&\ddd{s}_{1,2,0}= \alpha_1\ddd{r}_{0,1,2}+\alpha_2\ddd{r}_{0,3,0}+\alpha_3\ddd{r}_{0,2,1}
\end{align*}
\todo{primerjava kontrolnih točk, katere so določene s katerimi?}

Opazimo lahko, da morata biti točki $\ddd{s}_{1,j,k}$ in $\ddd{r}_{1,j,k}$, kjer 
je $j+k=2$ in $j\geq 0$, $k\geq 0$, kolinearni. Še več, trikotnika 
$\langle\ddd{r}_{1,0,2},\ddd{r}_{0,1,2},\ddd{r}_{0,0,3}\rangle$ in 
$\langle\ddd{s}_{0,1,2},\ddd{s}_{0,0,3},\ddd{s}_{1,0,2}\rangle$ morata biti afini sliki 
domenskih trikotnikov $T_1$ in $T_2$. Enako mora veljati tudi za trikotnika 
$\langle\ddd{r}_{1,1,1},\ddd{r}_{0,2,1},\ddd{r}_{0,1,2}\rangle$ in 
$\langle\ddd{s}_{0,2,1},\ddd{s}_{0,1,2},\ddd{s}_{1,1,1}\rangle$ ter trikotnika 
$\langle\ddd{r}_{1,2,0},\ddd{r}_{0,3,0},\ddd{r}_{0,2,1}\rangle$ in 
$\langle\ddd{s}_{0,3,0},\ddd{s}_{0,2,1},\ddd{s}_{1,2,0}\rangle$. 
\end{primer}

Izbira kontrolnih točk $C^1$-zveznih ploskev je torej v veliki meri odvisna od 
domenskih trikotnikov, kar nas precej omejuje pri konstrukciji. 
Oglejmo si primer, kjer imamo podane robne točke dveh ploskev, ni pa 
možno najti notranjih kontrolnih točk, da bi bil stik ploskev $C^1$-zvezen. 

\begin{primer}\label{primerT2}
Imejmo trikotni ploskvi $\ddd{R}$ in $\ddd{S}$ kot v primeru \ref{primerT1}. 
Predpostavljajmo, da imamo že vnaprej določene njune robne 
točke, tako da so robovi dobljenega zlepka 
$C^1$-zvezni. \todo{dodaj: kaj mora veljati, da so robovi c1?}
Naj velja
$$\ddd{s}_{1,0,2}= \alpha_1\ddd{r}_{1,0,2}+\alpha_2\ddd{r}_{0,1,2}+\alpha_3\ddd{r}_{0,0,3}$$
in
$$\ddd{s}_{1,2,0}= \beta_1\ddd{r}_{0,1,2}+\beta_2\ddd{r}_{0,3,0}+\beta_3\ddd{r}_{0,2,1},$$
kjer je $(\alpha_1,\alpha_2,\alpha_3)\neq(\beta_1,\beta_2,\beta_3)$.  
V tem primeru je nemogoče določiti točki $\ddd{r}_{1,1,1}$ in $\ddd{s}_{1,1,1}$, da bi 
bil dobljeni zlepek $C^1$-zvezen, saj para trikotnikov 
$\langle\ddd{r}_{1,0,2},\ddd{r}_{0,1,2},\ddd{r}_{0,0,3}\rangle$ in 
$\langle\ddd{s}_{0,1,2},\ddd{s}_{0,0,3},\ddd{s}_{1,0,2}\rangle$ ter 
$\langle\ddd{r}_{1,2,0},\ddd{r}_{0,3,0},\ddd{r}_{0,2,1}\rangle$ in 
$\langle\ddd{s}_{0,3,0},\ddd{s}_{0,2,1},\ddd{s}_{1,2,0}\rangle$ 
nista afini sliki istega para domenskih trikotnikov. 
\end{primer}

V zgornjem primeru smo videli, da je pogoj $C^1$-zveznosti za trikotne Bézierjeve 
ploskve dokaj strog in tesno povezan z izbiro domene. Geometrijska zveznost pa 
nam da milejše pogoje, saj nimamo več odvisnosti od domene. Če bi v zgornjem 
primeru zahtevali le $G^1$-zveznost, bi bilo mogoče določiti točki 
$\ddd{r}_{1,1,1}$ in $\ddd{s}_{1,1,1}$. 

Oglejmo si še en zelo enostaven primer ploskev, ki na stiku zaradi izbire domene ne moreta 
biti $C^1$-zvezni, lahko pa sta $G^1$-zvezni. 

\begin{primer}\label{primerT3}
Imejmo dve trikotni Bézierovi ploskvi stopnje 1:
$$\ddd{R}(\ddd{p})=\ddd{r}_{1,0,0}u+\ddd{r}_{0,1,0}v+\ddd{r}_{0,0,1}w; 
\quad(u,v,w)=\mbox{Bar}(\ddd{p};T_1)$$
in
$$\ddd{S}(\ddd{p})=\ddd{s}_{1,0,0}\tilde{u}+\ddd{s}_{0,1,0}\tilde{v}+
\ddd{s}_{0,0,1}\tilde{w};
\quad(\tilde{u},\tilde{v},\tilde{w})=\mbox{Bar}(\ddd{p};T_2).$$
Naj velja $\ddd{r}_{1,0,0}=(0,1,0)$, $\ddd{r}_{0,1,0}=\ddd{s}_{0,1,0}=(0,0,0)$, 
$\ddd{r}_{0,0,1}=\ddd{s}_{0,0,1}=(1,1,0)$ in $\ddd{s}_{0,0,1}=(1,0,0)$. Ploskvi 
$\ddd{R}$ in $\ddd{S}$ sta torej dva pravokotna trikotnika v ravnini $z=0$, ki 
skupaj tvorita kvadrat. 
Trikotnika $T_1$ in $T_2$ definirajmo kot $T_1=\langle(0,1),(0,0),(1,0)\rangle$ in 
$T_2=\\\langle(0,1),(0,0),(1,0)\rangle$. \todo{prelom?}

Očitno je, da je stik ploskev $\ddd{R}$ in $\ddd{S}$ $G^1$-zvezen, saj imata 
ploskvi vzdolž stične krivulje isto konstatno tangentno ravnino. 
Vendar pa stik teh dveh ploskev ni $C^1$-zvezen. 
Naj bo $\mathbf{\alpha}=\mbox{Bar}(\ddd{s}_{1,0,0};T_1)$. Torej je 
$\mathbf{\alpha}=(-1,2,0)$. Da bi bil stik ploskev $C^1$-zvezen, bi moralo biti 
zadoščeno enačbi \eqref{eq21}, torej bi moralo veljati 
$$\ddd{s}_{1,0,0}=-\ddd{r}_{1,0,0}+2\ddd{r}_{0,1,0},$$
kar pa v našem primeru ne drži.
\end{primer}

%%%%%%%%%%%%%%%%%%%%%%%%%%%%%%%%%%%%%%%%%%%%%%%%%%%%%%%%%%%%%%%%%%%%%%%%%%%%%%%%%%%%%%%%%%%%

\subsection{Konstrukcija $G^1$-zveznih trikotnih Bézierovih ploskev}

Tako kot v primeru Bézierovih ploskev iz tenzorskega produkta, obstaja več 
načinov konstrukcije $G^1$-zveznih trikotnih Bézierovih ploskev, odvisno 
od izbire stičnih funkcij. V tem podpoglavju si bomo ogledali enega izmed 
načinov konstrukcije dveh trikotnih Bézierovih ploskev, ki sta na skupnem robu 
zvezni, vendar pa se tega ne bomo lotili prek osnovne definicije geometrijske 
zveznosti oziroma izreka \ref{izrek1}, temveč prek geometrijske definicije.

Imejmo ploskvi $\ddd{R}$ in $\ddd{S}$ stopnje $n$ nad 
trikotnima domenama $T_1=\langle\ddd{p_1},\ddd{p_2},\ddd{p}_3\rangle$ in $T_2=\langle\ddd{p}_4,\ddd{p}_2,\ddd{p}_3\rangle$:
$$\ddd{R}(\ddd{p})=\sum_{i+j+k=n}\ddd{r}_{i,j,k}B_{i,j,k}^n(\ddd{u});
\quad\ddd{u}=\mbox{Bar}(\ddd{p};T_1),$$
$$\ddd{S}(\ddd{p})=\sum_{i+j+k=n}\ddd{s}_{i,j,k}B_{i,j,k}^n(\ddd{\tilde{u}});
\quad\ddd{\tilde{u}}=\mbox{Bar}(\ddd{p};T_2).$$
Naj bo $\ddd{u}=(u,v,w)$ in $\ddd{\widetilde{u}}=(\tilde{u},\tilde{v},\tilde{w})$. 
Ploskvi se stikata na robu nad daljico $\overline{\ddd{p}_2\ddd{p}_3}$, oziroma na 
robu, določenem z $u=\tilde{u}=0$.  
Torej velja 
$\ddd{r}_{0,j,k}=\ddd{s}_{0,j,k}$ za $j\geq 0$, $k\geq 0$, $j+k=0$. 
Stično krivuljo lahko zapišemo kot $C(v)=\sum_{j=0}^n\ddd{r}_{0,j,n-j}B_j^n(v)$, 
kjer je $v\in[0,1]$. 

Naj bo $\ddd{C}(v)$ točka na stični krivulji pri parametru $v$. 
V poglavju \ref{trikotne1} smo 
videli, da je mogoče tangentno ravnino na Bézierovo ploskev v neki točki 
izraziti s pomočjo točk, ki jih dobimo v predzadnjem koraku de Casteljaujevega 
algoritma. Tangentno ravnino na ploskev $\ddd{R}$ v točki $\ddd{C}(v)$ razpenjajo 
točke $\ddd{r}_{1,0,0}^{n-1}(v)$, $\ddd{r}_{0,1,0}^{n-1}(v)$ in $\ddd{r}_{0,0,1}^{n-1}(v)$, 
Tangentno ravnino na ploskev $\ddd{S}$ v točki $\ddd{C}(v)$ pa razpenjajo točke 
$\ddd{s}_{1,0,0}^{n-1}(v)$, $\ddd{s}_{0,1,0}^{n-1}(v)$ in $\ddd{s}_{0,0,1}^{n-1}(v)$. 
Da bo zlepek obeh ploskev $G^1$-zvezen, morata biti obe tangentni ravnini 
del ene ravnine. To pomeni, da se morata premici 
$\overline{\ddd{r}_{0,1,0}^{n-1}(v)\ddd{r}_{0,0,1}^{n-1}(v)}$ in 
$\overline{\ddd{r}_{1,0,0}^{n-1}(v)\ddd{s}_{1,0,0}^{n-1}(v)}$ sekati za vsako 
vrednost parametra $v\in[0,1]$. Torej morata obstajati funkciji 
$\lambda(v)$ in $\mu(v)$, za kateri velja 
\begin{align}\label{eq22}
  (1-\lambda(v))\ddd{s}_{1,0,0}^{n-1}(v)+\lambda(v)\ddd{r}_{1,0,0}^{n-1}=
  (1-\mu(v))\ddd{r}_{0,0,1}^{n-1}+\mu(v)\ddd{r}_{0,1,0}^{n-1}.
\end{align}
S tem smo med drugim zagotovili tudi, da se točki $\ddd{s}_{1,0,0}^{n-1}(v)$ 
in $\ddd{r}_{1,0,0}^{n-1}(v)$ za vsako vrednost $v\in[0,1]$ nahajata na 
različnih straneh robne krivulje. Dobljeni zlepek tako ne bo imel oblike špice. 
Veljati mora še, da sta funkciji $\lambda(v)$ in $\mu(v)$ na intervalu $[0,1]$ 
različni od 0 in 1, da tangentne ravnine na $\ddd{C}(v)$ niso izrojene. 
\todo{poveži s pogojem iz def, da mora biti funkcija različna od 0}

Zgornja enačba predstavlja pogoj, ki zagotavlja $G^1$-zveznost zlepka ploskev. 
\todo{ni pa to potreben pogoj?} 

Iz de Casteljaujevega algoritma sledi, da lahko točke, ki razpenjajo tangenti 
ravnini zapišemo na naslednji način
\begin{align*}
&\ddd{r}_{1,0,0}^{n-1}(v)=\sum_{i=0}^{n-1}\ddd{r}_{1,i,n-i-1}B_i^{n-1}(v)&&
&\ddd{r}_{0,1,0}^{n-1}(v)=\sum_{i=0}^{n-1}\ddd{r}_{0,i+1,n-i-1}B_i^{n-1}(v)\\
&\ddd{r}_{0,0,1}^{n-1}(v)=\sum_{i=0}^{n-1}\ddd{r}_{0,i,n-i}B_i^{n-1}(v)&&
&\ddd{s}_{1,0,0}^{n-1}(v)=\sum_{i=0}^{n-1}\ddd{s}_{1,i,n-i-1}B_i^{n-1}(v).
\end{align*}
Zaradi ujemanja kontrolnih točk na skupnem robu, je 
$\ddd{s}_{1,0,0}^{n-1}(v)=\ddd{r}_{1,0,0}^{n-1}(v)$ in \\
$\ddd{s}_{0,1,0}^{n-1}(v)=\ddd{r}_{0,1,0}^{n-1}(v)$. 

%Za boljšo preglednost vpeljimo oznake $\ddd{a}_i=\ddd{s}_{1,i,n-i-1}$, 
%$\ddd{b}_i=\ddd{r}_{0,i,n-i}$ in $\ddd{c}_i=\ddd{r}_{1,i,n-i-1}$. 

Najprej si oglejmo, kakšne pogoje nam da enačba \eqref{eq22} za robne kontrolne točke. 
Vstavimo v enačbo \eqref{eq22} vrednosti $v=0$ in $v=1$. 
Pri vrednosti $v=0$ dobimo: 
$$(1-\lambda_0)\ddd{s}_{1,0,n-1}+\lambda_0\ddd{r}_{1,0,n-1}=
(1-\mu_0)\ddd{r}_{0,0,n}+\mu_0\ddd{r}_{0,1,n-1},$$
kjer smo vpeljali oznaki $\lambda_0=\lambda(0)$ in $\mu_0=\mu(0)$. 
Pri vprednosti $v=1$ pa dobimo:
$$(1-\lambda_1)\ddd{s}_{1,n-1,0}+\lambda_1\ddd{r}_{1,n-1,0}=
(1-\mu_1)\ddd{r}_{0,n-1,1}+\mu_1\ddd{r}_{0,n,0},$$
kjer smo vpeljali še oznaki $\lambda_1=\lambda(1)$ in $\mu_1=\mu(1)$. 
Vrednosti $\lambda_0$ in $\mu_0$ opisujeta obliko prvega para trikotnikov 
v kontrolni mreži ob skupnem robu, vrednosti $\lambda_1$ in $\mu_1$ pa zadnji 
par trikotnikov. Vidimo, da oblika obeh parov ni več nujno enaka, torej para 
trikotnikov nista več nujno afini sliki istega para domenskih trikotnikov. 
Že tu vidimo, da so pogoji, ki jih zahteva $G^1$-zveznost, milejši od zahteve 
$C^1$-zveznosti. Posledično je, kot smo videli že v primeru ploskev iz tenzorskega 
produkta, mogoča konstrukcija ploskev veliko več različnih oblik. 

Sedaj za funkciji $\lambda(v)$ in $\mu(v)$ izberimo polinoma prve stopnje
%(možna bi bila tudi drugačna izbira funkcij, ki bi seveda privedla do 
%drugačnih rezultatov)
\begin{align*}
\lambda(v)=(1-v)\lambda_0+v\lambda_1,&&
\mu(v)=(1-v)\mu_0+v\mu_1.
\end{align*}

Da si olajšajmo nadaljnje delo, najprej nekoliko preoblikujmo izraza 
$1-\lambda(v)$ in $1-\mu(v)$. 

\begin{align*}
1-\lambda(v)&=1-\lambda_0(1-v)-\lambda_1v=1-\lambda_0+\lambda_0v-\lambda_1v=\\
&=(1-\lambda_0)(1-v)+(1-\lambda_1)v
\end{align*}
Na enak način dobimo 
$$1-\mu(v)=(1-\mu_0)(1-v)+(1-\mu_1)v$$

Polinoma vstavimo v enačbo \eqref{eq22}. Najprej si oglejmo izraz \\
$(1-\lambda(v))\sum_{i=0}^{n-1}\ddd{s}_{1,i,n-i-1}B_i^{n-1}$:
\begin{align*}
  &(1-\lambda(v))\sum_{i=0}^{n-1}\ddd{s}_{1,i,n-i-1}B_i^{n-1}=\\
  &=(1-\lambda_0)(1-v)\sum_{i=0}^{n-1}\ddd{s}_{1,i,n-i-1}B_i^{n-1}+
  (1-\lambda_1)v\sum_{i=0}^{n-1}\ddd{s}_{1,i,n-i-1}B_i^{n-1}=\\
  &=(1-\lambda_0)\sum_{i=0}^{n-1}\ddd{s}_{1,i,n-i-1}\frac{(n-1)!}{i!(n-i-1)!}v^i(1-v)^{n-i}+\\
  &+(1-\lambda_1)\sum_{i=0}^{n-1}\ddd{s}_{1,i,n-i-1}\frac{(n-1)!}{i!(n-i-1)!}v^{i+1}(1-v)^{n-i-1}=\\
  &=(1-\lambda_0)\sum_{i=0}^{n-1}\ddd{s}_{1,i,n-i-1}\frac{n-i}{n}B_i^n(v)+
  (1-\lambda_1)\sum_{i=0}^{n-1}\ddd{s}_{1,i-1,n-i}\frac{i}{n}B_i^n(v)=\\
  &=(1-\lambda_0)\sum_{i=0}^{n}\ddd{s}_{1,i,n-i-1}\frac{n-i}{n}B_i^n(v)+
  (1-\lambda_1)\sum_{i=0}^{n-1}\ddd{s}_{1,i-1,n-i}\frac{i}{n}B_i^n(v).
\end{align*}
\todo{ali naj raje to nekje izpeljem posebej in se potem samo sklicujem?}

Na enak način dobimo še 
\begin{align*}
&\lambda(v)\sum_{i=0}^{n-1}\ddd{r}_{1,i,n-i-1}B_i^{n-1}=
\lambda_0\sum_{i=0}^{n}\ddd{r}_{1,i,n-i-1}\frac{n-i}{n}B_i^n(v)+
\lambda_1\sum_{i=0}^{n-1}\ddd{r}_{1,i-1,n-i}\frac{i}{n}B_i^n(v),\\
&(1-\mu(v))\sum_{i=0}^{n-1}\ddd{r}_{0,i,n-i}B_i^{n-1}=
(1-\mu_0)\sum_{i=0}^{n}\ddd{r}_{0,i,n-i}\frac{n-i}{n}B_i^n(v)+\\
&+(1-\mu_1)\sum_{i=0}^{n-1}\ddd{r}_{0,i-1,n-i+1}\frac{i}{n}B_i^n(v)\textrm{ in }\\
&\mu(v)\sum_{i=0}^{n-1}\ddd{r}_{0,i+1,n-i-1}B_i^{n-1}=
\mu_0\sum_{i=0}^{n}\ddd{r}_{0,i+1,n-i-1}\frac{n-i}{n}B_i^n(v)+
\mu_1\sum_{i=0}^{n-1}\ddd{r}_{0,i,n-i}\frac{i}{n}B_i^n(v).
\end{align*}

Dobljeno vstavimo v enačbo \eqref{eq22} in primerjajmo člene ob baznih polinomih 
$B_i^n(v)$ za vsak $i=0,\ldots,n$. 
Dobimo, da mora za vsak $i=0,\ldots,n$ veljati enakost 
\begin{equation}\label{eq23}
  \begin{split}
&(1-\lambda_0)\ddd{s}_{1,i,n-i-1}\frac{n-i}{n}+
(1-\lambda_1)\ddd{s}_{1,i-1,n-i}\frac{i}{n}+
\lambda_0\ddd{r}_{1,i,n-i-1}\frac{n-i}{n}+
\lambda_1\ddd{r}_{1,i-1,n-i}\frac{i}{n}=\\
&=(1-\mu_0)\ddd{r}_{0,i,n-i}\frac{n-i}{n}
+(1-\mu_1)\ddd{r}_{0,i-1,n-i+1}\frac{i}{n}+
\mu_0\ddd{r}_{0,i+1,n-i-1}\frac{n-i}{n}+
\mu_1\ddd{r}_{0,i,n-i}\frac{i}{n}.
  \end{split}
\end{equation}

Pogoji, ki jih dobimo z enačbo \eqref{eq23} za kontrolne točke ploskev $\ddd{R}$ 
in $\ddd{S}$ zagotavljajo, da sta ploskvi na stiku $G^1$-zvezni. Seveda pa so to 
le zadostni pogoji, ne pa nujno potrebni. Z drugačno izbiro funkcij $\lambda(v)$ 
in $\mu(v)$ bi lahko prišli do drugačnih pogojev, ki bi imeli za rezultat 
$G^1$-ploskve drugačnih oblik.

Sedaj si natančneje oglejmo, kakšne pogoje za kontrolne točke bi dobili v tem 
primeru za ploskvi stopnje 3. 

\begin{primer}\label{primerT4}
  Imejmo ploskvi $\ddd{R}$ in $\ddd{S}$ stopnje 3 nad 
  trikotnima domenama $T_1=\langle\ddd{p_1},\ddd{p_2},\ddd{p}_3\rangle$ in $T_2=\langle\ddd{p}_4,\ddd{p}_2,\ddd{p}_3\rangle$:
  $$\ddd{R}(\ddd{p})=\sum_{i+j+k=3}\ddd{r}_{i,j,k}B_{i,j,k}^n(\ddd{u});
  \quad\ddd{u}=\mbox{Bar}(\ddd{p};T_1),$$
  $$\ddd{S}(\ddd{p})=\sum_{i+j+k=3}\ddd{s}_{i,j,k}B_{i,j,k}^n(\ddd{\tilde{u}});
  \quad\ddd{\tilde{u}}=\mbox{Bar}(\ddd{p};T_2),$$
  kjer je $\ddd{u}=(u,v,w)$ in $\ddd{\widetilde{u}}=(\tilde{u},\tilde{v},\tilde{w})$. 
  Ploskvi se ponovno stikata na robu nad daljico $\overline{\ddd{p}_2\ddd{p}_3}$, oziroma na 
  robu, določenem z $u=\widetilde{u}=0$. 
  
  Zlepek obeh ploskev bo $G^1$-zvezen, če bodo za njune kontrolne točke veljale 
  naslednje enačbe.

  \begin{align}\label{eq24}  
  (1-\lambda_0)\ddd{s}_{1,0,2}+\lambda_0\ddd{r}_{1,0,2}=(1-\mu_0)\ddd{r}_{0,0,3}+
  \mu_0\ddd{r}_{0,1,2},
  \end{align}

  \begin{align}\label{eq25}
  (1-\lambda_1)\ddd{s}_{1,2,0}+\lambda_1\ddd{r}_{1,2,0}=(1-\mu_1)\ddd{r}_{0,2,1}+
  \mu_1\ddd{r}_{0,3,0}.
  \end{align}

  Zgornji enačbi določata razmerje med robnimi kontrolnimi točkami. Iz enačbe 
  \eqref{eq23} pa sledita še pogoja 

  \begin{equation}\label{eq26}
    \begin{split}
  &(1-\lambda_0)\frac{2}{3}\ddd{s}_{1,1,1}+(1-\lambda_1)\frac{1}{3}\ddd{s}_{1,0,2}+
  \lambda_0\frac{2}{3}\ddd{r}_{1,1,1}+\lambda_1\frac{1}{3}\ddd{r}_{1,0,2}=\\
  &(1-\mu_0)\frac{2}{3}\ddd{r}_{0,1,2}+(1-\mu_1)\frac{1}{3}\ddd{r}_{0,0,3}+
  \mu_0\frac{2}{3}\ddd{r}_{0,2,1}+\mu_1\frac{1}{3}\ddd{r}_{0,1,2}
    \end{split}
  \end{equation}
  in 
  \begin{equation}\label{eq27}
    \begin{split}
  &(1-\lambda_0)\frac{1}{3}\ddd{s}_{1,2,0}+(1-\lambda_1)\frac{2}{3}\ddd{s}_{1,1,1}+
  \lambda_0\frac{1}{3}\ddd{r}_{1,2,0}+\lambda_1\frac{2}{3}\ddd{r}_{1,1,1}=\\
  &(1-\mu_0)\frac{1}{3}\ddd{r}_{0,2,1}+(1-\mu_1)\frac{2}{3}\ddd{r}_{0,1,2}+
  \mu_0\frac{1}{3}\ddd{r}_{0,3,0}+\mu_1\frac{2}{3}\ddd{r}_{0,2,1}.
    \end{split}
  \end{equation}

  \todo{primerjava prostih in določenih kontrolnih točk}
  %V primeru, da imamo vnaprej podane robne kontrolne točke ploskev $\ddd{R}$ 
  %in $\ddd{S}$, določiti pa moramo notranji kontrolni točki $\ddd{r}_{1,1,1}$ 
  %in $\ddd{s}_{1,1,1}$, nam enačbi \eqref{eq24} in \eqref{eq25} določata dva izmed 
  %parametrov $\lambda_0$, $\lambda_1$, $\mu_0$ in $\mu_1$, druga dva pa sta prosta. 
  %Po izbiri prostih parametrov, sta v primeru, da je $\lambda_0\neq \lambda_1$, 
  %kontrolni točki $\ddd{s}_{1,1,1}$ in $\ddd{r}_{1,1,1}$ natančno določeni z 
  %enačbama \eqref{eq26} in \eqref{eq27}. 
  %\todo{bolj natančna razlaga z determinanto sistema?}
%
  %Dobljeni rezultat primerjamo z rezultatom primera \ref{primerT2}. V primeru 
  %$C^1$-zveznosti je ena izmed točk $\ddd{r}_{1,1,1}$ in $\ddd{s}_{1,1,1}$ 
  %sicer prosta, druga pa določena z njo, vendar nimamo nobenih prostih parametrov, 
  %ki bi dodatno vplivali na obliko zlepka.

  %Oglejmo si še, kako je s kontrolnimi točkami v primeru, da imamo v celoti 
  %določeno ploskev $\ddd{R}$, določili pa bi radi kontrolne točke ploskve 
  %$\ddd{S}$, da bo zlepek $G^1$-zvezen. Točke $\ddd{s}_{0,j,k}$, $j+k=3$, so 
  %zaradi stikanja ploskev točno določene s kontrolnimi točkami ploskve $\ddd{R}$. 
  %Ker zahtevamo zgolj zveznost stopnje 1, so točke $\ddd{s}_{2,1,0}$, $\ddd{s}_{2,0,1}$ 
  %in $\ddd{s}_{3,0,0}$ proste. 

\todo{dodati še primer, ko je r določena, s pa ne? kako je v tem 
primeru s konsistentnostjo sistema?}
\end{primer}

\todo{primer od prej, ki ni šel s c1 samo da ga zdaj rešimo z g1?}

%-------------------------------------------------------------------------------
%
%\section{4 ploskve}
%
%ko imamo enkrat pogoje med dvema ploskvama, problem med večimi ploskvami okrog 
%skupnega vozlišča, problem zapolnjevanja poligonskih lukenj
%
%problem okrog vozlišča je odvisen od števila ploskev, sodo/liho število ploskev 
%(kaj je če je liho in kaj če sodo?) (dokaz tega?)
%
%4 ploskve, po 2 imata skupno robno krivuljo, dotikajo se v skupnem vozlišču V, 
%dodatni kompatibilnostni pogoji v vozlišču V
%
%ploskve $\ddd{R}^{(1)}$, $\ddd{R}^{(2)}$, $\ddd{R}^{(3)}$, $\ddd{R}^{(4)}$, 
%$\ddd{C}^{(1)}=\ddd{R}^{(1)}(1,v)=\ddd{R}^{(2)}(0,v)$, 
%$\ddd{C}^{(2)}=\ddd{R}^{(2)}(u,1)=\ddd{R}^{(3)}(u,0)$
%$\ddd{C}^{(3)}=\ddd{R}^{(3)}(1,v)=\ddd{R}^{(2)}(0,v)$
%
%iz izreka 1




%\section{Tehnični napotki za pisanje}
%
%\subsection{Sklicevanje in citiranje}
%Za sklice uporabljamo \verb|\ref|, za sklice na enačbe \verb|\eqref|, za citate \verb|\cite|. Pri
%sklicevanju in citiranju sklicano številko povežemo s prejšnjo besedo z nedeljivim presledkom
%$\sim$, kot npr.\ \verb|iz trditve~\ref{trd:obstoj-omega} vidimo|.
%
%\begin{primer}
%  Zaporedje~\eqref{eq:zero-kompleks} iz dokaza trditve~\ref{trd:obstoj-omega} na
%  strani~\pageref{trd:obstoj-omega} lahko najdemo tudi v Spletni enciklopediji zaporedij~\cite{oeis}.
%  Citiramo lahko tudi bolj natančno~\cite[trditev 2.1, str.\ 23]{lebedev2009introduction}.
%\end{primer}
%
%\subsection{Okrajšave}
%Pri uporabi okrajšav \LaTeX{} za piko vstavi predolg presledek, kot npr. tukaj. Zato se za vsako
%piko, ki ni konec stavka doda presledek običajne širine z ukazom \verb*|\ |, kot npr.\ tukaj.
%Primerjaj z okrajšavo zgoraj za razliko.
%
%\subsection{Vstavljanje slik}
%Sliko vstavimo v plavajočem okolju \texttt{figure}. Plavajoča okolja \emph{plavajo} po tekstu, in
%jih lahko postavimo na vrh strani z opcijskim parametrom `\texttt{t}', na lokacijo, kjer je v kodi s
%`\texttt{h}', in če to ne deluje, potem pa lahko rečete \LaTeX u, da ga \emph{res} želite tukaj,
%kjer ste napisali, s `\texttt{h!}'. Lepo je da so vstavljene slike vektorske (recimo \texttt{.pdf}
%ali \texttt{.eps} ali \texttt{.svg}) ali pa \texttt{.png} visoke resolucije (več kot
%\unit[300]{dpi}).  Pod vsako sliko je napis in na vsako sliko se skličemo v besedilu. Primer
%vektorske slike je na sliki~\ref{fig:sample}. Vektorsko sliko prepoznate tako, da močno
%zoomate v sliko, in še vedno ostane gladka. Več informacij je na voljo na
%\url{https://en.wikibooks.org/wiki/LaTeX/Floats,_Figures_and_Captions}. Če so slike bitne, kot na
%primer slika~\ref{fig:image}, poskrbite, da so v dovolj visoki resoluciji.

%\begin{figure}[h]
%  \centering
%  \includegraphics[width=0.6\textwidth]{images/sample.pdf}
%% \caption[caption za v kazalo]{Dolg caption pod sliko}
%  \caption[Primer vektorske slike.]{Primer vektorske slike z oznakami v enaki pisavi, kot jo
%     uporablja \LaTeX{}.  Narejena je s programom Inkscape, \LaTeX{} oznake so importane v
%     Inkscape iz pomožnega PDF.}
%  \label{fig:sample}
%\end{figure}

%\begin{figure}[h]
%  \centering
%  \includegraphics[width=0.8\textwidth]{images/image.png}
%  \caption[Primer bitne slike.]{Primer bitne slike, izvožene iz Matlaba. Poskrbite, da so slike v
%  dovolj visoki resoluciji in da ne vsebujejo prosojnih elementov (to zahteva PDF/A-1b format).}
%  \label{fig:image}
%\end{figure}

%\subsection{Kako narediti stvarno kazalo}
%Dodate ukaze \verb|\index{polje}| na besede, kjer je pojavijo, kot tukaj\index{tukaj}.
%Več o stvarnih kazalih je na voljo na \url{https://en.wikibooks.org/wiki/LaTeX/Indexing}.
%
%\subsection{Navajanje literature}
%Članke citiramo z uporabo \verb|\cite{label}|, \verb|\cite[text]{label}| ali pa več naenkrat s
%\verb|\cite\{label1, label2}|. Tudi tukaj predhodno besedo in citat povežemo z nedeljivim presledkom
%$\sim$. Na primer~\cite{chen2006meshless,liu2001point}, ali pa \cite{kibriya2007empirical}, ali pa
%\cite[str.\ 12]{trobec2015parallel}, \cite[enačba (2.3)]{pereira2016convergence}.
%Vnosi iz \verb|.bib| datoteke, ki niso citirani, se ne prikažejo v seznamu literature, zato jih
%tukaj citiram.~\cite{vene2000categorical}, \cite{gregoric2017stopniceni}, \cite{slak2015induktivni},
%\cite{nsphere}, \cite{kearsley1975linearly}, \cite{STtemplate}, \cite{NunbergerTand}.

% Literatura:
% Primer navajanja na http://www.fmf.uni-lj.si/storage/24240/LiteraturaM.pdf,
% ampak bi moral stil poskrbeti za vse. Reference se uredijo po abecedi.
% Če nobena izbira izmed @book, @atricle,... ni ok, potem se lahko vse napiše v
% @misc pod note={} in deluje tako kot normalen LaTeX.
% Komentar v bib datoteki se naredi samo s parom { }
% Za urejanje literature avtor priporoča program Jabref, ki zna tudi avtomatsko
% okrajšati imena revij. Za pravilno sortiranje vnosov brez avtorja, uporabite
% polje key={ }, kot v primeru.
% V primeru napak ustvarite issue na GitHubu ali pišite na jure.slak@fmf.uni-lj.si.

\cleardoublepage                           % na desni strani
\phantomsection                            % da prav delujejo hiperlinki
%\addcontentsline{toc}{section}{\bibname}   % dodajmo v kazalo
%\bibliographystyle{fmf-sl}                 % uporabljen stil je v datoteki fmf-sl.bst, na voljo tudi angleška verzija
%\bibliography{\literatura}                 % literatura je v datoteki, definirani na začetku

% Za stvarno kazalo
\cleardoublepage                           % na desni strani
\phantomsection                            % da prav delujejo hiperlinki
\addcontentsline{toc}{section}{\indexname} % dodajmo v kazalo
\printindex

\end{document}
