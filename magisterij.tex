% !TeX spellcheck = sl_SI
% vim: set spell spelllang=sl:
% za preverjanje črkovanja, če se uporablja Texstudio ali vim
\documentclass[12pt,a4paper,twoside]{article}
\usepackage[utf8]{inputenc}  % pravilno razpoznavanje unicode znakov

% NASLEDNJE UKAZE USTREZNO POPRAVI
\newcommand{\program}{Matematika} % ime studijskega programa
\newcommand{\imeavtorja}{Katarina Černe} % ime avtorja
\newcommand{\imementorja}{prof.~dr.~} % akademski naziv in ime mentorja, uporabi poln naziv, prof.~dr.~, doc.~dr., ali izr.~prof.~dr.
\newcommand{\imesomentorja}{} % akademski naziv in ime somentorja, če ga imate
\newcommand{\naslovdela}{Naslov vašega dela}
\newcommand{\letnica}{2020} % letnica magistriranja
\newcommand{\opis}{}  % Opis dela v eni povedi. Ne sme vsebovati matematičnih simbolov v $ $.
\newcommand{\kljucnebesede}{\sep } % ključne besede, ločene z \sep, da se PDF metapodatki prav procesirajo
\newcommand{\keywords}{\sep } % ključne besede v angleščini
\newcommand{\organization}{Univerza v Ljubljani, Fakulteta za matematiko in fiziko} % fakulteta
\newcommand{\literatura}{literatura}  % pot do datoteke z literaturo (brez .bib končnice)
\newcommand{\sep}{, }  % separator med ključnimi besedami v besedilu
% KONEC PODATKOV

\usepackage{bibentry}         % za navajanje literature v programu dela s celim imenom
\nobibliography{\literatura}
\newcommand{\plancite}[1]{\item[\cite{#1}] \bibentry{#1}} % citiranje v programu dela

\usepackage{filecontents}  % za pisanje datoteke s PDF metapodatki
\usepackage{silence} \WarningFilter{latex}{Overwriting file}  % odstrani annoying warning o obstoju datoteke
% datoteka s PDF metapodatki, zgenerira se kot magisterij.xmpdata
\begin{filecontents*}{\jobname.xmpdata}
  \Title{\naslovdela}
  \Author{\imeavtorja}
  \Keywords{\kljucnebesede}
  \Subject{\opis}
  \Org{\organization}
\end{filecontents*}

\usepackage[a-1b]{pdfx}  % zgenerira PDF v tem PDF/A-1b formatu, kot zahteva knjižnica
\hypersetup{bookmarksopen, bookmarksdepth=3, colorlinks=true,
  linkcolor=black, anchorcolor=black, citecolor=black, filecolor=black,
  menucolor=black, runcolor=black, urlcolor=black, pdfencoding=auto,
  breaklinks=true, psdextra}

\usepackage[slovene]{babel}  % slovenščina
\usepackage[T1]{fontenc}     % naprednejše kodiranje fonta
\usepackage{amsmath,amssymb,amsfonts,amsthm} % matematični paketi
%\usepackage[dvipsnames,usenames]{color} % barve
\usepackage{graphicx}     % za slike
\usepackage{emptypage}    % prazne strani so neoštevilčene, ampak so štete
\usepackage{units}        % fizikalne enote kot \unit[12]{kg} s polovico nedeljivega presledka, glej primer v kodi
\usepackage{makeidx}      % za stvarno kazalo, lahko zakomentiraš, če ne rabiš
\makeindex                % za stvarno kazalo, lahko zakomentiraš, če ne rabiš
% oblika strani
\usepackage[
  top=3cm,
  bottom=3cm,
  inner=3.5cm,      % margini za dvostransko tiskanje
  outer=2.5cm,
  footskip=40pt     % pozicija številke strani
]{geometry}

% VEČ ZANIMIVIH PAKETOV
% \usepackage{array}      % več možnosti za tabele
% \usepackage[list=true,listformat=simple]{subcaption}  % več kot ena slika na figure, omogoči slika 1a, slika 1b
% \usepackage[all]{xy}    % diagrami
% \usepackage{doi}        % za clickable DOI entrye v bibliografiji
% \usepackage{enumerate}     % več možnosti za sezname

% Za barvanje source kode
% \usepackage{minted}
% \renewcommand\listingscaption{Program}

% Za pisanje psevdokode
% \usepackage{algpseudocode}  % za psevdokodo
% \usepackage{algorithm}
% \floatname{algorithm}{Algoritem}
% \renewcommand{\listalgorithmname}{Kazalo algoritmov}

% DRUGI TVOJI PAKETI:
% tukaj
\usepackage[utf8]{inputenc}
\usepackage{lmodern}
\usepackage{eurosym}
\usepackage{hyperref}
\usepackage{subfigure}
\usepackage{xcolor}
\usepackage{tcolorbox}
\usepackage{enumitem}

\setlength{\overfullrule}{50pt} % označi predlogo vrstico
\pagestyle{plain}               % samo številka strani na dnu, nobene glave / noge

% ukazi za matematična okolja
\theoremstyle{definition} % tekst napisan pokončno
\newtheorem{definicija}{Definicija}[section]
\newtheorem{primer}[definicija]{Primer}
\newtheorem{opomba}[definicija]{Opomba}
\newtheorem{aksiom}{Aksiom}

\theoremstyle{plain} % tekst napisan poševno
\newtheorem{lema}[definicija]{Lema}
\newtheorem{izrek}[definicija]{Izrek}
\newtheorem{trditev}[definicija]{Trditev}
\newtheorem{posledica}[definicija]{Posledica}

\numberwithin{equation}{section}  % števec za enačbe zgleda kot (2.7) in se resetira v vsakem poglavju

% Matematični ukazi
\newcommand{\R}{\mathbb R}
\newcommand{\N}{\mathbb N}
\newcommand{\Z}{\mathbb Z}
\renewcommand{\C}{\mathbb C}
\newcommand{\Q}{\mathbb Q}

\newcommand{\todo}[1]{{\color{red}{#1}}}
\newcommand{\nevem}[1]{{\color{blue}{#1}}}

\newcommand{\ddd}[1]{\textbf{#1}}

% \DeclareMathOperator{\tr}{tr}  % morda potrebuješ operator za sled ali kaj drugega?

% bold matematika znotraj \textbf{ }, tudi v naslovih, kot \omega spodaj
\makeatletter \g@addto@macro\bfseries{\boldmath} \makeatother

% Poimenuj kazalo slik kot ``Kazalo slik'' in ne ``Slike''
\addto\captionsslovene{
  \renewcommand{\listfigurename}{Kazalo slik}%
}

% če želiš, da se poglavja začnejo na lihih straneh zgoraj
% \let\oldsection\section
% \def\section{\cleardoublepage\oldsection}

%%%%%%%%%%%%%%%%%%%%%%%%%%%%%%%%%%%%%%%%%%
%%%%%%           DOCUMENT           %%%%%%
%%%%%%%%%%%%%%%%%%%%%%%%%%%%%%%%%%%%%%%%%%

\begin{document}

\pagenumbering{roman} % začnemo z rimskimi številkami
\thispagestyle{empty} % ampak na prvi strani ni številke

\noindent{\large
UNIVERZA V LJUBLJANI\\[1mm]
FAKULTETA ZA MATEMATIKO IN FIZIKO\\[5mm]
\program\ -- 2.~stopnja}
% ustrezno dopolni za IŠRM
\vfill

\begin{center}
  \large
  \imeavtorja\\[3mm]
  \Large
  \textbf{\MakeUppercase{\naslovdela}}\\[10mm]
  \large
  Magistrsko delo \\[1cm]
  Mentor: \imementorja \\[2mm] % ustrezno popravi spol
%   Somentor: \imesomentorja   % dodaj, če potrebno
\end{center}
\vfill

\noindent{\large Ljubljana, \letnica}

\cleardoublepage

%% IZJAVA O AVTORSTVU
%\pdfbookmark[1]{Izjava o avtorstvu}{izjava} % bookmark v PDF, \pdfbookmark[nivo]{text}{label}
%
%% izjava: po potrebi spremeni v žensko obliko
%\setlength\topsep{0pt}
%\setlength\parskip{0pt}
%\begin{center}
%  \textbf{Univerza v Ljubljani} \\
%  \textbf{Fakulteta za matematiko in fiziko}
%
%  \vfill
%
%  \underline{Izjava o avtorstvu, istovetnosti tiskane in elektronske verzije magistrskega dela in} \\
%  \underline{objavi osebnih podatkov študenta}
%
%  \vfill
%
%  \setlength\topsep{0pt}
%  \setlength\parskip{0pt}
%  \begin{flushleft}
%    Spodaj podpisani študent \imeavtorja{} avtor magistrskega dela (v nadaljevanju: pisnega
%    zaključnega dela študija) z naslovom:
%  \end{flushleft}
%
%  \vfill
%
%  \textbf{\naslovdela}
%
%  \vfill
%
%  IZJAVLJAM
%\end{center}
%
%\begin{enumerate}[1. ]
%  \item \emph{Obkrožite eno od variant a) ali b)}
%  \begin{enumerate}[a)]
%    \item da sem pisno zaključno delo študija izdelal samostojno;
%    \item da je pisno zaključno delo študija rezultat lastnega dela več kandidatov in izpolnjuje
%      pogoje, ki jih Statut UL določa za skupna zaključna dela študija ter je v zahtevanem deležu
%      rezultat mojega samostojnega dela;
%  \end{enumerate}
%  pod mentorstvom IZPOLNI. % dopiši \imementorja v rodilniku
%%   \\ in somentorstvom IZPOLNI. % dopiši \imesomentorja v rodilniku
%  \item da je tiskana oblika pisnega zaključnega dela študija istovetna elektronski obliki
%    pisnega zaključnega dela študija;
%  \item da sem pridobil vsa potrebna dovoljenja za uporabo podatkov in avtorskih del v pisnem
%    zaključnem delu študija in jih v pisnem zaključnem delu študija jasno označil;
%  \item da sem pri pripravi pisnega zaključnega dela študija ravnal v skladu z etičnimi načeli in,
%    kjer je to potrebno, za raziskavo pridobil soglasje etične komisije;
%  \item da soglašam, da se elektronska oblika pisnega zaključnega dela študija uporabi za preverjanje
%    podobnosti vsebine z drugimi deli s programsko  opremo za preverjanje podobnosti
%    vsebine, ki je povezana s študijskim informacijskim sistemom fakultete;
%  \item da na UL neodplačno, neizključno, prostorsko in časovno neomejeno prenašam pravico shranitve
%    avtorskega dela v elektronski obliki, pravico reproduciranja ter pravico dajanja pisnega
%    zaključnega dela študija na voljo javnosti na svetovnem spletu preko Repozitorija UL;
%  \item da dovoljujem objavo svojih osebnih podatkov, ki so navedeni v pisnem zaključnem delu študija
%    in tej izjavi, skupaj z objavo pisnega zaključnega dela študija.
%\end{enumerate}
%
%\vfill
%
%\noindent
%Kraj:  \hfill   Podpis študenta: \phantom{prostor za podpis}
%
%\vfill
%
%\noindent
%Datum:
%
%\cleardoublepage
%% END IZJAVA O AVTORSTVU

% zahvala
\pdfbookmark[1]{Zahvala}{zahvala} %
\section*{Zahvala}
Neobvezno.
Zahvaljujem se \dots
% end zahvala -- izbriši vse med zahvala in end zahvala, če je ne rabiš

\cleardoublepage

\pdfbookmark[1]{\contentsname}{kazalo-vsebine}
\tableofcontents

% list of figures
% \cleardoublepage
% \pdfbookmark[1]{\listfigurename}{kazalo-slik}
% \listoffigures
% end list of figures

\cleardoublepage

\section*{Program dela}
\addcontentsline{toc}{section}{Program dela} % dodajmo v kazalo
Mentor naj napiše program dela skupaj z osnovno literaturo. Na literaturo se
lahko sklicuje kot~\cite{lebedev2009introduction}, \cite{gurtin1982introduction},
\cite{zienkiewicz2000finite}, \cite{STtemplate}.

\section*{Osnovna literatura}
Literatura mora biti tukaj posebej samostojno navedena (po pomembnosti) in ne
le citirana. V tem razdelku literature ne oštevilčimo po svoje, ampak uporabljamo
okolje itemize in ukaz plancite, saj je celotna literatura oštevilčena na koncu.
\begin{itemize}
  \plancite{lebedev2009introduction}
  \plancite{gurtin1982introduction}
  \plancite{zienkiewicz2000finite}
  \plancite{STtemplate}
\end{itemize}

\vspace{2cm}
\hspace*{\fill} Podpis mentorja: \phantom{prostor za podpis}

% \vspace{2cm}
% \hspace*{\fill} Podpis somentorja: \phantom{prostor za podpis}

\cleardoublepage
\pdfbookmark[1]{Povzetek}{abstract}

\begin{center}
\textbf{\naslovdela} \\[3mm]
\textsc{Povzetek} \\[2mm]
\end{center}
Tukaj napišemo povzetek vsebine. Sem sodi razlaga vsebine in ne opis tega, kako je delo
organizirano.

\vfill
\begin{center}
\textbf{English translation of the title} \\[3mm] % prevod slovenskega naslova dela
\textsc{Abstract}\\[2mm]
\end{center}

An abstract of the work is written here. This includes a short description of
the content and not the structure of your work.

\vfill\noindent
\textbf{Math.~Subj.~Class.~(2010):} oznake kot 74B05, 65N99, na voljo so na naslovu
\url{http://www.ams.org/msc/msc2010.html?t=65Mxx} \\[1mm]
\textbf{Ključne besede:} \kljucnebesede \\[1mm]
\textbf{Keywords:} \keywords

\cleardoublepage

\setcounter{page}{1}    % od sedaj naprej začni zopet z 1
\pagenumbering{arabic}  % in z arabskimi številkami

\section{Uvod}

\todo{začneš s tem, da bi radi različne oblike opisali s čim bolj enostavnimi elementi. 
v ta namen uporabljamo enostavne parametrične ploskve (zelo pogosto polinomske npr. 
bezierove), ki jih nato lepimo skupaj v kompleksnejše oblike. želimo, da bi bil stik 
dveh takih ploskev videti gladek, ploskvi morata biti zato prek skupne meje zvezni. 
predstaviš običajno zveznost, poveš, zakaj ni ustrezna}

\todo{lahko najprej poveš, kaj je c zveznost, potem pa navedeš primer, kjer c zveznost 
ne pride v poštev (farin? skopirana knjiga?)}

\todo{geometrijska zveznost je zelo uporabna v praksi, ker lahko modeliramo različne situacije, 
kjer c zveznost odpove (npr. zvezda, suitcase corner, house corner)}

\todo{je invarianta za parametrične transformacije, tj neodvisna od parametrizacije}

\todo{geometrijska zveznost je posplošitev c zveznosti. torej vse nedegenerirane 
(kaj to pomeni?) ploskve, ki so c zvezne, so tudi geometrisko zvezne, niso pa vse 
geometrijsko zvezne ploskve tudi c zvezne (primer v farin, ne razumem)}

\todo{s čim se to delo ukvarja in kaj bo v kakšnem poglavju}

\section{Geometrijska zveznost}\label{geom.zv.}

\todo{nek uvodni tekst?}
Najprej si oglejmo povsem splošno definicijo geometriske zveznosti neke ploskve.

\begin{definicija}
  Ploskev pripada razredu $G^n$ oziroma je \emph{geometrijsko zvezna z redom $n$}, če v okolici vsake njene točke obstaja lokalna regularna parametrizacija razreda $C^n$.
\end{definicija}

\todo{definicija regularne ploskve - lahko poveš definicijo}

\todo{potem lahko poveš, da to v praksi pomeni, da je normala na ploskev v vsaki točki 
neničelna (ali je to res?) ... ampak če hočeš to, moraš verjetno povedati, kako to sledi 
iz definicije (kako?)}

\todo{razloži lokalnost?}

\nevem{Naj bosta $R: \Omega_1 \subseteq \R^2 \rightarrow \R^3$ in 
$S: \Omega_2 \subseteq \R^2 \rightarrow \R^3$ regularni parametrizaciji}

V nadaljevanju se bomo ukvarjali s ploskvami, ki so same po sebi že geometrijsko zvezne, 
zanimalo nas bo le, kakšni pogoji morajo veljati, da je tudi stik dveh takih  
ploskev geometrijsko zvezen, torej da je celotna ploskev, ki jo dobimo, ko zlepimo dve ploskvi, 
geometrijsko zvezna. 

\begin{definicija}\label{def2}
  Naj bosta $R(x,y)$ in $S(u,v)$ regularni $C^n$ parametrizaciji dveh ploskev, ki se stikata v krivulji $C(y)=R(x_0,y)=S(u_0,y)$. 
  Pravimo, da se $R$ in $S$ \emph{stikata z $G^n$-zveznostjo vzdolž krivulje $C$}, če za vsako točko $b_0=C(y_0)$ obstaja lokalno regularna $C^n$ reparametrizacijska funkcija $f(x,y)=(u(x,y),v(x,y))$, da je $f(x_0,y)=(u_0,y)$ za vsak $y \in I_0$ in da velja
  \begin{equation}\label{eq0}
  \frac{\partial^{m+k}}{\partial x^m \partial y^k}R\Bigr|_{\substack{(x_0,y)}}=\frac{\partial^{m+k}}{\partial x^m \partial y^k}(S\circ f)\Bigr|_{\substack{(x_0,y)}} \textrm{ za } m+k=1,\ldots,n,
  \end{equation}
  kjer je $I_0$ neka okolica $y_0$. %bolj natančno, kaj je ta okolica
\end{definicija}

\todo{razloži, kaj je regularna reparametrizacijska funkcija? ali je dovolj, da je 
bila prej razložena regularnost?}

Zaradi stikanja ploskev v krivulji $C$ oziroma, ker vzdolž krivulje $C$ velja $y=v$, 
so delni odvodi parametrizacij 
po spremenljivki $y$ vzdolž krivulje $C$ enaki, zato je dovolj, da pri obravnavi 
geometrijske zveznosti dveh ploskev opazujemo le delne odvode po 
spremenljivki $x$.
\todo{dodati sliko?}
Te delne odvode imenujemo \todo{crossboundary derivatives}.

Oglejmo si, kakšni pogoji morajo veljati v primeru, ko želimo, da je stik med dvema 
ploskvama $G^2$ zvezen.
\begin{primer}
Naj bodo parametrizaciji ploskev $R$ in $S$, krivulja $C$ in reparametrizacijska 
funkcija $f$ kot v definiciji \ref{def2}.
Da bo stik teh dveh ploskev $G^2$ zvezen, mora po definiciji \ref{def2} in ugotovitvi, 
da je dovolj obravnavati le odvode po spremenljivki $y$, veljati
$$\frac{\partial^{k}}{\partial x^k}R\Bigr|_{\substack{(x_0,y)}}=\frac{\partial^{k}}{\partial x^k}(S\circ f)\Bigr|_{\substack{(x_0,y)}} \textrm{ za } k=0,1,2.$$
za vsak $y$ v neki okolici točke\todo{?} $y_0$. 
Da dosežemo zgolj geometrijsko zveznost razreda $G^0$, je dovolj, da med 
ploskvama $R$ in $S$ velja pogoj 
$$R(x_0,y)=(S \circ f)(x_0,y)\textrm{, oziroma }R(x_0,y)=S(u(x_0,y),v(x_0,y)).$$ 
Da imamo na stiku geometrijsko zveznost stopnje $G^1$, mora poleg 
pogoja za $G^0$ veljati še 
$$\frac{\partial R}{\partial x}\Bigr|_{\substack{(x_0,y)}}=\frac{\partial}{\partial x}(S\circ f)\Bigr|_{\substack{(x_0,y)}}$$
Če ustrezno razpišemo parcialni odvod funkcije $S \circ f$, se ta pogoj prepiše v
$$\frac{\partial R}{\partial x}\Bigr|_{\substack{(x_0,y)}}=
\frac{\partial S}{\partial u}\frac{\partial u}{\partial x}\Bigr|_{\substack{(x_0,y)}}+
\frac{\partial S}{\partial v}\frac{\partial v}{\partial x}\Bigr|_{\substack{(x_0,y)}}.$$
Za $G^2$ mora poleg pogojev za $G^0$ in $G^1$ veljati še
\begin{align*} 
\frac{\partial^2 R}{\partial x^2}\bigg|_{\substack{(x_0,y)}} &=
\frac{\partial^2 S}{\partial u^2}\bigg(\frac{\partial u}{\partial x}\bigg)^2\bigg|_{\substack{(x_0,y)}} + 
2\frac{\partial^2 S}{\partial u \partial v}\frac{\partial u}{\partial x}\frac{\partial v}{\partial x}\bigg|_{\substack{(x_0,y)}} + 
\frac{\partial^2 S}{\partial v^2}\bigg(\frac{\partial v}{\partial x}\bigg)^2\bigg|_{\substack{(x_0,y)}} + \\
&+\frac{\partial S}{\partial u}\frac{\partial^2 u}{\partial x^2}\bigg|_{\substack{(x_0,y)}} +
\frac{\partial S}{\partial v}\frac{\partial^2 v}{\partial x^2}\bigg|_{\substack{(x_0,y)}}
\end{align*}
%$G^2$: $R_xx = S_{uu} u_x^2 + 2S_{uv}u_x v_x + S_{vv} v_x^2 + S_u u_{xx} + S_v v_{xx}$ vzdolž $C$
\end{primer}

V splošnem za geometrijsko zveznost stopnje $n$, kjer je $n\in \N_0$ velja naslednje:
\begin{equation}\label{eq1}
\frac{\partial^k R}{\partial x^k}\bigg|_{\substack{C}}=\sum_{i=1}^k \sum_{|m_i|=k}A_{m_i}^k
\sum_{h=0}^i{i \choose h}u_{x^{m_1}}\cdots u_{x^{m_h}}v_{x^{m_{h+1}}}\cdots v_{x^{m_h}}
\frac{\partial^i S}{\partial u^h \partial v^{i-h}}\bigg|_{\substack{C}}
\end{equation}
za vsak $k=0,1,\ldots,n$. Tu z $A_{m_i}^k$ iznačujemo koeficient
$$A_{m_i}^k = \frac{k!}{i!m_1!\cdots m_i!}.$$
Z $u_x^{m_i}$ je označen $m_i$-ti delni odvod funkcije $u$ po $x$, z oznako $|m_i|$ pa 
označimo vsoto $|m_i|=m_1+m_2+\cdots + m_i$, kjer velja $m_j>0$ za vsak $j=1,\ldots, i$.
\todo{dokaz z indukcijo?}

%$$\frac{\partial^j R}{\partial x^j}\Bigr|_{\substack{C}}= \sum_{k=1}^j\sum_{h=0}^k A_{jkh}\frac{\partial^k S}{\partial u^h \partial v^{k-h}}\Bigr|_{\substack{C}}$$
%za vsak $j=0,1,\ldots,n$.
%Tu z $A_{jkh}$ označujemo koeficient
%$$A_{jkh}={k \choose h}\sum_{\substack{m_1+\cdots +m_k =j \\ m_1, \ldots, m_k >0}}
%\frac{j!}{k!m_1!\cdots m_k!} u_{x^{m_1}} \cdots u_{x^{m_h}} v_{x^{m_{h+1}}} \cdots v_{x^{m_k}}\Bigr|_{\substack{C}}.$$

%\todo{kaj je s to lemo?}
%\begin{lema}\todo{?? ali je dokaz tega potreben?}
%  Naj bo $f(u,v)$ funkcija razreda $C^n$ in $u(t)$ in $v(t)$ reparametrizaciji razreda $C^n$. Potem
%  \begin{align*}
%  \frac{d^k f}{dt^k}=& \sum_{i=1}^k\sum_{|\mathbf{m_i}|=k} A_{\mathbf{m_i}}^k \sum_{h=0}^i {i \choose h} u^{(m_1)}\cdots u^{(m_h)}(v) v^{(m_{h+1})}\cdots v^{(m_i)} \\
%  &\cdot\frac{\partial^i f}{\partial u_r^h \partial v_r^{i-h}},
%  \end{align*}
%  kjer $k=1,\ldots,n$ ter\\
%  $\mathbf{m_i} = (m1,m2,\ldots,m_i)$, $|\mathbf{m_i}|=m_1+m_2+\cdots+m_i$ in $A_{\mathbf{m_i}}^k = \frac{k!}{i!m_1!\cdots m_i!}$.
%\end{lema}
%\todo{preoblikuj to lemo, da bo koeficient A kot zgoraj. }
%\todo{dokaz najdeš nekje k piše v viru}

Taka definicija geometrijske zveznosti med dvema ploskvama sama po sebi pri konstrukciji geometrijsko 
zveznih ploskev ni najbolj koristna. V nadaljevanju bo veliko uporabnejši naslednji izrek.

\begin{izrek}\label{izrek1}
  Naj bosta $R(x,y)$ in $S(u,v)$ regularni $C^n$ parametrizaciji dveh ploskev, ki se stikata v krivulji $C(y)=R(x_0,y)=S(u_0,y)$. 
  Ploskvi $R$ in $S$ sta vzdolž skupnega roba $G^n$-zvezni natanko tedaj ko obstajajo 
  $C^n$ funkcije $\alpha_1(y), \ldots, \alpha_n(y)$ in $\beta_1(y) \ldots, \beta_n(y)$, 
  da velja 
  \begin{equation}\label{eq2}
  \frac{\partial^k R}{\partial x^k}\bigg|_{\substack{C}}=\sum_{i=1}^k \sum_{|m_i|=k}A_{m_i}^k
  \sum_{h=0}^i{i \choose h}\alpha_{m_1}\cdots \alpha_{m_h}\beta_{m_{h+1}}\cdots \beta_{m_h}
  \frac{\partial^i S}{\partial u^h \partial v^{i-h}}\bigg|_{\substack{C}},
  \end{equation} 
  kjer je $A_{m_i}^k = \frac{k!}{i!m_1!\cdots m_i!}$.
  Veljati mora tudi, da je $\alpha_1(y)\neq 0$ \todo{in predznak}
\end{izrek}

\begin{opomba}
  Funkcije $\alpha_1(y), \ldots, \alpha_n(y)$ in $\beta_1(y) \ldots, \beta_n(y)$ 
  imenujemo \emph{stične funkcije} \todo{junction/connection functions}
\end{opomba}

\begin{proof}
  \todo{kaj je z lokalnostjo in b0???}
  Najprej prodpostavimo, da obstajajo $C^n$ funkcije $\alpha_1(y), \ldots, \alpha_n(y)$ 
  in $\beta_1(y) \ldots, \beta_n(y)$, za katere velja enakost \ref{eq2} v izreku, in 
  da je $\alpha_1(y)\neq 0$. 
  Dokazati želimo, da od tod 
  sledi $G^n$-zveznost stika ploskev $R(x,y)$ in $S(u,v)$. 

  Definirajmo reparametrizacijsko funkcijo $f(x,y)=(u(x,y),v(x,y))$ na naslednji način: 
  $$u(x,y) =  u_0 + \sum_{i=0}^n \frac{1}{i!}\alpha_i(y)(x-x_0)^i,$$
  $$v(x,y) = y + \sum_{i=0}^n \frac{1}{i!}\beta_i(y)(x-x_0)^i.$$
  
  Ker so po predpostavki funkcije $\alpha_i$ in $\beta_i$, $i=1,\ldots,n$ razreda $C^n$, 
  tudi funkcija $f$ pripada temu razredu. 
  
  Opazimo še, da za $i=1, \ldots, k$ velja
  $$\frac{\partial^i u}{\partial x^i}(x_0,y)=\alpha_i(y),$$
  $$\frac{\partial^i v}{\partial x^i}(x_0,y)=\beta_i(y).$$
  Če dobljeno vstavimo v enačbo \ref{eq2}, dobimo ravno enačbo \ref{eq1}, od koder zaradi 
  ujemanja ploskev $R$ in $S$ v krivulji $C$ sledi tudi enakost \ref{eq0}. Pokazati moramo 
  le še, da je $f$ lokalno regularna.
  
  Vemo, da je reparametrizacijska funkcija $f$ regularna vzdolž $C$, če sta oba njena parcialna odvoda prvega 
  reda linearno neodvisna, torej če velja 
  $$\frac{\partial f}{\partial x}(x_0,y)\times \frac{\partial f}{\partial y}(x_0,y) \neq 0.$$

  Razpišimo oba odvoda reparametrizacijske funkcije $f(x,y)=(u(x,y),v(x,y))$ vzdolž krivulje $C$ 
  in ju skušajmo zapisati s pomočjo stičnih funkcij. Za odvod po spremenljivki $x$ velja:
  $$\frac{\partial f}{\partial x}(x_0,y)=(\frac{\partial u}{\partial x}(x_0,y),\frac{\partial v}{\partial x}(x_0,y))=(\alpha_1(y),\beta_1(y)).$$

  Če razpišemo odvod po spremenljivki $y$, pa dobimo:
  $$\frac{\partial f}{\partial y}(x_0,y)=(\frac{\partial u}{\partial y}(x_0,y),\frac{\partial v}{\partial y}(x_0,y))=(0,1).$$

  Vektorski produkt $\frac{\partial f}{\partial x}(x_0,y)\times \frac{\partial f}{\partial y}(x_0,y)$
  je torej enak 
  $$\frac{\partial f}{\partial x}(x_0,y)\times \frac{\partial f}{\partial y}(x_0,y) = 
  (\alpha_1(y),\beta_1(y))\times (0,1) = \alpha_1(y)$$

  Vektorski produkt obeh parcialnih odvodov prvega reda je torej različen od $0$ 
  natanko tedaj, ko je $\alpha_1(y)\neq 0$, kar pa velja po začetni predpostavki.
  Sledi, da je reparametrizacijska funkcija $f$  regularna. 

  Pokazali smo torej, da obstaja lokalno regularna reparametrizacijska funkcija $f$, ki 
  ustreza pogojem iz definicije \ref{def2}, od koder sledi, da se ploskvi $R$ in $S$ stikata 
  z $G^n$-zveznostjo.

  Dokažimo izrek še v drugo smer.
  Če predpostavimo, da sta ploskvi $R$ in $S$ na stiku $G^n$-zvezni, 
  obstoj funkcij $\alpha_1(y), \ldots, \alpha_n(y)$ in $\beta_1(y) \ldots, \beta_n(y)$
  in enačba \ref{eq2} sledjo neposredno iz definicije \ref{def2} in enačbe \ref{eq1}, 
  če definiramo $\alpha_i(y)=u_{x^i}(y)$ in $\beta_i(y)=v_{x^i}(y)$ za $i=1,\ldots,n$. 
  \todo{ali so take funkcije Cn??}

  Ker je stik ploskev $R$ in $S$ $G^n$-zvezen, do definiciji \ref{def2} obstaja 
  lokalno regularna reparametrizacijska funkcija $f(x,y)=(u(x,y),v(x,y))$, ki ustreza 
  pogojem iz definicije \ref{def2}. Videli smo že, da je funkcija $f$ regularna 
  natanko tedaj, ko je $u_x(x_0,y)=\alpha_1(y)\neq 0$. 
  Torej smo okazali še neničelnost funkcije $\alpha_1$, s čimer je dokaz končan. 
\end{proof}

%Stične funkcije lahko izberemo skoraj povsem poljubno. Upoštevati moramo le dva pogoja, 
%ki omejujeta izbiro funkcije $\alpha_1$. Prvi pogoj sledi iz zahteve po regularnosti 
%reparametrizacijske funkcije $f$.
%Reparametrizacijska funkcija $f$ je regularna vzdolž $C$, če sta oba njena parcialna odvoda prvega 
%reda linearno neodvisna, torej če velja $\frac{\partial f}{\partial x}(x_0,y)\times \frac{\partial f}{\partial y}(x_0,y) \neq 0$.
%Razpišimo oba odvoda reparametrizacijske funkcije $f(x,y)=(u(x,y),v(x,y))$ vzdolž krivulje $C$ 
%in ju skušajmo zapisati s pomočjo stičnih funkcij. Za odvod po spremenljivki $x$ velja:
%$$\frac{\partial f}{\partial x}(x_0,y)=(\frac{\partial u}{\partial x}(x_0,y),\frac{\partial v}{\partial x}(x_0,y))=(\alpha_1(y),\beta_1(y)),$$
%pri čemer smo uporabili opazko \todo{sklic}.
%Če razpišemo odvod po spremenljivki $y$, pa dobimo:
%$$\frac{\partial f}{\partial y}(x_0,y)=(\frac{\partial u}{\partial y}(x_0,y),\frac{\partial v}{\partial y}(x_0,y))=(0,1).$$
%Vektorski produkt $\frac{\partial f}{\partial x}(x_0,y)\times \frac{\partial f}{\partial y}(x_0,y)$
%je torej enak 
%$$\frac{\partial f}{\partial x}(x_0,y)\times \frac{\partial f}{\partial y}(x_0,y) = 
%(\alpha_1(y),\beta_1(y))\times (0,1) = \alpha_1(y)$$
%Sledi, da je reparametrizacija regularna, natanko tedaj \todo{?}, ko za pripradajočo stično \todo{junction?} 
%funkcijo $\alpha_1$ velja $\alpha_1(y)\neq 0$ vzdolž stične krivulje $C$. 
%\todo{ali je potrben podatek "vzdolž krivulje C"?}
%\todo{to moraš nekako motivirati: zakaj si to sploh pogledamo? zato, ker to predstavlja 
%pogoj za $\alpha_1$?}

\todo{Drugo, na kar moramo paziti pri izbiri funkcije $\alpha_1$ pa je njen predznak. 
Pri izbiri napačnega predznaka namreč lahko pride do stika v obliki "špice".} 

%Vpeljava stičnih funkcij nas pripelje do nekoliko drugačne definicije geometrijske 
%zveznosti.
%\begin{definicija}
%  Naj bosta $R(x,y)$ in $S(u,v)$ regularni $C^n$ parametrizaciji dveh ploskev, 
%  ki se stikata v krivulji $C(y)=R(x_0,y)=S(u_0,y)$. 
%  Pravimo, da se $R$ in $S$ stikata z $G^n$-zveznostjo vzdolž krivulje $C$, 
%  če za vsako točko $b_0=C(y_0)$ obstajajo take funkcije razreda $C^n$ ene 
%  spremenljivke $\alpha_1, \ldots, \alpha_n$ in $\beta_1 \ldots, \beta_n$, 
%  da $\alpha_1(y)\neq0$, pri čemer mora imeti $\alpha_1$ ustrezen predznak \todo{??}, 
%  in da velja
%  $$\frac{\partial^j R}{\partial x^j}\Bigr|_{\substack{C}}= \sum_{k=1}^j\sum_{h=0}^k A_{jkh}\frac{\partial^k S}{\partial u^h \partial v^{k-h}}\Bigr|_{\substack{C}}$$
%  za vsak $j=0,1,\ldots,n$.
%  Tu z $A_{jkh}$ označujemo koeficient
%  $$A_{jkh}={k \choose h}\sum_{\substack{m_1+\cdots +m_k =j \\ m_1, \ldots, m_k >0}}
%  \frac{j!}{k!m_1!\cdots m_k!} \alpha_1 \cdots \alpha_{m_h} \beta_{m_{h+1}} \cdots \beta_{m_k}\Bigr|_{\substack{C}}$$
%\end{definicija}
%\todo{popravi A, indekse pri alfah in betah}

\todo{nekaj za zaključek poglavja in napeljavo na novo poglavje}

\section{$G^1$ zveznost}

\todo{nekaj v stilu, da se bomo natančneje ukvarjali z G1 zveznostjo.}
\todo{lahko povem, da je to zveznost tangentnih ravnin oz. zveznost enotskih normal in da si 
bomo ogledali, kako do tega pridemo.}

Imejmo ploskvi $R(x,y)$ in $S(u,v)$, ki se v krivulji $C(y) = R(x_0,y) = S(u_0,y)$ 
stikata z geometrijsko zveznostjo $G^1$.  
Sledi, da je $R_y(x_0,y) = S_y(x_0,y) = S_v(x_0,y)$. Kot smo že videli v poglavju \ref{geom.zv.}, 
nam je zato potrebno opazovati zgolj odvode v smeri $x$.

Ker je stik obeh ploskev v $C$ $G^1$-zvezen, po izreku \ref{izrek1} obstajata 
funkciji $\alpha_1$ in $\beta_1$, kjer je $\alpha_1(y) \neq 0$ za vsak $y$ 
in ima ustrezen predznak, da velja:

\begin{equation}\label{eq3}
R_x(x_0,y)=\alpha_1(y) S_u(u_0,y)+\beta_1(y)S_v(u_0,y).
\end{equation}

Zgornja enačba nam pove, da so parcialni odvodi $R_x(x_0,y)$, $S_u(u_0,y)$ in 
$S_v(u_0,y)$ v vsaki točki $y$ linearno neodvisni. Torej so v vsaki točki $y$ del 
iste tangentne ravnine na krivuljo $C$. Zato torej $G^1$-zveznost imenujemo tudi 
\emph{zveznost tangentnih ravnin}.

%\nevem{v predstavitvi sem šla tako: $R(x_0,y)=S(u_0,y)$,  
%  $R_x(x_0,y)=u_x(x_0,y)S_u(u_0,y)+v_x(x_0,y)S_v(u_0,y)$,  
%  $R_x(x_0,y)=\alpha_1(y) S_u(u_0,y)+\beta_1(y)S_v(u_0,y)$}  
\todo{slika}

Oglejmo si še, od kod pride poimenovanje \emph{zveznost enotskih normal}. 
Znova opazujemo enačbo \ref{eq3}. 
Enačbo sedaj z obeh strani vektorsko pomnožimo z $R_y(x_0,y)$:
$$R_x(x_0,y)\times R_y(x_0,y)=\alpha_1(y) S_u(u_0,y)\times R_y(x_0,y)+\beta_1(y)S_v(u_0,y)\times R_y(x_0,y).$$
Upoštevamo lahko, da je $R_y(x_0,y)=S_v(u_0,y)$. Dobimo:
$$R_x(x_0,y) \times R_y(x_0,y) = \alpha_1(y) S_u(u_0,y) \times S_v(u_0,y).$$
Od tod vidimo, da sta normali na ploskvi $R$ in $S$ na njunu stični 
krivulji $C$ vzporedni. Na skupnem robu imata torej obe ploskvi enaki 
enotski normali:
$$\frac{R_x(x_0,y) \times R_y(x_0,y)}{||R_x(x_0,y) \times R_y(x_0,y)||}=\frac{S_u(u_0,y) \times S_v(u_0,y)}{||S_u(u_0,y) \times S_v(u_0,y)||}.$$

Ker parcialni odvodi $R_x(x_0,y)$, $S_u(u_0,y)$ in $S_v(u_0,y)$ ležijo na isti 
tangentni ravnini, velja tudi:
$$\det(R_x(x_0,y), S_u(u_0,y), S_v(u_0,y))=0.$$

\nevem{Torej obstajajo funkcije (povedati kakšne, iz kje kam?) $\lambda$, 
$\mu$ in $\gamma$, da velja:
$$\lambda(y)R_x(x_0,y)=\mu(y)S_u(u_0,y)+\gamma(y)S_v(u_0,y).$$}

\nevem{Če predpostavimo, da sta ploskvi $R$ in $S$ polinomski, lahko tudi za $\lambda$, $\mu$ in $\gamma$ 
izberemo polinome, kar nam zelo olajša 
konstrukcijo geometrijsko zveznih ploskev.} 

\todo{najbrž lahko poveš, da se bomo v naslednjih poglavjih ukvarjali z izbiro 
teh polinomov}

\todo{mogoče moraš tu napisati, kako se pride do teh polinomov: tiste prve komponente. 
ampak tega ne razumem.}

\section{Bézierjeve ploskve}
\todo{pogledamo si poseben primer polinomskih param ploskev, ki so tudi uporabne v praksi}

$i$-ti Bernsteinov bazni polinom
$$B_i^n(t)={n \choose i}t^i (1-t)^{n-i}\textrm{, } t\in[0,1]$$
Lastnosti:
\begin{itemize}
\item $B_i^n(0) =\delta_{i,0}$
\item $B_i^n(1)=\delta_{i,n}$
%tvorijo bazo prostora polinomov Pn
\end{itemize}

\begin{definicija}
  Naj bodo dane točke $\mathbf{b}_{i,j}\in \R^d$, $i=0,1,\ldots,m$, $j=0,1,\ldots,n$. Bézierjeva ploskev iz tenzorskega produkta je parametrično podana ploskev
  $$\mathbf{b}^{m,n} : [0,1]\times[0,1] \rightarrow \R^d$$
  s predpisom
  $$\mathbf{b}^{m,n}(u,v)=\sum_{i=0}^m \sum_{j=0}^n \mathbf{b_{i,j}} B_i^m(u) B_j^n(v).$$
  Točke $\mathbf{b}_{i,j}$ imenujemo kontrolne točke, poligon, ki jih povezuje, pa kontrolni poligon.
\end{definicija}
Velja: $\mathbf{b^{m,n}}(0,0)=\mathbf{b}_{0,0}$, $\mathbf{b^{m,n}}(1,0)=\mathbf{b}_{m,0}$, $\mathbf{b^{m,n}}(0,1)=\mathbf{b}_{0,n}$, $\mathbf{b^{m,n}}(1,1)=\mathbf{b}_{m,n}$
%interpolacija robnih točk
%vstavi sliko

Odvod Bézierjeve ploskve iz tenzorskega produkta:\\
$$\frac{\partial^{r+s}}{\partial u^r \partial v^s}\mathbf{b}^{m,n}(u,v)=\frac{m!}{(m-r)!}\frac{n!}{(n-s)!}\sum_{i=0}^{m-r} \sum_{j=0}^{n-s} \Delta^{r,s}\mathbf{b}_{i,j}B_i^{m-r}(u)B_j^{n-s}(v),$$
kjer $\Delta^{1,0} \mathbf{b}_{i,j} = \mathbf{b}_{i+1,j}-\mathbf{b}_{i,j}$,\\
$\Delta^{0,1} \mathbf{b}_{i,j} = \mathbf{b}_{i,j+1}-\mathbf{b}_{i,j}$,\\
$\Delta^{r,0} \mathbf{b}_{i,j} = \Delta^{r-1,0} \mathbf{b}_{i+1,j}-\Delta^{r-1,0} \mathbf{b}_{i,j}$,\\
$\Delta^{0,s} \mathbf{b}_{i,j} = \Delta^{0,s-1} \mathbf{b}_{i,j+1}-\Delta^{0,s-1} \mathbf{b}_{i,j}$.
%lahko poveš, da so to v bistvu vektorji
%posebej lahko napišeš (ali dodaš) odvode 1. stopnje
%dodaj pogoje gladkosti za C1 ?
%dodaj trikotne krpe

\section{$G^n$-zveznost med dvema Bézierjevima\\ ploskvama}

\todo{Nekaj v smislu, da sedaj prevedemo že dobljene splošne pogoje na pogoje za 
bezierove ploskve oz. kako ti pogoji izgledajo za te ploskve.}

\todo{ker so bezierov ploskve polinomske, so funkcije alfa in beta iz splošne definicije 
racionalne. dokazali bomo izrek, ki pravi, da lahko za stične funkcije vzamemo tudi 
polinome, kar nam zelo koristi pri konkretnih konstrukcijah in iskanju pogojev 
za kontrolne točke}

Imejmo dve polinomski Bézierjevi ploskvi $\textbf{R}$ in $\textbf{S}$, podani na 
naslednji način: 
$$\textbf{R}(x,y)=\sum_{i=0}^{m_r}\sum_{j=0}^{n_r}\textbf{P}_{ij}B_i^{m_r}B_j^{n_r}$$
$$\textbf{S}(u,v)=\sum_{i=0}^{m_s}\sum_{j=0}^{n_s}\textbf{Q}_{ij}B_i^{m_s}B_j^{n_s},$$
kjer so $\{\textbf{P}_{i,j}, i=1,\ldots, m_r, j=1, \ldots, n_r\}$ in 
$\{\textbf{Q}_{i,j}, i=1, \ldots, m_s, j=1, \ldots, n_s\}$ kontrolne točke ploskev 
$\textbf{R}$ in $\textbf{S}$, $x, y, u$ in $v$ pa parametri z vrednostmi 
na intervalu $[0,1]$.

Ploskvi $\textbf{R}$ in $\textbf{S}$ naj se stikata v skupni robni krivulji
$\textbf{C}(v)=\textbf{R}(0,v)=\textbf{S}(0,v)$
\todo{pojasni, zakaj lahko to predpostavimo. zakaj?? ker lahko parametriziramo? poglej.} 
\todo{pojasni še, kako je s tem, da sta ploskvi prav obrnjeni, da ni špice} 
Robno krivuljo $\textbf{C}$ zapišemo kot Bézierjevo krivuljo na naslednji način:
$$\textbf{C}=\sum_{i=0}^{n_c}\textbf{Z}_i B_i^{n_c},$$
kjer so $\{\textbf{Z}_i, i=1,\ldots n_c\}$ njene kontrolne točke. Stopnja $n_c$ krivulje 
$\textbf{C}$ ni nujno enaka stopnjama $n_r$ ali $n_s$, velja pa, da je $n_c \leq \min(n_r,n_s)$. 
\todo{ali moram povedati, zakaj? ker ne vem}

Naj bosta ploskvi $\textbf{R}$ in $\textbf{S}$ regularni vzdolž krivulje $\textbf{C}$, 
torej naj bodo normale na ploskvi vzdolž krivulje $\textbf{C}$ neničelne:
$$N_r=\bigg(\frac{\partial \textbf{R}}{\partial x}\times \frac{\partial \textbf{R}}{\partial y}\bigg)\bigg|_{\textbf{C}}\neq 0$$
$$N_s=\bigg(\frac{\partial \textbf{S}}{\partial u}\times \frac{\partial \textbf{S}}{\partial v}\bigg)\bigg|_{\textbf{C}}\neq 0$$
\todo{Nr(y)?}

O pogojih za geometrijsko zveznost teh dveh ploskev govori naslednji izrek: 

\begin{izrek}\label{izrek2}
  Naj bosta $\textbf{R}$ in $\textbf{S}$ zgoraj definirani \todo{to ok?} Bézierovi 
  ploskvi, ki se stikata v robni krivulji $\textbf{C}$ (kot zgoraj). Stik ploskev je 
  $G^n$-zvezen natanko tedaj, ko obstajajo polinomi $D(y)$, $E_i(y)$ in $F_i(y)$, da 
  velja:
  \begin{equation}\label{eq4}
    \begin{split}
      D^{2k-1}(y)\frac{\partial^k\textbf{S}}{\partial u^k}\bigg|_{\textbf{C}} = &
      \sum_{i=0}^k \sum_{|m_i|=k} A_{m_i}^k \sum_{h=0}^i {i\choose h} D^{i-1}(y)E_{m_1}(v)
      \cdots E_{m_h}(y) \\
      &\cdot F_{m_{h+1}}(y)\cdots F_{m_i}(y) \frac{\partial^i \textbf{R}}{\partial x^h \partial y^{i-h}},
    \end{split}  
  \end{equation}
  kjer je $i=1, \ldots n$ in $k=1, \ldots n$. Z $A_{m_i}^k$ zopet označujemo 
  $A_{\mathbf{m_i}}^k = \frac{k!}{i!m_1!\cdots m_i!}$ in $|\mathbf{m_i}|=m_1+m_2+\cdots+m_i$.
  Velja še $D(y)E_1(y)\neq 0$ za $y\in [0,1]$, za stopnje polinomov pa velja:\\
  $st(D)\leq n_r + n_c -1,$\\
  $st(E_i)\leq (2i-2)n_r + in_s +in_c -2i +1$ in\\
  $st(F_i)\leq (2i-1)n_r +in_s + (i-1)n_c -2i +1.$
\end{izrek}

\begin{proof}
  Najprej predpostavljajmo, da obstajajo polinomi $D$, $E_i$ in $F_i$, $i=1, \ldots n$, 
  ki ustrezajo enačbi \ref{eq4} in ostalim pogojem v izreku. Pokazati hočemo, da 
  od tod sledi geometrijska zveznost ploskev $\textbf{R}$ in $\textbf{S}$. V ta namen 
  bomo uporabili izrek \ref{izrek1}. 
  
  Preoblikujmo enačbo 
  \begin{align*}
    D^{2k-1}(y)\frac{\partial^k\textbf{S}}{\partial u^k}\bigg|_{\textbf{C}} = &
    \sum_{i=0}^k \sum_{|m_i|=k} A_{m_i}^k \sum_{h=0}^i {i\choose h} D^{i-1}(y)E_{m_1}(v)
    \cdots E_{m_h}(y) \\
    &\cdot F_{m_{h+1}}(y)\cdots F_{m_i}(y) \frac{\partial^i \textbf{R}}{\partial x^h \partial y^{i-h}}.
  \end{align*}
  Če celotno enačbo delimo z $D^{2k-1}$ (predpostavka, da $D(y)E_1(y)\neq 0$ na $[0,1]$, zagotavlja neničelnost polinoma $D$ na $[0,1]$), 
  dobimo 
  \begin{equation}\label{eq5}
    \begin{split}
    \frac{\partial^k\textbf{S}}{\partial u^k}\bigg|_{\textbf{C}} = &
    \sum_{i=0}^k \sum_{|m_i|=k} A_{m_i}^k \sum_{h=0}^i {i\choose h} D^{i-2k}(y)E_{m_1}(v)
    \cdots E_{m_h}(y) \\
    &\cdot F_{m_{h+1}}(y)\cdots F_{m_i}(y) \frac{\partial^i \textbf{R}}{\partial x^h \partial y^{i-h}}.
    \end{split}
  \end{equation}
  Funkcijo $D^{2k-i}$ lahko zapišemo kot produkt \\
  $D^{2k-i}(y)=D^{2m_1-1}(y)D^{2m_2-1}(y)\cdots D^{2m_h-1}(y) D^{2m_{h+1}}(y)\cdots D^{2m_i-1}(y)$,
  saj je $|m_i|=k$.

  Dobljeno vstavimo v enačbo \ref{eq5}:
  \begin{equation}\label{eq6}
    \begin{split}
    \frac{\partial^k\textbf{S}}{\partial u^k}\bigg|_{\textbf{C}} = &
    \sum_{i=0}^k \sum_{|m_i|=k} A_{m_i}^k \sum_{h=0}^i {i\choose h} \frac{E_{m_1}(v)}{D^{2m_1-1}(y)}
    \cdots \frac{E_{m_h}(y)}{D^{2m_h-1}(y)} \\
    &\cdot \frac{F_{m_{h+1}}(y)}{D^{2m_{h+1}-1}(y)}\cdots \frac{F_{m_i}(y)}{D^{2m_i-1}(y)} \frac{\partial^i \textbf{R}}{\partial x^h \partial y^{i-h}}.
    \end{split}
  \end{equation}

  Definirajmo
  $$\alpha_i(y)=\frac{E_i(y)}{D^{2i-1}(y)}\textrm{ in }\beta_i(y) = \frac{F_i(y)}{D^{2i-1}(y)},$$
  kjer je $i=1, \ldots, n$. Potem enačba \ref{eq6} dobi enako obliko kot  
  enačba \ref{eq2} v izreku \ref{izrek1}. Iz izreka \ref{izrek1} torej sledi, da se 
  ploskvi $\textbf{R}$ in $\textbf{S}$ stikata z geometrijsko zveznostjo $G^n$. S tem smo 
  dokazali \todo{eno smer ekvivalence ? kako se to lepo reče}

  Sedaj dokažimo še \todo{drugo smer ekvivalence v izreku}. Predpostavimo, da se ploskvi 
  $\textbf{R}$ in $\textbf{S}$, definirani kot zgoraj \todo{ok?}, stikata v robni krivulji 
  $\textbf{C}$ z geometrijsko zveznostjo $G^n$. Pokazati hočemo, da od tod sledi obstoj 
  polinomov $D$, $E_i$ in $F_i$ z lastnostmi kot v izreku.
  Dokaza se lotimo z indukcijo po $k$.

  Naj bo najprej $k=1$. Ker je slik ploskev $\textbf{R}$ in $\textbf{S}$ $G^n$-zvezen, 
  torej vsaj $G^1$-zvezen,
  po izreku \ref{izrek1} obstajata $C^n$ funkciji $\alpha_1(y)$ in $\beta_1(y)$, ki zadoščata 
  enačbi
  \begin{equation}\label{eq7}
  \frac{\partial \textbf{S}}{\partial u}\bigg|_{\textbf{C}}(y) = 
  \alpha_1(y)\frac{\partial \textbf{R}}{\partial x}\bigg|_{\textbf{C}}(y) + 
  \beta_1(y) \frac{\partial \textbf{R}}{\partial y}\bigg|_{\textbf{C}}(y).
  \end{equation}

  Dobljeno enačbo z desne vektorsko množimo z $\frac{\partial \textbf{R}}{\partial y}$. 
  Dobimo:
  \begin{equation}\label{eq8}
  \frac{\partial \textbf{S}}{\partial u} \times \frac{\partial \textbf{R}}{\partial y}\bigg|_{\textbf{C}}(y) = 
  \frac{\partial \textbf{S}}{\partial u} \times \frac{\partial \textbf{S}}{\partial v}\bigg|_{\textbf{C}}(y) = 
  \alpha_1(y)\frac{\partial \textbf{R}}{\partial x}\times \frac{\partial \textbf{R}}{\partial y}\bigg|_{\textbf{C}}(y).
  \end{equation}
  V poglavju \ref{geom.zv.} smo namreč že videli, da je 
  $\frac{\partial \textbf{R}}{\partial y}\big|_{\textbf{C}}=\frac{\partial \textbf{S}}{\partial v}\big|_{\textbf{C}}=\textbf{C'}$.

  Enačbo \ref{eq7} sedaj z desne vektorsko množimo še z $\frac{\partial \ddd{R}}{\partial x}$ 
  in dobimo:
  \begin{equation}\label{eq9}
  \frac{\partial \ddd{S}}{\partial u}\bigg|_{\ddd{C}}(y)\times 
  \frac{\partial \ddd{R}}{\partial x}\bigg|_{\ddd{C}}(y) = 
  \beta_1(y) \frac{\partial \ddd{R}}{\partial y}\bigg|_{\ddd{C}}
  \times \frac{\partial \ddd{R}}{\partial x}\bigg|_{\ddd{C}}
  \end{equation}

  Z $\ddd{W}(y)$ označimo vektorsko funkcijo
  $\ddd{W}(y)=\frac{\partial \ddd{R}}{\partial x}|_{\ddd{C}}\times \frac{\partial \ddd{S}}{\partial u}|_{\ddd{C}}$, 
  z $\ddd{N}_r$ in $\ddd{N}_s$ pa normalo na ploskev $\ddd{R}$ oziroma $\ddd{S}$ v neki točki 
  na mejni \todo{?} krivulji $\ddd{C}$.
  %, torej 
  %$\ddd{N}_r = \big(\frac{\partial \ddd{R}}{\partial x} \times \frac{\partial \ddd{R}}{\partial y}\big)\big|_{\ddd{C}}$ 
  %in $\ddd{N}_s = \big(\frac{\partial \ddd{S}}{\partial u} \times \frac{\partial \ddd{S}}{\partial v}\big)\big|_{\ddd{C}}$.

  Prej dobljeni enačbi \ref{eq8} in \ref{eq9} torej zapišemo na naslednji način:
  \begin{equation}\label{eq10} 
  \ddd{N}_s(y)=\alpha_1(y)\ddd{N}_r(y)
  \end{equation}
  in
  \begin{equation}\label{eq11} 
  \ddd{W}(y)=\beta_1(y)\ddd{N}_r(y).
  \end{equation}

  Stopnja $\frac{\partial\ddd{R}}{\partial y}|_{\ddd{C}}$ je največ $n_c-1$, 
  saj je $\frac{\partial\ddd{R}}{\partial y}|_{\ddd{C}}=\ddd{C}'$. Enako velja za 
  $\frac{\partial \ddd{S}}{\partial v}|_{\ddd{C}}$. Stopnja $\frac{\partial\ddd{R}}{\partial x}|_{\ddd{C}}$ 
  je manjša ali enaka $n_r$, stopnja $\frac{\partial\ddd{S}}{\partial u}|_{\ddd{C}}$ 
  pa manjša ali enaka $n_s$. \todo{razloži?}
  Od tod in iz definicij funkcij $\ddd{N}_r$, $\ddd{N}_s$ in $\ddd{W}$ sledi
  $st(\ddd{N}_r)\leq n_r + n_s -1$, $st(\ddd{N}_s)\leq n_s + n_c -1$ in 
  $st(\ddd{W})\leq n_r + n_s$.

  Videli smo že, da sta zaradi predpostavke o regularnosti ploskev 
  $\ddd{R}$ in $\ddd{S}$ funkciji $\ddd{N}_r(y)$ in $\ddd{N}_s(y)$ za vsak $y\in[0,1]$
  različni od $0$.
  Ker je $\ddd{N}_r(y)$ neničelna, mora biti vsaj ena izmed njenih koordinatnih funkcij  
  neničeln polinom. Brez škode za splošnost predpostavimo, da je neničelna $x$-koordinata, 
  torej polinom $N_{rx}(y)$. 
  %\nevem{za vsak v na [0,1] mora biti vsaj ena od treh 
  %koordinatnih funkcij različna od 0. bšs je to N_rx}
  Če enačbi iz \ref{eq10} in \ref{eq11} razpišemo po koordinatah, za $x$-koordinato dobimo 
  $$N_{sx}(y)=\alpha_1(y)N_{rx}(y)$$
  in $$W_x(y)=\beta_1(y)N_{rx}(y),$$
  kjer je $N_{sx}$ $x$-koordinata funkcije $\ddd{N}_s$, $W_x$ pa $x$-koordinata 
  funkcije $\ddd{W}$.

  Iz zgornjih enačb lahko vidimo, da so vse realne ničle polinoma $N_{rx}(y)$ na intervalu $[0,1]$ 
  tudi ničle polinomov $N_{sx}(y)$ in $W_x(y)$, torej da polinom $U(y)$, ki je zgrajen 
  kot produkt vseh linearnih faktorjev v polinomskem razcepu polinoma $N_{rx}(y)$, deli 
  polinoma $N_{sx}(y)$ in $W_x(y)$. Da to res drži, lahko vidimo na naslednji način.
  Zapišimo $N_{rx}(y) = U(y)D(y)$, kjer je $U(y)$ produkt vseh linearnih faktorjev, 
  $D(y)$ pa produkt vseh nelinearnih faktorjev v polinomskem razcepu polinoma $N_{rx}(y)$. 
  Predpostavimo, da $U(y)$ ne deli polinoma $N_{sx}(y)$. Ker je $N_{sx}(y)=\alpha_1(y)U(y)D(y)$, 
  je to mogoče le, če je $\alpha_1(y)$ racionalna funkcija, katere imenovalec deli polinom $U(y)$. 
  Funkcija $\alpha_1(y)$ ima torej na intervalu $[0,1]$ pol. Ker velja $\ddd{N}_s(y)=\alpha_1(y)\ddd{N}_r(y)$
  in so vse koordinatne funkcije funkcij $\ddd{N}_s(y)$ in $\ddd{N}_r(y)$ polinomi, 
  mora veljati, da imenovalec funkcije $\alpha_1(y)$ deli $N_{rx}(y)$, $N_{ry}(y)$ in 
  $N_{rz}(y)$. Funkcija $\alpha_1(y)$ ima pol, označimo ga z $y_0$. Sledi, da je $y_0$ 
  ničla polinomov $N_{rx}(y)$, $N_{ry}(y)$ in $N_{rz}(y)$, in zato je $\ddd{N}_r(y_0)=0$, 
  kar pa je v nasprotju s predpostavko o regularnosti ploskve $\ddd{R}$. Torej mora polinom 
  $U(y)$ deliti polinom $N_{sx}(y)$. Z enakimi sklepi trditev pokažemo še za polinom $W_x(y)$.


  \todo{vprašanje: ali U vsebuje vse linearne faktorje ali samo tiste, ki imajo zvezo 
  z ničlami na [0,1]???}


  Polinom $N_{rx}$ sedaj znova zapišimo kot produkt $N_{rx}(y) = U(y)D(y)$, kjer sta 
  polinoma $U(y)$ in $D(y)$ definirana kot zgoraj. Torej velja 
  $$N_{sx}(y)=U(y)\alpha_1(y)D(y)$$ in $$W_x(y)=U(y)\beta_1(y)D(y).$$
  Naj bo $E_1(y)=\alpha_1(y)D(y)$ in $F_1(y)=\beta_1(y)D(y)$. Pokazati moramo, da sta 
  dobljeni funkciji $E_1$ in $F_1$ polinoma. Ker sta funkciji $N_{sx}(y)$ in $W_x(y)$ 
  polinoma, morata imenovalca funkcij $\alpha_1$ in $\beta_1$ deliti ali polinom $U$ ali 
  polinom $D$. Videli smo že, da $\alpha_1$ in $\beta_1$ nimata polov na intervalu 
  $[0,1]$, torej njuna imenovalca ne delita polinoma $U$. Sledi, da morata njuna imenovalca 
  deliti polinom $D$, s čimer smo pokazali, da sta $E_1$ in $F_1$ res polinoma.

  Videti želimo še, da je $D(y)E_1(y)\neq 0$ na intervalu $[0,1]$. Polinom $D(y)$ po definiciji 
  vsebuje vse nelinearne faktorje v polinomskem razcepu polinoma $N_{rx}(y)$, torej 
  na intervalu $[0,1]$ nima ničel. Polinom $E_1(y)$ je enak $E_1(y)=\alpha_1(y)D(y)$. 
  Ker je stik ploskev $\ddd{R}$ in $\ddd{S}$ $G^n$-zvezen, funkcija $\alpha_1(y)$ 
  po izreku \ref{izrek1} na 
  intervalu $[0,1]$ ni enaka nič, zato tudi $E_1(y)$ na tem intervalu 
  nima ničel. 

  Oglejmo si še stopnje polinomov $D(y)$, $E_1(y)$ in $F_1(y)$. 
  Očitno velja:
  $$st(D(y))\leq st(N_{rx}(y))\leq st(\ddd{N}_r(v))\leq n_r+n_c-1$$
  $$st(E_1(y))\leq st(N_{sx}(y))\leq st(\ddd{N}_s(v))\leq n_s+n_c-1$$
  $$st(F_1(y))\leq st(W_x(y))\leq st(\ddd{W}(v))\leq n_r+n_s,$$
  s čimer dokažemo izrek za $k=1$.

  Lotimo se še dokaza za $k>1$. Prepodstavimo, da izrek velja za vse $k\leq m$, kjer je $m\in \N$, $m<n$. 
  Torej obstajajo polinomi $D(y)$, $E_1(y), \ldots, E_m(y)$, $F_1(y), \ldots, F_m(y)$ 
  z ustreznimi stopnjami, da velja enačba \ref{eq4} za $k=1,2,\ldots,m$.
  
  Izhajamo iz predpostavke, da je stik ploskev $\ddd{R}$ in $\ddd{S}$ $G^n$-zvezen. 
  Iz izreka \ref{izrek1} sledi, da obstajajo funkcije 
  $\alpha_1, \ldots, \alpha_{m+1}$ in $\beta_1 \ldots, \beta_{m+1}$, da velja
  \begin{align*}
    \frac{\partial^{m+1} \ddd{S}}{\partial u^{m+1}}\Bigr|_{\substack{\ddd{C}}}&= 
    \sum_{i=1}^{m+1}\sum_{|\ddd{m}_i|=m+1}A_{m_i}^{m+1}
    \sum_{h=0}^i {i \choose h}\alpha_{m_1}\cdots \alpha_{m_h}\beta_{m_{h+1}}\beta_{m_i} 
    \frac{\partial^i \ddd{R}}{\partial x^h \partial y^{i-h}}\Bigr|_{\substack{C}}\\
    & = \alpha_{m+1} \frac{\partial \ddd{R}}{\partial u}|_{\ddd{C}} + 
    \beta_{m+1} \frac{\partial \ddd{R}}{\partial v}|_{\ddd{C}} + \\
    & + \sum_{i=2}^{m+1} \sum_{|\ddd{m}_i|}A_{m_i}^{m+1} \sum_{h=0}^i {i \choose h}
    \alpha_{m_1} \cdots \alpha_{m_h} \beta_{m_{h+1}} \cdots \beta_{m_i}
    \frac{\partial^i \ddd{R}}{\partial x^h \partial y^{i-h}}\Bigr|_{\substack{C}}
  \end{align*}

  Po indukcijski predpostavki je $\alpha_i(y) = \frac{E_i(y)}{D^{2i-1}(y)}$ in 
  $\beta_i(y)=\frac{F_i(y)}{D^{2i-1}(y)}$ za $i=1, \ldots, m$. \todo{ta i.p. pride od nikoder?}
  Uporabimo to v zgornji 
  enačbi in dobimo:
  \begin{equation}\label{eq12}
    \begin{split}
    \frac{\partial^{m+1} \ddd{S}}{\partial u^{m+1}}\Bigr|_{\substack{\ddd{C}}}&= 
    \alpha_{m+1} \frac{\partial \ddd{R}}{\partial u}|_{\ddd{C}} + 
    \beta_{m+1} \frac{\partial \ddd{R}}{\partial v}|_{\ddd{C}} + \\
    &+ \sum_{i=2}^{m+1} \sum_{|\ddd{m}_i|}A_{m_i}^{m+1} \sum_{h=0}^i {i \choose h}
    \frac{E_{m_1}(y)}{D^{2m_1-1}(y)} \cdots \frac{E_{m_h}(y)}{D^{2m_h-1}(y)}\\ 
    &\cdot \frac{F_{m_{h+1}}(y)}{D^{2m_{h+1}-1}(y)} \cdots \frac{F_{m_i}(y)}{D^{2m_i-1}(y)}
    \frac{\partial^i \ddd{R}}{\partial x^h \partial y^{i-h}}\Bigr|_{\substack{C}} = \\
    &= \alpha_{m+1} \frac{\partial \ddd{R}}{\partial u}|_{\ddd{C}} + 
    \beta_{m+1} \frac{\partial \ddd{R}}{\partial v}|_{\ddd{C}} + \\
    &+ \sum_{i=2}^{m+1} \sum_{|\ddd{m}_i|}A_{m_i}^{m+1} \sum_{h=0}^i {i \choose h}
    D^{i-2}(y)D^{-2m}E_{m_1}(y) \cdots E_{m_h}(y) \\
    & \cdot F_{m_{h+1}}(y) \cdots F_{m_i}(y)
    \frac{\partial^i \ddd{R}}{\partial x^h \partial y^{i-h}}\Bigr|_{\substack{C}},
    \end{split}
  \end{equation}
  saj je $|\ddd{m}_i|=m_1+m_2+\cdots + m_i = m+1$ in zato je \\
  $D^{-2m_1+1}(y) D^{-2m_2+1}(y) \cdots D^{-2m_i+1}(y)=D^{-2(m+1)}(y)D^i(y)$.

  Sedaj definirajmo \todo{krivuljo? vektorsko polinomsko funkcijo?} $\ddd{S}_{m+1}$:
  \begin{align*}\ddd{S}_{m+1}(y)=&D^{2m}(y)\frac{\partial^{m+1} \ddd{S}}{\partial u^{m+1}}|_{\ddd{C}} - 
  \sum_{i=2}^{m+1} \sum_{|\ddd{m}_1|=m+1} A_{m_i}^{m+1} \sum_{h=0}^i {i \choose h}
  D^{i-1}(y)E_{m_1}(y)\cdots E_{m_h}(y) \\ &\cdot F_{m_{h+1}}(y)\ldots F_{m_i}(y)
  \frac{\partial^i \ddd{R}}{\partial x^h \partial y^{i-h}}|_{\ddd{C}}
  \end{align*}
  Če enačbo \ref{eq12} pomnožimo z $D^{2m}(y)$ in jo nekoliko preoblikujemo, dobimo:
  \begin{equation}\label{eq13} 
  \ddd{S}_{m+1}(y)=D^{2m}\alpha_{m+1}(y)\frac{\partial \ddd{R}}{\partial x}|_{\ddd{C}} + 
  D^{2m}(y)\beta_{m+1}(y)\frac{\partial \ddd{R}}{\partial y}|_{\ddd{C}}.
  \end{equation}

  Na dobljeni enačbi sedaj uporabimo podoben postopek, kot smo ga uporabili pri dokazu 
  za $k=1$. Enačbo \ref{eq13} z leve vektorsko množimo z $\frac{\partial \ddd{R}}{\partial x}|_{\ddd{C}}$ 
  in dobimo 
  $$\frac{\partial \ddd{R}}{\partial x}|_{\ddd{C}} \times \ddd{S}_{m+1} = 
  D^{2m}\beta_{m+1} \frac{\partial \ddd{R}}{\partial x}|_{\ddd{C}} \times \frac{\partial \ddd{R}}{\partial y}|_{\ddd{C}}.$$
  Če pa enačbo \ref{eq13} z desne pomnožimo z $\frac{\partial \ddd{R}}{\partial y}|_{\ddd{C}}$, 
  dobimo 
  $$\ddd{S}_{m+1}\times\frac{\partial \ddd{R}}{\partial x}|_{\ddd{C}} = 
  D^{2m}\alpha_{m+1} \frac{\partial \ddd{R}}{\partial x}|_{\ddd{C}} \times \frac{\partial \ddd{R}}{\partial y}|_{\ddd{C}}.$$
  Označimo $\ddd{W}_1 = \ddd{S}_{m+1}\times\frac{\partial \ddd{R}}{\partial y}|_{\ddd{C}}$ in 
  $\ddd{W}_2 = \frac{\partial \ddd{R}}{\partial x}|_{\ddd{C}}\times \ddd{S}_{m+1}$ ter kakor prej 
  $N_r = \frac{\partial \ddd{R}}{\partial x}|_{\ddd{C}} \times \frac{\partial \ddd{R}}{\partial y}|_{\ddd{C}}$. 
  Kot v primeru za $k=1$ spet lahko predpostavimo, da je polinom $N_{rx}(y)$ neničeln in 
  ga zapišemo kot $N_{rx}(y)=U(y)D(y)$. Velja 
  $W_{1x} = D^{2m+1}(y)U(y)\alpha_{m+1}(y)$ in $W_{2x} = D^{2m+1}(y)U(y)\beta_{m+1}(y)$ 
  in enaki argumenti kot v primeru za $k=1$ nas pripeljejo do razulatata, da sta 
  $E_{m+1}(y) = D^{2m+1}(y)\alpha_{m+1}(y)$ in \\$F_{m+1}(y) = D^{2m+1}(y)\beta_{m+1}(y)$ 
  res polinoma. 

  Pokazati moramo še, da je $st(E_{m+1})\leq 2mn_r + (m+1)n_s+(m+1)n_c-2m-1$ in 
  $st(F_{m+1})\leq (2m+1)n_r + (m+1)n_s+mn_c-2m$. Tega se lotimo tako, da si najprej 
  ogledamo stopnjo $\ddd{S}_{m+1}$. Spomnimo se:
  \begin{align*}\ddd{S}_{m+1}=&D^{2m}(v)\frac{\partial^{m+1} \ddd{S}}{\partial u^{m+1}}|_{\ddd{C}} - 
    \sum_{i=2}^{m+1} \sum_{|\ddd{m}_1|=m+1} A_{m_i}^{m+1} \sum_{h=0}^i {i \choose h}
    D^{i-1}(y)E_{m_1}(y)\cdots E_{m_h}(y) \\ &\cdot F_{m_{h+1}}(y)\ldots F_{m_i}(y)
    \frac{\partial^i \ddd{R}}{\partial x^h \partial y^{i-h}}|_{\ddd{C}}
  \end{align*}
  Očitno je
  \begin{align*} 
    st(D^{2m}(y)\frac{\partial^{m+1} \ddd{S}}{\partial u^{m+1}})&\leq 
    2m(n_r+n_c-1)+n_s \\
    &\leq 2mn_r + (m+1)n_s + mn_c -2m, 
  \end{align*}
  kjer v prvi neenakosti uporabimo dejstvo, da je $st(D(y))\leq n_r + n_c-1$ in 
  $st(\frac{\partial^{m+1} \ddd{S}}{\partial u^{m+1}})\leq n_s$, v drugi neenakosti pa, 
  da je $n_c\leq n_s$.

  Oglejmo si še, kakšna je \\
  $st(D^{i-1}(y)E_{m_1}(y)\cdots E_{m_h}(y)F_{m_{h+1}}(y)\ldots F_{m_i}(y)
  \frac{\partial^i \ddd{R}}{\partial x^h \partial y^{i-h}}|_{\ddd{C}})$.
  Najprej si jo oglejmo za $h=0$:
  \begin{align*}
    &st(D^{i-2}(y)F_{m_1}(y)\cdots F_{m_i}(y)\frac{\partial^i \ddd{R}}{\partial y^i}|_{\ddd{C}})\leq\\
    &\leq(i-2) st(D(y)) + \sum_{j=1}^i st(F_{m_j}(y)) + st(\frac{\partial^i \ddd{R}}{\partial y^i}|_{\ddd{C}}) 
  \end{align*}
  Vemo, da je $st(D(y))\leq n_r+n_c-1$ in $st(\frac{\partial^i \ddd{R}}{\partial y^i}|_{\ddd{C}})\leq n_c-i$, 
  po indukcijski predpostavki pa velja še 
  $st(E_i(y)\leq (2i-2)n_r+in_s+in_c-2i+1)$ in $st(F_i(y)\leq (2i-1)n_r+in_s+(i-1)n_c-2i+2)$, 
  kjer je $i=1, \ldots, m$. 
  Torej je 
  \begin{align*}
    &st(D^{i-2}(y)F_{m_1}(y)\cdots F_{m_i}(y)\frac{\partial^i \ddd{R}}{\partial y^i}|_{\ddd{C}})\leq\\
    &(i-2)(n_r+n_c-1)+\\
    &+\sum_{j=1}^i((2m_j-1)n_r+m_jn_s+(m_j-1)n_c-2m_j+2) + (n_c -i)= \\
    &= (i-2)(n_r + n_c -1) + 2(m+1)n_r - in_r + (m+1)n_s + (m+1)n_c -in_c -\\
    &- 2(m+1) +2i + n_c -i = \\
    &= 2mn_r + (m+1)n_s + mn_c -2m -in_r \leq \\
    &\leq 2mn_r +(m+1)n_s + mn_c -2m.
  \end{align*}
  Tu smo uporabili, da je $\sum_{j=1}^{i}m_j = m+1$.

  Sedaj obravnavajmo še primer, ko je $h>1$.
  \begin{align*}
    &st(D^{i-2}(y)E_{m_1}(y)\cdots E_{m_h}(y)F_{m_{h+1}}(y)\cdots F_{m_i}(y)
    \frac{\partial^i\ddd{R}}{\partial x^h \partial y^{i-h}}|_{\ddd{C}})\leq \\
    &\leq (i-2)st(D(y)) + \sum_{j=1}^h st(E_{m_j}) + \sum_{j=h+1}^i st(F_{m_j}) + \\
    &+ st(\frac{\partial^i \ddd{R}}{\partial x^h \partial y^{i-h}}|_{\ddd{C}})
  \end{align*}
  Zopet uporabimo indukcijsko predpostavko za stopnje polinomov $E_i(y)$ in $F_i(y)$, 
  kjer je $i=1,\ldots, m$, ter dejstvo, da je 
  $st(\frac{\partial^i \ddd{R}}{\partial x^h \partial y^{i-h}}|_{\ddd{C}})=n_r-i+h$, in 
  dobimo
  \begin{align*}
    &st(D^{i-2}(y)E_{m_1}(y)\cdots E_{m_h}(y)F_{m_{h+1}}(y)\cdots F_{m_i}(y)
    \frac{\partial^i\ddd{R}}{\partial x^h \partial y^{i-h}}|_{\ddd{C}})\leq \\
    &\leq (i-2)(n_r+n_c-1)+2n_r\sum_{j=1}^h m_j - 2n_rh + n_s\sum_{j=1}^hm_j +
    + n_c\sum_{j=1}^hm_j-\\ &- 2\sum_{j=1}^hm_j + h + 2n_r\sum_{j=h+1}^im_j - (i-h)n_r
    + n_s\sum_{j=h+1}^im_j + n_c\sum_{j=h+1}^i - (i-h)n_c-\\ &
    - 2\sum_{j=h+1}^i + 2(i-h) +n_r-i+h=\\
    &=(i-2)(n_r+n_c-1)+2n_r(m+1)+n_s(m+1)+n_c(m+1)-2(m+1)-\\
    &-2n_rh+h-(i-h)n_r-(i-h)n_c+2(i-h)+n_r-i-h.
  \end{align*}
  V zadnji enakosti smo uprabili, da je $\sum_{j=1}^im_j=m+1$. Nadaljujmo za računom:
  \begin{align*}
    &st(D^{i-2}(y)E_{m_1}(y)\cdots E_{m_h}(y)F_{m_{h+1}}(y)\cdots F_{m_i}(y)
    \frac{\partial^i\ddd{R}}{\partial x^h \partial y^{i-h}}|_{\ddd{C}})\leq \\
    &\leq(i-2)(n_r+n_c-1)+2n_r(m+1)+n_s(m+1)+n_c(m+1)-2(m+1)-\\
    &-2n_rh+h-(i-h)n_r-(i-h)n_c+2(i-h)+n_r-i-h \leq \\
    &\leq 2mn_r+(m+1)n_s+mn_c-2m-n_rh+n_ch-n_c+n_r = \\
    &=2mn_r + (m+1)n_s + mn_c-2m+(h-1)(n_c-n_r)\leq \\
    &\leq 2mn_r + (m+1)n_s +mn_c-2m.
  \end{align*}
  V zadnji neenakosti smo uporabili, da je $n_c\leq n_r$, torej je $n_c-n_r\leq 0$.
  S tem smo torej pokazali, da je $st(\ddd{S}_{m+1})\leq 2mn_r + (m+1)n_s+(m+1)n_c-2m$.

  Iz enačbe \todo{sklic} je razvidno naslednje:
  \begin{align*}
    &st(E_{m+1})\leq st(W_{1x})\leq st(\ddd{W}_1)\leq st(\ddd{S}_{m+1})+
    st(\frac{\partial \ddd{R}}{\partial y}|_{\ddd{C}}) \leq \\
    &\leq (2mn_r+(m+1)n_s+mn_c-2m)+(n_c-1)=\\
    &=2mn_r + (m+1)n_s + (m+1)n_c -2m -1.
  \end{align*}

  Podobno dobimo oceno za stopnjo polinoma $F_{m+1}$:
  \begin{align*}
    &st(F_{m+1})\leq st(W_{2x})\leq st(\ddd{W}_2)\leq st(\ddd{S}_{m+1})+
    st(\frac{\partial \ddd{R}}{\partial x}|_{\ddd{C}}) \leq \\
    &\leq (2mn_r+(m+1)n_s+mn_c-2m)+n_r=\\
    &=(2m+1)n_r + (m+1)n_s + mn_c -2m.
  \end{align*}
  S tem smo dokazali izrek še za $k>1$.
\end{proof}

%\subsection{Kompatibilnostni pogoji}
%
%enačba \ref{eq13}
%
%vstavimo $\alpha_{m+1}(y)=\frac{E_{m+1}(y)}{D^{2m+1}(y)}$ in 
%$\beta_{m+1}(y)=\frac{F_{m+1}(y)}{D^{2m+1}(y)}$
%
%ker je bil $m$ v dokazu izreka poljuben, lahko zamenjamo $m+1$ s $k$, kjer je 
%$k=1,\ldots,n$
%
%$$D(y)\ddd{S}_{k}(y)=E_{k}(y)\frac{\partial\ddd{R}}{\partial x}|_{\ddd{C}}+
%F_{k}(y)\frac{\partial \ddd{R}}{\partial y}|_{C}$$
%
%ta enačba je ekvivalentna enačbi v izreku
%
%rabili jo bomo za preučevanje pogojev, ki omejujejo izbiro koeficientnih funkcij
%
%ker ??, koeficientne funkcije ne morejo biti povsem poljubne, 
%zadoščati morajo nekaterim pogojem - kompatibilnostni pogoji 
%
%\todo{v praksi: nizke stopnje koeficientnih polinomov - kompatibilnostni pogoji 
%izginejo - to daj na konec, povezava s primerom}
%
%\todo{kompatibilnostni pogoji določijo nekatere kontrolne točke pri konstrukciji}
%
%zaradi lažje notacije označimo 
%$\ddd{A}(y)=\frac{\partial \ddd{R}}{\partial x}|_{\ddd{C}}(y)$, 
%$\ddd{B}(y)=\frac{\partial \ddd{R}}{\partial y}|_{\ddd{C}}(y)$, 
%$\ddd{W}(y)=\ddd{S}_k(y)$, $E(y)=E_k(y)$, $F(y)=F_k(y)$
%
%$$D(y)\ddd{W}(y)=E(y)\ddd{A}(y)+
%F(y)\ddd{B}(y)$$
%
%$st(D)+st(\ddd{W})=st(E)+st(\ddd{A})=st(F)+st(\ddd{B})=n$
%
%$st(\ddd{W})=n_w$, $st(\ddd{A})=n_a$, $st(\ddd{B})=n_b$
%
%potem $st(D)=n_d=n-n_w$, $st(E)=n_e=n-n_a$, $st(F)=n_f=n-n_b$
%
%\todo{nekako povej, da imajo W, A, B kontrolne vektorje}
%
%$\{\ddd{w}_i; i=0,1,\ldots,n_w\}$, $\{\ddd{a}_i; i=0,1,\ldots,n_a\}$, 
%$\{\ddd{b}_i; i=0,1,\ldots, n_b\}$ 
%kontrolni vektorji od $\ddd{W}$, $\ddd{A}$ in $\ddd{B}$. 
%če so $D$, $E$ in $F$ polinomi v Bernsteinovi bazi, naj bodo 
%$\{d_i, i=0,1,\ldots, n_d\}$, $\{e_i; i=0,1,\ldots, n_e\}$ in 
%$\{f_i; i=0,1,\ldots, n_f\}$ 
%kontrolni koeficienti teh polinomov.
%
%označimo 
%$\ddd{D}_i = [\underbrace{0,0,\ldots,0}_i,d_0,n_dd_0,\ldots {n_d \choose j}d_j,
%\ldots, n_dd_{n_d-1},d_{n_d},\underbrace{0,0,\ldots,0}_{n_w-i}]^T$
%
%$\ddd{E}_i = [\underbrace{0,0,\ldots,0}_i,e_0,n_ee_0,\ldots {n_e \choose j}e_j,
%\ldots, n_ee_{n_e-1},e_{n_e},\underbrace{0,0,\ldots,0}_{n_a-i}]^T$
%
%$\ddd{F}_i = [\underbrace{0,0,\ldots,0}_i,f_0,n_ff_0,\ldots {n_f \choose j}f_j,
%\ldots, n_ff_{n_f-1},f_{n_f},\underbrace{0,0,\ldots,0}_{n_b-i}]^T$
%
%$\overline{\ddd{W}}=[\ddd{w}_0,n_w\ddd{w}_1,\ldots,{n_w \choose i}\ddd{w}_i,
%\ldots, n_w\ddd{w}_{n_w-1},\ddd{w}_{n_w}]^T$
%
%$\overline{\ddd{A}}=[\ddd{a}_0,n_a\ddd{a}_1,\ldots,{n_a \choose i}\ddd{a}_i,
%\ldots, n_a\ddd{a}_{n_a-1},\ddd{a}_{n_a}]^T$
%
%$\overline{\ddd{B}}=[\ddd{b}_0,n_w\ddd{b}_1,\ldots,{n_b \choose i}\ddd{b}_i,
%\ldots, n_b\ddd{b}_{n_b-1},\ddd{b}_{n_b}]^T$
%
%sestavimo matrike
%$M_d = [\ddd{D}_0, \ddd{D}_1, \ldots, \ddd{D}_{n_w}]$
%
%$M_e = [\ddd{E}_0, \ddd{E}_1, \ldots, \ddd{E}_{n_a}]$
%
%$M_f = [\ddd{F}_0, \ddd{F}_1, \ldots, \ddd{F}_{n_b}]$
%
%dimenzij $(n+1)\times (n_w+1)$, $(n+1)\times(n_a+1)$, $(n+1)\times(n_b+1)$
%
%\begin{align*}
%  M_d=
%\begin{bmatrix}
%  M_d^1\\
%  M_d^2
%\end{bmatrix} 
%\end{align*}, 
%\begin{align*}
%  M_e&=
%\begin{bmatrix}
%  M_e^1\\
%  M_e^2
%\end{bmatrix} 
%\end{align*}, 
%\begin{align*}
%  M_f&=
%\begin{bmatrix}
%  M_f^1\\
%  M_f^2
%\end{bmatrix} 
%\end{align*}, 
%
%$M_d^1$, $M_e^1$ in $M_f^1$ so dimenzij $(n_w+1)\times(n_w+1)$, $(n_a+1)\times(n_a+1)$ 
%in $(n_b+1)\times (n_b+1)$. 
%
%$M_d^2$, $M_e^2$ in $M_f^2$ so dimenzij $n_d\times(n_w+1)$, $n_e\times(n_a+1)$ 
%in $n_f \times (n_b+1)$
%
%$M_d^1\overline{\ddd{W}}=M_e^1\overline{\ddd{A}}+M_f^1\overline{\ddd{B}}$
%
%$M_d^2\overline{\ddd{W}}=M_e^2\overline{\ddd{A}}+M_f^2\overline{\ddd{B}}$
%
%$M_d^1$ je obrnljiva
%\todo{blabla}
%
%$\overline{\ddd{W}}=(M_d^1)^{-1}M_e^1\overline{\ddd{A}}+(M_d^1)^{-1}M_f^1\overline{\ddd{B}}$
%
%$(M_e^2-M_d^2(M_d^1)^{-1}M_e^1)\overline{\ddd{A}} + 
%(M_f^2-M_d^2(M_d^1)^{-1}M_f^1)\overline{\ddd{B}} = 0$

\section{Primeri konstrukcij geometrijsko zveznih ploskev}

\todo{nek uvod}

\subsection{Konstrukcija $G^1$-zveznih Bézierjevih ploskev iz tenzorskega produkta}

V tem podpoglavju si bomo ogledali, kako na različne načine konstruirati ploskvi, ki 
sta na stiku $G^1$-zvezni, torej kakšne pogoje prinesejo različni načini konstrukcije 
za njune kontrolne točke oziroma kontrolne vektorje. Pri tem bomo predpostavljali, 
da so robovi obeh ploskev vnaprej določeni. Pogoje, ki jih prinese zahteva $G^1$-zveznosti 
bomo primerjali z $C^1$-zveznostjo. 

Imejmo dve bikubični Bézierjevi ploskvi iz tenzorskega produkta:
$$\ddd{R}(u,v)=\sum_{i=0}^3 \sum_{j=0}^3 \mathbf{P}_{i,j}B_i^3(u) B_j^3(v)$$ in
$$\ddd{S}(u,v)=\sum_{i=0}^3 \sum_{j=0}^3 \mathbf{Q}_{i,j}B_i^3(u) B_j^3(v),$$
kjer velja $u,v\in[0,1]$.

Ploskvi $\ddd{R}$ in $\ddd{S}$ se stikata v $\ddd{C}(v)=\ddd{R}(0,v)=\ddd{S}(0,v)$, torej naj velja
$$\ddd{C}(v)=\sum_{i=0}^{n_c} \ddd{Z}_i B_i^{n_c}(v),$$
kjer so $\ddd{Z}_i = \ddd{P}_{0,i} = \ddd{Q}_{0,i}$ kontrolne točke krivulje $\ddd{C}$. 
Stopnja $n_c$ krivulje $\ddd{C}$ ni nujno enaka $3$, veljati pa mora $n_c\leq 3$. 
Obravnavali bomo primere, v katerih je $n_c=3$ in primer, kjer je $n_c=2$.
Predpostavili bomo, da so robne krivulje ploskev $\ddd{R}$ in $\ddd{S}$ 
že določene na tak način, da bomo imeli na robu zahtevano zveznost. 
Zanimalo nas bo,  
kakšne zveze v teh primerih veljajo za notranje kontrolne točke, torej za 
$\ddd{P}_{1,1}$, $\ddd{P}_{1,2}$, $\ddd{Q}_{1,1}$ in $\ddd{Q}_{1,2}$., 
da bo stik ploskev $G^1$-zvezen. 
\todo{razlaga s ploščinami trikotnikov?} 
V nadaljevanju bomo uporabljali še naslednje oznake za kontrolne vektorje obeh 
ploskev in robne krivulje:
$\ddd{p}_{i,j}=\ddd{P}_{i+1,j}-\ddd{P}_{i,j}$, 
$\ddd{q}_{i,j}=\ddd{Q}_{i+1,j}-\ddd{Q}_{i,j}$ in 
$\ddd{z}_i=\ddd{Z}_{i+1}-\ddd{Z}_i$.

Najprej si oglejmo, kakšne pogoje in omejitve nam da zahteva $C^1$ zveznosti 
na stiku teh dveh ploskev.

\begin{primer}\label{primer1}
  Domena ploskev $\ddd{R}$ in $\ddd{S}$ je kvadrat $[0,1]\times[0,1]$. Da lahko 
  obravnavamo $C^1$-zveznost stika ploskev, moramo najprej reparametrizirati 
  ploskev $\ddd{R}$ tako, da bo njena domena $[-1,0]\times[0,1]$ in bosta obe domeni 
  skupaj po stiku tvorili pravokotnik $[-1,1]\times[0,1]$. Da to dosežemo, 
  moramo ploskev $\ddd{R}$ zapisati na naslednji način:
  $$\ddd{R}(u,v) = \sum_{i=0}^3 \sum_{j=0}^3 \mathbf{P}_{i,j}B_i^3(-u) B_j^3(v),$$
  kjer je $u\in[-1,0]$ in $v\in[0,1]$.

  Da je stik ploskev $\ddd{R}$ in $\ddd{S}$ $C^1$-zvezen, se morata krivulji 
  ujemati v kontrolnih točkah, ki določajo stično krivuljo: 
  $\ddd{P}_{0,j}=\ddd{Q}_{0,j}$ za $j=0,\ldots,3$, s čimer dosežemo $C^0$-zveznost. 
  Poleg tega pa se morata ujemati še odvoda obeh ploskev v $u$-smeri v robnih 
  točkah: 
  $\frac{\partial}{\partial u}\ddd{R}(u,v)|_{u=0}=\frac{\partial}{\partial u}\ddd{S}(u,v)|_{u=0}$.
  Če razpišemo oba parcialna odvoda, dobimo naslednje pogoje:
  $$-(\ddd{P}_{1,j}-\ddd{P}_{0,j})=\ddd{Q}_{1,j}-\ddd{Q}_{0,j}$$
  oziroma 
  $$\ddd{q}_{0,j}=-\ddd{p}_{0,j}$$
  za $j=0,\ldots,3$.

  Vidimo torej, da morata biti za dosego $C^1$-zveznosti zlepka ploskev vektorja 
  $\ddd{p}_{0,j}$ in $\ddd{q}_{0,j}$ kolinearna za vsak $j=0,\ldots,3$, poleg tega 
  pa morata biti njuni dolžini v razmerju, ki ga določata parametrizaciji obeh 
  ploskev. Kar se tiče oblike ploskve, ki jo na ta način lahko konstruiramo kot 
  zlepek ploskev $\ddd{R}$ in $\ddd{S}$, torej nimamo ravno veliko izbire. 
  Nekoliko več svobode imamo le pri izbiri notranjih kontrolnih točk. 
  Kontrolni točki $\ddd{Q}_{1,1}$ in $\ddd{Q}_{1,2}$ sta točno določeni 
  z izbiro kontrolnih točk $\ddd{P}_{1,1}$ in $\ddd{P}_{1,2}$, medtem ko sta 
  $\ddd{P}_{1,1}$ in $\ddd{P}_{1,2}$ prosti. Ker zahtevamo zgolj zveznost stopnje 
  1, so proste tudi kontrolne točke $\ddd{Q}_{2,1}$, $\ddd{Q}_{2,2}$, 
  $\ddd{P}_{2,1}$ in $\ddd{P}_{2,2}$.

\end{primer}

Sedaj si oglejmo nekaj primerov konstrukcij $G^1$-zveznih ploskev in jih primerjajmo 
z rezultatom, dobljenim v primeru \ref{primer1}. 
Izrek \ref{izrek1} pravi, da je stik obeh ploskev $G^1$-zvezen, natanko tedaj ko obstajata funkciji 
$\alpha_1(v)$ in $\beta_1(v)$, \todo{pogoj regularnosti? lastnosti teh funkcij?} da velja 
$$\frac{\partial \ddd{S}}{\partial u}|_{u=0} = \alpha_1(v)\frac{\partial \ddd{R}}{\partial u}|_{u=0} + 
\beta_1(v)\frac{\partial \ddd{R}}{\partial v}|_{u=0},$$
kjer je $\alpha_1(v)\neq 0$ na intervalu $[0,1]$ in ima ustrezen predznak.
V našem primeru gre za polinomske ploskve, zato lahko uporabimo izrek \ref{izrek2},
ki pove, da to velja natanko tedaj, 
ko obstajajo polinomi $D(v)$, $E_1(v)$ 
in $F_1(v)$, kjer sta polinoma $D$ in $E_1$ stopnje največ 5, polinom $F_1$ pa stopnje 
največ 6, da velja
\begin{align}\label{eq14}
D(v)\frac{\partial \ddd{S}}{\partial u}|_{u=0} = E_1(v)\frac{\partial \ddd{R}}{\partial u}|_{u=0} + 
F_1(v)\frac{\partial \ddd{R}}{\partial v}|_{u=0}.
\end{align}

Razpišimo prve odvode parametrizacij ploskev $\ddd{R}$ in $\ddd{S}$: 
\todo{tu se nekako sklicuješ na formulo za odvod in evalvacijo v 0}
$$\frac{\partial \ddd{S}}{\partial u}|_{u=0}=3\sum_{i=0}^2\sum_{j=0}^3
(\ddd{P}_{i+1,j}-\ddd{P}_{i,j})B_i^2(u)B_j^3(v)|_{u=0}=
3\sum_{j=0}^3(\ddd{P}_{1,j}-\ddd{P}_{0,j})B_j^3(v),$$
$$\frac{\partial \ddd{R}}{\partial u}|_{u=0}=3\sum_{i=0}^2\sum_{j=0}^3
(\ddd{Q}_{i+1,j}-\ddd{Q}_{i,j})B_i^2(u)B_j^3(v)|_{u=0}=
3\sum_{j=0}^3(\ddd{Q}_{1,j}-\ddd{Q}_{0,j})B_j^3(v),$$
in 
$$\frac{\partial \ddd{R}}{\partial v}|_{u=0}=3\sum_{i=0}^3\sum_{j=0}^2
(\ddd{Q}_{i,j+1}-\ddd{Q}_{i,j})B_i^2(u)B_j^3(v)|_{u=0}=
3\sum_{j=0}^2(\ddd{Q}_{0,j}-\ddd{Q}_{0,j+1})B_j^2(v).$$

Dobljeno vstavimo v enačbo \ref{eq14}. Vidimo, da mora veljati:
\begin{equation}\label{eq15}
  \begin{split} 
&D(v)\sum_{j=0}^3(\ddd{P}_{1,j}-\ddd{P}_{0,j})B_j^3(v) = \\
&=E_1(v)\sum_{j=0}^3(\ddd{Q}_{1,j}-\ddd{Q}_{0,j})B_j^3(v) + 
F_1(v)\sum_{j=0}^2(\ddd{Q}_{0,j}-\ddd{Q}_{0,j+1})B_j^2(v).
  \end{split}
\end{equation}

Najprej si oglejmo, kakšni pogoji v primeru $G^1$ zveznosti veljajo za robne 
kontrolne točke ploskev $\ddd{R}$ in $\ddd{S}$. V enačbo \ref{eq15} vstavimo 
vrednosti $v=0$ in $v=1$.

Pri vrednosti $v=0$ dobimo:
$$(\ddd{Q}_{1,0}-\ddd{Q}_{0,0}) = a_0(\ddd{P}_{1,0}-\ddd{P}_{0,0})+b_0(\ddd{P}_{0,1}-\ddd{P}_{0,0}),$$
oziroma:
$$\ddd{q}_{0,0} = a_0 \ddd{p}_{0,0} + b_0 \ddd{z}_0.$$  
Tu smo z $a_0$ označili vrednost $\alpha_1(0)$, z $b_0$ pa vrednost $\beta_1(0)$.
Pri vrednosti $v=1$ pa dobimo:
$$\ddd{Q}_{1,3}= a_1(\ddd{P}_{1,3}-\ddd{P}_{0,3})+b_1(\ddd{P}_{0,3}-\ddd{P}_{0,2}),$$
oziroma:
$$\ddd{q}_{0,3} = a_1 \ddd{p}_{0,3} + b_1 \ddd{z}_2.$$
Tu smo z $a_1$ označili vrednost $\alpha_1(1)$ in z $b_1$ vrednost $\beta_1(1)$.

Pogoji, ki veljajo za robne kontrolne točke so enaki neglede na način konstrukcije 
$G^1$-zveznega zlepka ploskev. Pogoji, ki veljajo za notranje kontrolne točke, 
število svobodnih parametrov, ki določajo obliko dobljene ploskve, in število 
prostih kontrolnih točk pa so odvisni od izbire načina konstrukcije, natančneje, 
od izbire stopnje koeficientnih polinomskih funkcij in stopnje odvodov parametrizacij 
ploskev $\ddd{R}$ in $\ddd{S}$.

Izbira stopenj koeficientnih funkcij ni povsem poljubna, temveč je odvisna od 
stopnje geometrijske zveznosti, ki jo zahtevamo, pa tudi od
stopenj odvodov $\frac{\partial \ddd{S}}{\partial u}|_{u=0}$, 
$\frac{\partial \ddd{R}}{\partial u}|_{u=0}$ in $\frac{\partial \ddd{R}}{\partial v}|_{u=0}$ 
oziroma $\ddd{C}'(v)$. 

V praksi se običajno uporabljajo koeficientne funkcije čim nižje stopnje, saj s 
tem dobimo manj pogojev za kontrolne točke. V primeru, da sta funkciji $D(v)$ in 
$E(v)$ konstantni, funkcija $F(v)$ pa kvečjemu linearna, dobimo pogoje le za dve 
notranji kontrolni točki, vse ostale pa so proste, podobno kot v primeru 
$C^1$-zveznosti (primer \ref{primer1}). Če za koeficientne funkcije izberemo polinome 
višjih stopenj, se lahko zgodi, da dobimo pogoje za tri ali štiri kontrolne točke. 

Najprej si oglejmo situacijo, v kateri za koeficientne funkcije izberemo polinome 
minimalne stopnje.

\begin{primer}\label{primer2}
  Da zagotovimo $G^1$-zveznost na stiku ploskev $\ddd{R}$ in $\ddd{S}$, mora 
  poleg pogoja, da se ploskvi stikata v robni krivulji, veljati enakost \ref{eq15}, 
  oziroma 
  $$D(v)\sum_{j=0}^3 \ddd{q}_j B_j^3(v) = E_1(v) \sum_{j=0}^3 \ddd{p}_j B_j^3(v) + 
  F_1(v) \sum_{j=0}^2 \ddd{z}_j B_j^2(v).$$

  Ker je stopnja krivulj $\frac{\partial \ddd{S}}{\partial u}|_{u=0}$ in 
  $\frac{\partial \ddd{R}}{\partial u}|_{u=0}$ enaka 3, stopnja krivulje 
  $\frac{\partial \ddd{R}}{\partial v}|_{u=0}$ pa 2 in če njihovih stopenj 
  ne nižamo oziroma višamo, bodo stopnje polinomov $D(v)$, $E_1(v)$ in $F_1(v)$ 
  minimalne, če bosta $D(v)$ in $E_1(v)$ konstantna polinoma, $F_1(v)$ pa linearen. 
  V tem primeru namreč obe strani enačbe predstavljata Bézierjevo krivuljo stopnje 3.

  
Ker mora veljati še $\alpha_1(0)=a_0$, $\alpha_1(1)=a_1$, $\beta_1(0)=b_0$ in 
$\beta_1(1)=b_1$, mora veljati: 
$$\alpha_1(v)= a_0 = a_1.$$
in 
$$\beta_1(v) = b_0(1-v) + b_1v.$$
\todo{kompatibilnostni pogoji izginejo ? ?}
Vstavimo dobljena polinoma $\alpha_1$ in $\beta_1$ v enačbo \todo{sklic} in dobimo:
\begin{align*}
  &\sum_{j=0}^3\ddd{q}_jB_j^3(v) = a_0\sum_{j=0}^3\ddd{p}_jB_j^3(v) + 
  (b_0(1-v)+b_1v)\sum_{j=0}^2\ddd{z}_jB_j^2(v)=\\
  &=a_0\sum_{j=0}^3\ddd{p}_jB_j^3(v) + \sum_{j=0}^2\ddd{z}_jb_0{2 \choose j}v^j(1-v)^{3-j} + 
  \sum_{j=0}^2\ddd{z}_jb_1{2 \choose j}v^{j+1}(1-v)^{2-j}=\\
  &=\sum_{j=0}^3a_0\ddd{p}_jB_j^3(v) + \sum_{j=0}^3\ddd{z}_jb_0{2 \choose j}v^j(1-v)^{3-j} + 
  \sum_{j=1}^3b_1\ddd{z}_{j-1}{2 \choose j-1}v^j(1-v)^{3-j}=\\
  &=\sum_{j=0}^3a_0\ddd{p}_jB_j^3(v) + \sum_{j=0}^3b_0\ddd{s}_j\frac{3-j}{3}{3 \choose j}v^j(1-v)^{3-j} + 
  \sum_{j=1}^3b_1\ddd{z}_{j-1}\frac{j}{3}{3 \choose j}v^3(1-v)^{3-j}=\\
  &=\sum_{j=0}^3a_0\ddd{p}_jB_j^3(v) + \sum_{j=0}^3b_0\ddd{z}_j\frac{3-j}{3}B_j^3(v) + 
  \sum_{j=0}^3b_1\ddd{z}_{j-1}\frac{j}{3}B_j^3(v)
\end{align*}
\todo{V zadnji vrstici zgornjega izračuna smo uporabili, da je v vsoti $\sum_{j=0}^3b_1\ddd{z}_{j-1}\frac{j}{3}B_j^3(v)$ 
člen pri $j=0$ enak 0.}

\todo{ali moram to bolje razložiti}Od tod dobimo pogoje za kontrolna vektorja $\ddd{q}_1$ in $\ddd{q}_2$:
$$\ddd{q}_1=a_0\ddd{p}_1+\frac{1}{3}b_1\ddd{z}_0+\frac{2}{3}b_0\ddd{z}_1$$
in
$$\ddd{q}_2=a_0\ddd{p}_2+\frac{2}{3}b_1\ddd{z}_1+\frac{1}{3}b_0\ddd{z}_2.$$

\end{primer}


\todo{SITUACIJA 2}

V tem primeru pa predpostavimo, da je krivulja $\frac{\partial \ddd{R}}{\partial u}|_{u=0}$ 
stopnje 2, torej jo določajo kontrolni vektorji $\ddd{p}_0$, $\ddd{p}_m$ in $\ddd{p}_3$. 
\todo{vektorja p0 in p3 sta že določena zaradi določenosti robnih krivulj}

\todo{kompatibilnostni pogoji} 

\todo{ne vem, kako naj to. tega ne razumem zakaj je linearna kombinacija in kako 
je to enako ne razumem}

$$\frac{1}{3}\frac{\partial \ddd{R}}{\partial u}|_{u=0} = 
\sum_{i=0}^3\ddd{p}_iB_i^3(v)=(1-v)^2\ddd{p}_0 + 2(1-v)v\ddd{p}_m + v^2\ddd{p}_3$$
\todo{zakaj se p1 in p2 zapisujeta kot lin. kombinaciji p0, pm, p3?}

\nevem{Vektorji $\ddd{p}_0$, $\ddd{p}_m$ in $\ddd{p}_1$ so del iste ravnine. Enako velja za vektorje 
$\ddd{p}_2$, $\ddd{p}_m$ in $\ddd{p}_3$, zato lahko vektorja $p_1$ in $p_2$ zapišemo 
na naslednji način:}
$$\ddd{p}_1=a\ddd{p}_m+b\ddd{p}_0$$
$$\ddd{p}_2= c\ddd{p}_m+d\ddd{p}_3$$
Dobljeno vstavimo v izraz $\sum_{i=0}^3\ddd{p}_iB_i^3(v)$.
\begin{align*}
  \sum_{i=0}^3\ddd{p}_iB_i^3(v)&=\ddd{p}_0(1-v)^3+2a\ddd{p}_m(1-v)^2v+3b\ddd{p}_0(1-v)^2v+
  3c\ddd{p}_m(1-v)v^2+\\ &+3d\ddd{p}_3(1-v)v^2+v^3\ddd{p}_3=\\
  &=\ddd{p}_0(1-v)^2(1-v+3bv)+\ddd{p}_m(1-v)v(3a(1-v)+3cv)+\\ &+\ddd{p}_3v^2(3d(1-v)+v)
\end{align*}
Sedaj to vstavimo v enačbo \todo{sklic} in enačbimo koeficiente\todo{?} pred 
vektorji $\ddd{p}_0$, $\ddd{p}_m$ in $\ddd{p}_3$, 
da dobimo enačbe za izračun vrednosti 
$a$, $b$, $c$ in $d$: 
$3b-1=0$, $3a=2$, $3c-3a=0$, $3d=1.$ 

Od tod sledi $b=\frac{1}{3}$, $a=\frac{2}{3}$, $c=\frac{2}{3}$ in $d=\frac{1}{3}$. 
Dobili smo, da se $\ddd{p}_1$ in $\ddd{p}_2$ izražata na naslednji način:
$$\ddd{p}_1=\frac{2}{3}\ddd{p}_m+\frac{1}{3}\ddd{p}_0$$
$$\ddd{p}_2=\frac{2}{3}\ddd{p}_m+\frac{1}{3}\ddd{p}_3.$$

Kot v \todo{situaciji 1} ponovno zapišimo enačbo \todo{sklic}, ki velja zaradi $G^1$ 
zveznosti ploskev $\ddd{R}$ in $\ddd{S}$.
$$\sum_{j=0}^3\ddd{q}_jB_j^3(v)=\alpha_1(v)\sum_{j=0}^2\tilde{\ddd{p}}_jB_j^2(v)+
\beta_1(v)\sum_{j=0}^2\ddd{s}_jB_j^2(v)$$
kjer smo označili $\tilde{\ddd{p}}_0=\ddd{p}_0$, $\tilde{\ddd{p}}_1=\ddd{p}_m$ in $\tilde{\ddd{p}}_2=\ddd{p}_3$

Ponovno zaradi enostavnosti predpostavimo, da sta funkciji $\alpha_1(y)$ in $\beta_1(y)$ 
polinoma. Ponovno opazimo, da imamo na levi strani enačbe Bézierjevo krivulo stopnje 3, 
torej mora tudi desna stran predstavljati krivuljo stopnje 3. Polinoma $\alpha_1(y)$ in 
$\beta_1(y)$ morata biti zato linerana:
$$\alpha_1(v)=a_0(1-v)+a_1v$$
$$\beta_1(v)=b_0(1-v)+b_1v$$

Vstavimo polinoma v enačbo \todo{sklic} in na podoben način kot v \todo{situaciji 1} dobimo:
\begin{align*}
  &\sum_{j=0}^3\ddd{q}_jB_j^3(v)=(a_0(1-v)+a_1v)\sum_{j=0}^2\tilde{\ddd{p}}_jB_j^2(v)
  + (b_0(1-v)+b_1v)\sum_{j=0}^2\ddd{s}_jB_j^2(v) =\\
  &=a_0\sum_{j=0}^2\tilde{\ddd{p}}_j{2 \choose j}v^j(1-v)^{3-j} + 
  a_1\sum_{j=0}^2\tilde{\ddd{p}}_j{2 \choose j}v^{j+1}(1-v)^{2-j} +\\ 
  &+b_0\sum_{j=0}^2\tilde{\ddd{z}}_j{2 \choose j}v^j(1-v)^{3-j} + 
  b_1\sum_{j=0}^2\tilde{\ddd{z}}_j{2 \choose j}v^{j+1}(1-v)^{2-j}=\\
  &=\sum_{j=0}^3(a_1\tilde{\ddd{p}}_{j-1}\frac{j}{3}+a_0\tilde{\ddd{p}}_j\frac{3-j}{3}+
  b_1\ddd{z}_{j-1}\frac{j}{3}+b_0\ddd{z}_j\frac{3-j}{3})B_j^3(v)
\end{align*}
\todo{tu sem kar spustila vse korake, ker so isti}

Od tod sledijo pogoji za vektorja $\ddd{q}_1$ in $\ddd{q}_2$. 
$$\ddd{q}_1=\frac{1}{3}(a_1\ddd{p}_0+2a_0\ddd{p}_m+b_1\ddd{z}_0+2b_0\ddd{z}_1)$$
$$\ddd{q}_2=\frac{1}{3}(a_0\ddd{p}_3+2a_1\ddd{p}_m+2b_1\ddd{z}_1+b_0\ddd{z}_2).$$

Seveda pa to nista edini možni situaciji.


%\section{Tehnični napotki za pisanje}
%
%\subsection{Sklicevanje in citiranje}
%Za sklice uporabljamo \verb|\ref|, za sklice na enačbe \verb|\eqref|, za citate \verb|\cite|. Pri
%sklicevanju in citiranju sklicano številko povežemo s prejšnjo besedo z nedeljivim presledkom
%$\sim$, kot npr.\ \verb|iz trditve~\ref{trd:obstoj-omega} vidimo|.
%
%\begin{primer}
%  Zaporedje~\eqref{eq:zero-kompleks} iz dokaza trditve~\ref{trd:obstoj-omega} na
%  strani~\pageref{trd:obstoj-omega} lahko najdemo tudi v Spletni enciklopediji zaporedij~\cite{oeis}.
%  Citiramo lahko tudi bolj natančno~\cite[trditev 2.1, str.\ 23]{lebedev2009introduction}.
%\end{primer}
%
%\subsection{Okrajšave}
%Pri uporabi okrajšav \LaTeX{} za piko vstavi predolg presledek, kot npr. tukaj. Zato se za vsako
%piko, ki ni konec stavka doda presledek običajne širine z ukazom \verb*|\ |, kot npr.\ tukaj.
%Primerjaj z okrajšavo zgoraj za razliko.
%
%\subsection{Vstavljanje slik}
%Sliko vstavimo v plavajočem okolju \texttt{figure}. Plavajoča okolja \emph{plavajo} po tekstu, in
%jih lahko postavimo na vrh strani z opcijskim parametrom `\texttt{t}', na lokacijo, kjer je v kodi s
%`\texttt{h}', in če to ne deluje, potem pa lahko rečete \LaTeX u, da ga \emph{res} želite tukaj,
%kjer ste napisali, s `\texttt{h!}'. Lepo je da so vstavljene slike vektorske (recimo \texttt{.pdf}
%ali \texttt{.eps} ali \texttt{.svg}) ali pa \texttt{.png} visoke resolucije (več kot
%\unit[300]{dpi}).  Pod vsako sliko je napis in na vsako sliko se skličemo v besedilu. Primer
%vektorske slike je na sliki~\ref{fig:sample}. Vektorsko sliko prepoznate tako, da močno
%zoomate v sliko, in še vedno ostane gladka. Več informacij je na voljo na
%\url{https://en.wikibooks.org/wiki/LaTeX/Floats,_Figures_and_Captions}. Če so slike bitne, kot na
%primer slika~\ref{fig:image}, poskrbite, da so v dovolj visoki resoluciji.

%\begin{figure}[h]
%  \centering
%  \includegraphics[width=0.6\textwidth]{images/sample.pdf}
%% \caption[caption za v kazalo]{Dolg caption pod sliko}
%  \caption[Primer vektorske slike.]{Primer vektorske slike z oznakami v enaki pisavi, kot jo
%     uporablja \LaTeX{}.  Narejena je s programom Inkscape, \LaTeX{} oznake so importane v
%     Inkscape iz pomožnega PDF.}
%  \label{fig:sample}
%\end{figure}

%\begin{figure}[h]
%  \centering
%  \includegraphics[width=0.8\textwidth]{images/image.png}
%  \caption[Primer bitne slike.]{Primer bitne slike, izvožene iz Matlaba. Poskrbite, da so slike v
%  dovolj visoki resoluciji in da ne vsebujejo prosojnih elementov (to zahteva PDF/A-1b format).}
%  \label{fig:image}
%\end{figure}

%\subsection{Kako narediti stvarno kazalo}
%Dodate ukaze \verb|\index{polje}| na besede, kjer je pojavijo, kot tukaj\index{tukaj}.
%Več o stvarnih kazalih je na voljo na \url{https://en.wikibooks.org/wiki/LaTeX/Indexing}.
%
%\subsection{Navajanje literature}
%Članke citiramo z uporabo \verb|\cite{label}|, \verb|\cite[text]{label}| ali pa več naenkrat s
%\verb|\cite\{label1, label2}|. Tudi tukaj predhodno besedo in citat povežemo z nedeljivim presledkom
%$\sim$. Na primer~\cite{chen2006meshless,liu2001point}, ali pa \cite{kibriya2007empirical}, ali pa
%\cite[str.\ 12]{trobec2015parallel}, \cite[enačba (2.3)]{pereira2016convergence}.
%Vnosi iz \verb|.bib| datoteke, ki niso citirani, se ne prikažejo v seznamu literature, zato jih
%tukaj citiram.~\cite{vene2000categorical}, \cite{gregoric2017stopniceni}, \cite{slak2015induktivni},
%\cite{nsphere}, \cite{kearsley1975linearly}, \cite{STtemplate}, \cite{NunbergerTand}.

% Literatura:
% Primer navajanja na http://www.fmf.uni-lj.si/storage/24240/LiteraturaM.pdf,
% ampak bi moral stil poskrbeti za vse. Reference se uredijo po abecedi.
% Če nobena izbira izmed @book, @atricle,... ni ok, potem se lahko vse napiše v
% @misc pod note={} in deluje tako kot normalen LaTeX.
% Komentar v bib datoteki se naredi samo s parom { }
% Za urejanje literature avtor priporoča program Jabref, ki zna tudi avtomatsko
% okrajšati imena revij. Za pravilno sortiranje vnosov brez avtorja, uporabite
% polje key={ }, kot v primeru.
% V primeru napak ustvarite issue na GitHubu ali pišite na jure.slak@fmf.uni-lj.si.

\cleardoublepage                           % na desni strani
\phantomsection                            % da prav delujejo hiperlinki
%\addcontentsline{toc}{section}{\bibname}   % dodajmo v kazalo
%\bibliographystyle{fmf-sl}                 % uporabljen stil je v datoteki fmf-sl.bst, na voljo tudi angleška verzija
%\bibliography{\literatura}                 % literatura je v datoteki, definirani na začetku

% Za stvarno kazalo
\cleardoublepage                           % na desni strani
\phantomsection                            % da prav delujejo hiperlinki
\addcontentsline{toc}{section}{\indexname} % dodajmo v kazalo
\printindex

\end{document}
