\documentclass[11pt]{beamer}

\usetheme{Berlin}
\usecolortheme{rose}

\usepackage[slovene]{babel}
\usepackage[utf8]{inputenc} 
\usepackage[T1]{fontenc}
\usepackage{lmodern}
\usepackage {eurosym}
\usepackage{amsfonts}
\usepackage{graphicx}

\usepackage{amsmath,amssymb,amsthm}
\usepackage{mathrsfs}
\usepackage{amsbsy}

\newcommand{\R}{\mathbb R}
\newcommand{\N}{\mathbb N}
\newcommand{\Z}{\mathbb Z}
\newcommand{\C}{\mathbb C}
\newcommand{\Q}{\mathbb Q}

\setbeamercovered{invisible}

\newtheorem{izrek}{Izrek} 

\newtheorem{definicija}{Definicija} 
\newtheorem{lema}{Lema}

\title[Konstrukcija geometrijsko zveznih parametričnih ploskev]{Konstrukcija geometrijsko zveznih parametričnih ploskev}

\author{Katarina Černe}

\date{

 \small{FMF, Ljubljana, 22. 4. 2019}
}

\begin{document}

\begin{frame}
\maketitle
\end{frame}

\section{Geometrijska zveznost}

\begin{frame}
\begin{definicija}
Ploskev pripada razredu $G^n$ oziroma je geometrijsko zvezna z redom $n$, če v okolici vsake njene točke obstaja lokalna regularna parametrizacija razreda $C^n$.
\end{definicija}
\end{frame}

\begin{frame}
\begin{definicija}
Naj bosta $R(x,y)$ in $S(u,v)$ regularni $C^n$ parametrizaciji dveh ploskev, ki se stikata v krivulji $C(y)=R(x_0,y)=S(u_0,y)$. %povej da na C je y=v
 Pravimo, da se $R$ in $S$ stikata z $G^n$-zveznostjo vzdolž krivulje $C$, če za vsako točko $b_0=C(y_0)$ obstaja lokalno regularna $C^n$ reparametrizacijska funkcija $f(x,y)=(u(x,y),v(x,y))$, da je $f(x_0,y)=(u_0,y)$ za vsak $y \in I_0$ in da velja
$$\frac{\partial^{m+k}}{\partial x^m \partial y^k}R\Bigr|_{\substack{(x_0,y)}}=\frac{\partial^{m+k}}{\partial x^m \partial y^k}(S\circ f)\Bigr|_{\substack{(x_0,y)}} \textrm{ za } m+k=1,\ldots,n$$
za $y\in I_0$, kjer je $I_0$ neka okolica $y_0$.
\end{definicija}
%napiši R:omega1 pod r2 v r3, S:omega2pod r2 v r3
%povej, kaj pomeni regularna
%povej, da so odvodi vzdolž y=v enaki, zato je dovolj, da gledamo samo odvode vzdolž x in u
\end{frame}

\begin{frame}
$G^0$: $R=S \circ f$ vzdolž $C$ \\
$G^1$: $R_x = S_u u_x + S_v v_x$ vzdolž $C$ ter pogoj za $G^0$\\
$G^2$: $R_xx = S_{uu} u_x^2 + 2S_{uv}u_x v_x + S_{vv} v_x^2 + S_u u_{xx} + S_v v_{xx}$ vzdolž $C$ ter pogoja za $G^0$ in $G^1$.\\
\vdots
\end{frame}

\begin{frame}
$$\frac{\partial^j R}{\partial x^j}\Bigr|_{\substack{C}}= \sum_{k=1}^j\sum_{h=0}^k A_{jkh}\frac{\partial^k S}{\partial u^h \partial v^{k-h}}\Bigr|_{\substack{C}}$$
$$A_{jkh}={k \choose h}\sum_{\substack{m_1+\cdots +m_k =j \\ m_1, \ldots, m_k >0}}
 \frac{j!}{k!m_1!\cdots m_k!} u_{x^{m_1}} \cdots u_{x^{m_h}} v_{x^{m_{h+1}}} \cdots v_{x^{m_k}}\Bigr|_{\substack{C}}$$
%povej, da to velja za vsak j=1,...,n
\end{frame}

\begin{frame}
%zdaj si pogledamo ekvivalentno/alternativno definicijo G zveznosti
%definirajmo reparametrizacijo
Naj bodo $\alpha_1, \ldots, \alpha_n$ in $\beta_1 \ldots, \beta_n$ funkcije razreda $C^n$ ene spremenljivke. Definirajmo:\\
$$u(x,y) =  u_0 + \sum_{i=0}^n \frac{1}{i!}\alpha_i(y)(x-x_0)^i$$
$$v(x,y) = y + \sum_{i=0}^n \frac{1}{i!}\beta_i(y)(x-x_0)^i$$

Opazimo:\\
\begin{flalign*}\frac{\partial^i u}{\partial x^i}(x_0,y)&=\alpha_i(y)&\\
\frac{\partial^i v}{\partial x^i}(x_0,y)&=\beta_i(y)\textrm{ za }i=1, \ldots, k.\end{flalign*}
\end{frame}

\begin{frame}
Regularnost reparametrizacije:\\
\begin{flalign*}&\frac{\partial f}{\partial x}\times \frac{\partial f}{\partial y} \neq 0&\\
&\frac{\partial f}{\partial x}(x_0,y)=(\frac{\partial u}{\partial x}(x_0,y),\frac{\partial v}{\partial x}(x_0,y))=(\alpha_1(y),\beta_1(y))\\
&\frac{\partial f}{\partial y}(x_0,y)=(\frac{\partial u}{\partial y}(x_0,y),\frac{\partial v}{\partial y}(x_0,y))=(0,1)\\
&\Longrightarrow \alpha_1(y) \neq 0 \textrm{ vzdolž C}\end{flalign*}
\end{frame}

\begin{frame}
\begin{definicija}
Naj bosta $R(x,y)$ in $S(u,v)$ regularni $C^n$ parametrizaciji dveh ploskev, ki se stikata v krivulji $C(y)=R(x_0,y)=S(u_0,y)$. 
Pravimo, da se $R$ in $S$ stikata z $G^n$-zveznostjo vzdolž krivulje $C$, če za vsako točko $b_0=C(y_0)$ obstajajo take funkcije razreda $C^n$ ene spremenljivke $\alpha_1, \ldots, \alpha_n$ in $\beta_1 \ldots, \beta_n$ , da $\alpha_1(y)\neq0$ in velja
$$\frac{\partial^j R}{\partial x^j}\Bigr|_{\substack{C}}= \sum_{k=1}^j\sum_{h=0}^k A_{jkh}\frac{\partial^k S}{\partial u^h \partial v^{k-h}}\Bigr|_{\substack{C}}$$
$$A_{jkh}={k \choose h}\sum_{\substack{m_1+\cdots +m_k =j \\ m_1, \ldots, m_k >0}}
 \frac{j!}{k!m_1!\cdots m_k!} \alpha_1 \cdots \alpha_{m_h} \beta_{m_{h+1}} \cdots \beta_{m_k}\Bigr|_{\substack{C}}$$
\end{definicija}
%povej, da to velja za vsak j=1,...,n
%popravi za vrednosti funkcij
%$\alpha_j$, $\beta_j$ stične funkcije\\
%$\alpha_1(y)\neq 0$ in potrebno izbrati pravi predznak $\alpha_1$ %??
\end{frame}

\section{$G^1$ zveznost}

\begin{frame}
\begin{flalign*}
&R(x_0,y)=S(u_0,y)&\\ %povej da Ry = Sy vzdolž C!
&R_x(x_0,y)=u_x(x_0,y)S_u(u_0,y)+v_x(x_0,y)S_v(u_0,y)\\ %oziroma kot smo prej dobili...obstajata alfa1 in beta1, kjer alfa1 ni 0 in ustreznega predznaka
&R_x(x_0,y)=\alpha_1(y) S_u(u_0,y)+\beta_1(y)S_v(u_0,y)\\ %kako to geometrijsko interpretiramo: vidimo, da so Rx, Su, Sv linearno odvisni v vsaki točki y tj. v vsaki točki del iste tangentne ravnine
\end{flalign*}
%vidimo, da se na skupnem robu ujemata enotski normali
%G1 zveznosti rečemo tudi zveznost tangentnih ravnin oz zveznost enotskih normal
\end{frame}

\begin{frame}
\begin{flalign*}
%&R(x_0,y)=S(u_0,y)&\\ %povej da Ry = Sy vzdolž C!
%&R_x(x_0,y)=u_x(y)S_u(u_0,y)+v_x(y)S_v(u_0,y)\\ %oziroma kot smo prej dobili...obstajata alfa1 in beta1, kjer alfa1 ni 0 in ustreznega predznaka
&R_x(x_0,y)=\alpha_1(y) S_u(u_0,y)+\beta_1(y)S_v(u_0,y)\\ %kako to geometrijsko interpretiramo: vidimo, da so Rx, Su, Sv linearno odvisni v vsaki točki y tj. v vsaki točki del iste tangentne ravnine
%če enačbo vektorsko pomnožimo z Ry(x0,y)
&R_x(x_0,y) \times R_y(x_0,y) = \alpha_1(y) S_u(u_0,y) \times S_v(u_0,y)\\
&\frac{R_x(x_0,y) \times R_y(x_0,y)}{||R_x(x_0,y) \times R_y(x_0,y)||}=\frac{S_u(u_0,y) \times S_v(u_0,y)}{||S_u(u_0,y) \times S_v(u_0,y)||}
\end{flalign*}
%vidimo, da se na skupnem robu ujemata enotski normali
%G1 zveznosti rečemo tudi zveznost tangentnih ravnin oz zveznost enotskih normal
\end{frame}

\begin{frame}
\begin{flalign*}
&R_x(x_0,y)=\alpha_1(y) S_u(u_0,y)+\beta_1(y)S_v(u_0,y)&\\
&ekvivalentno: \det(R_x(x_0,y), S_u(u_0,y), S_v(u_0,y))=0\\
&\Longrightarrow \exists\textrm{ funkcije }\lambda, \mu, \gamma:\\
&\lambda(y)R_x(x_0,y)=\mu(y)S_u(u_0,y)+\gamma(y)S_v(u_0,y)
\end{flalign*}
%povej: če sta R, S polinomski, lahko tudi lambda, mi, gama izberemo polinomske
\end{frame}

%pogledamo si poseben primer polinomskih param ploskev, ki so tudi uporabne v praksi
\section{Bézierjeve ploskve}

\begin{frame}
$i$-ti Bernsteinov bazni polinom stopnje $n$
$$B_i^n(t)={n \choose i}t^i (1-t)^{n-i}\textrm{, } t\in[0,1]$$
Lastnosti:
\begin{itemize}
\item $B_i^n(0) =\delta_{i,0}$
\item $B_i^n(1)=\delta_{i,n}$
%tvorijo bazo prostora polinomov Pn
\end{itemize}
\end{frame}

\begin{frame}
\begin{definicija}
Naj bodo dane točke $\mathbf{b}_{i,j}\in \R^3$, $i=0,1,\ldots,m$, $j=0,1,\ldots,n$. Bézierjeva ploskev iz tenzorskega produkta je parametrično podana ploskev
$$\mathbf{b}^{m,n} : [0,1]\times[0,1] \rightarrow \R^3$$
s predpisom
$$\mathbf{b}^{m,n}(u,v)=\sum_{i=0}^m \sum_{j=0}^n \mathbf{b_{i,j}} B_i^m(u) B_j^n(v).$$
Točke $\mathbf{b}_{i,j}$ imenujemo kontrolne točke, poligon, ki jih povezuje, pa kontrolni poligon.
\end{definicija}
Velja: $\mathbf{b^{m,n}}(0,0)=\mathbf{b}_{0,0}$, $\mathbf{b^{m,n}}(1,0)=\mathbf{b}_{m,0}$, $\mathbf{b^{m,n}}(0,1)=\mathbf{b}_{0,n}$, $\mathbf{b^{m,n}}(1,1)=\mathbf{b}_{m,n}$
%interpolacija robnih točk
%vstavi sliko
\end{frame}

%\begin{frame}
%\includegraphics[scale=0.5]{tenzorski}
%\end{frame}

\begin{frame}
Odvod Bézierjeve ploskve iz tenzorskega produkta:\\
$$\frac{\partial^{r+s}}{\partial u^r \partial v^s}\mathbf{b}^{m,n}(u,v)=\frac{m!}{(m-r)!}\frac{n!}{(n-s)!}\sum_{i=0}^{m-r} \sum_{j=0}^{n-s} \Delta^{r,s}\mathbf{b}_{i,j}B_i^{m-r}(u)B_j^{n-s}(v),$$
kjer $\Delta^{1,0} \mathbf{b}_{i,j} = \mathbf{b}_{i+1,j}-\mathbf{b}_{i,j}$,\\
$\Delta^{0,1} \mathbf{b}_{i,j} = \mathbf{b}_{i,j+1}-\mathbf{b}_{i,j}$,\\
$\Delta^{r,0} \mathbf{b}_{i,j} = \Delta^{r-1,0} \mathbf{b}_{i+1,j}-\Delta^{r-1,0} \mathbf{b}_{i,j}$,\\
$\Delta^{0,s} \mathbf{b}_{i,j} = \Delta^{0,s-1} \mathbf{b}_{i,j+1}-\Delta^{0,s-1} \mathbf{b}_{i,j}$.
%lahko poveš, da so to v bistvu vektorji
%posebej lahko napišeš (ali dodaš) odvode 1. stopnje
%dodaj pogoje gladkosti za C1 ?
%dodaj trikotne krpe??
\end{frame}

\begin{frame}
\begin{definicija}
Naj bodo $\mathbf{a}$, $\mathbf{b}$ in $\mathbf{c}$ krajišča trikotnika $T$ in naj bo $\mathbf{p}$  točka v $T$. Koeficiente $\mathbf{u}=(u,v,w)$, za katere velja $\mathbf{p}=u\mathbf{a}+ v \mathbf{b} + w \mathbf{c}$ in $u+v+w$ imenujemo baricentrične koordinate točke $\mathbf{p}$ glede na $T$.
\end{definicija}
Pišemo: $\mathbf{u}=Bar(\mathbf{p},T)$
\end{frame}

\begin{frame}
Bernsteinov bazni polinom stopnje $n$ dveh spremenljivk
$$B_{i,j,k}^n(\mathbf{u}) = \frac{n!}{i! j! k!}u^i v^j w^k,$$
kjer $\mathbf{u}=(u,v,w)$ in $u+v+w=1$
\end{frame}

\begin{frame}
\begin{definicija}
Naj bodo dane točke $\mathbf{b}_{i,j,k}\in \R^3$, $i+j+k=n$. Trikotna Bézierjeva ploskev je parametrično podana ploskev
$$\mathbf{b}_n : T \rightarrow \R^3$$
s predpisom
$$\mathbf{b}_n(u,v,w)=\sum_{i+j+k=n}\mathbf{b_{i,j,k}} B_{i,j,k}^n(u,v,w).$$
Točke $\mathbf{b}_{i,j,k}$ imenujemo kontrolne točke, poligon, ki jih povezuje, pa kontrolni poligon.
\end{definicija}
Velja: $\mathbf{b_n}(0,0,1)=\mathbf{b}_{0,0,n}$, $\mathbf{b_n}(0,1,0)=\mathbf{b}_{0,n,0}$, $\mathbf{b_n}(1,0,0)=\mathbf{b}_{n,0,0}$
\end{frame}

%\begin{frame}
%\includegraphics[scale=0.4]{trikotna.jpg}
%\end{frame}

\begin{frame}
Odvod trikotne Bézierjeve ploskve v smeri vektorja $\mathbf{d}$:\\
$$D_{\mathbf{d}}\mathbf{b}_n(\mathbf{u})=n \sum_{i+j+k=n-1}(\mu_1 \mathbf{b}_{i+1,j,k} + \mu_2 \mathbf{b}_{i,j+1,k} + \mu_3 \mathbf{b}_{i,j,k+1})B_{i,j,k}^{n-1}(\mathbf{u}),$$
kjer je $\mathbf{d}=\mu_1 \mathbf{a} + \mu_2 \mathbf{b} + \mu_3 \mathbf{c}$ in $\mu_1 + \mu_2 + \mu_3 =0$
%povej, da so to baricentrične koordinate  vektorja d
%odvod na robu?
%vmesne točke de casteljaujevega algoritma?
%tangenta na točko
\end{frame}

\section{Primeri konstrukcij $G^1$ ploskev}
\begin{frame}
Dve bikubični Bézierjevi ploskvi iz tenzorskega produkta:
$$R(x,y)=\sum_{i=0}^3 \sum_{j=0}^3 \mathbf{P}_{i,j}B_i^3(x) B_j^3(y)$$ in
$$S(u,v)=\sum_{i=0}^3 \sum_{j=0}^3 \mathbf{Q}_{i,j}B_i^3(u) B_j^3(v),$$
ki se stikata v $C(v)=R(0,v)=S(0,v)$.
\end{frame}

%\begin{frame}
%\begin{figure}
%\includegraphics[scale=0.5]{primer1}
%\end{figure}
%\end{frame}

\begin{frame}
\begin{flalign*}
&\sum_{j=0}^3(Q_{1,j}-Q_{0,j})B_j^3(v) =\\
&=\alpha(v) \sum_{j=0}^3(P_{1,j}-P_{0,j})B_j^3(v) +\beta(v) \sum_{j=0}^2(P_{0,j+1}-P_{0,j})B_j^2(v)\\
%\vspace{10mm}
%&v=0: \hspace{5mm} \mathbf{q}_0 = \alpha_0 \mathbf{p}_0 + \beta_0 \mathbf{s}_0\\
%&v=1: \hspace{5mm} \mathbf{q}_3 = \alpha_1 \mathbf{p}_3 + \beta_1 \mathbf{s}_2
\end{flalign*}
\end{frame}

\begin{frame}
\begin{flalign*}
&\sum_{j=0}^3(Q_{1,j}-Q_{0,j})B_j^3(v) =\\
&=\alpha(v) \sum_{j=0}^3(P_{1,j}-P_{0,j})B_j^3(v) +\beta(v) \sum_{j=0}^3(P_{0,j+1}-P_{0,j})B_j^3(v)\\
\vspace{10mm}
&v=0: \hspace{5mm} \mathbf{q}_0 = \alpha_0 \mathbf{p}_0 + \beta_0 \mathbf{s}_0\\
&v=1: \hspace{5mm} \mathbf{q}_3 = \alpha_1 \mathbf{p}_3 + \beta_1 \mathbf{s}_2\\
\end{flalign*}
\end{frame}

\begin{frame}
\begin{flalign*}
&\sum_{j=0}^3(Q_{1,j}-Q_{0,j})B_j^3(v) =\\
&=\alpha(v) \sum_{j=0}^3(P_{1,j}-P_{0,j})B_j^3(v) +\beta(v) \sum_{j=0}^3(P_{0,j+1}-P_{0,j})B_j^3(v)\\
\vspace{10mm}
&v=0: \hspace{5mm} \mathbf{q}_0 = \alpha_0 \mathbf{p}_0 + \beta_0 \mathbf{s}_0\\
&v=1: \hspace{5mm} \mathbf{q}_3 = \alpha_1 \mathbf{p}_3 + \beta_1 \mathbf{s}_2\\
&\alpha_0 = \frac{|\mathbf{q}_0 \times \mathbf{s}_0|}{|\mathbf{p}_0 \times \mathbf{s}_0|}\hspace{20mm} \alpha_1 = \frac{|\mathbf{s}_2 \times \mathbf{q}_3|}{|\mathbf{s}_2 \times \mathbf{p}_3|}\\
&\beta_0 = \frac{|\mathbf{p}_0 \times \mathbf{q}_0|}{|\mathbf{p}_0 \times \mathbf{s}_0|}\hspace{20mm} \beta_1 = \frac{|\mathbf{p}_3 \times \mathbf{q}_3|}{|\mathbf{p}_3 \times \mathbf{s}_2|}
\end{flalign*}
\end{frame}

%\begin{frame}
%\begin{flalign*}
%&v=0: \hspace{5mm} \mathbf{q}_0 = \alpha_0 \mathbf{p}_0 + \beta_0 \mathbf{s}_0\\
%&v=1: \hspace{5mm} \mathbf{q}_3 = \alpha_1 \mathbf{p}_3 + \beta_1 \mathbf{s}_2\\
%&\alpha_0 = \frac{|\mathbf{q}_0 \times \mathbf{s}_0|}{|\mathbf{p}_0 \times \mathbf{s}_0|}\hspace{20mm} \alpha_1 = \frac{|\mathbf{s}_2 \times \mathbf{q}_3|}{|\mathbf{s}_2 \times \mathbf{p}_3|}\\
%&\beta_0 = \frac{|\mathbf{p}_0 \times \mathbf{q}_0|}{|\mathbf{p}_0 \times \mathbf{s}_0|}\hspace{20mm} \beta_1 = \frac{|\mathbf{p}_3 \times \mathbf{q}_3|}{|\mathbf{p}_3 \times \mathbf{s}_2|}
%\end{flalign*}
%\end{frame}

\begin{frame}
Primer 1:\\
Znano: robovi ploskev, $\mathbf{p}_1$, $\mathbf{p}_2$\\
Iščemo: $\mathbf{q}_1$, $\mathbf{q}_2$
\end{frame}

\begin{frame}
Primer 1:\\
Znano: robovi ploskev, $\mathbf{p}_1$, $\mathbf{p}_2$\\
Iščemo: $\mathbf{q}_1$, $\mathbf{q}_2$\\
\vspace{5mm}
$\mathbf{q}_1= \alpha_0 \mathbf{p}_1+\frac{1}{3}\beta_1\mathbf{s}_0+\frac{2}{3}\beta_0\mathbf{s}_1$\\
$\mathbf{q}_2= \alpha_0 \mathbf{p}_2+\frac{2}{3}\beta_1\mathbf{s}_1+\frac{1}{3}\beta_0\mathbf{s}_2$
\end{frame}

\begin{frame}
Primer 2:\\
Znano: robovi ploskev, $\mathbf{p}_m$, krivulja $C$ je kvadratična\\
Iščemo: $\mathbf{p}_1$, $\mathbf{p}_2$, $\mathbf{q}_1$, $\mathbf{q}_2$
\end{frame}

%\begin{frame}
%\begin{figure}
%\includegraphics[scale=0.5]{primer2}
%\end{figure}
%\end{frame}

\begin{frame}
Primer 2:\\
Znano: robovi ploskev, $\mathbf{p}_m$, $R_x(0,v)$ je stopnje 2\\
Iščemo: $\mathbf{p}_1$, $\mathbf{p}_2$, $\mathbf{q}_1$, $\mathbf{q}_2$\\
\vspace{5mm}
$\mathbf{p}_1=\frac{2}{3}\mathbf{p}_m+\frac{1}{3}\mathbf{p}_0$\\
 $\mathbf{p}_2=\frac{2}{3}\mathbf{p}_m+\frac{1}{3}\mathbf{p}_3$\\
$\mathbf{q}_1 = \frac{1}{3}\alpha_1 \mathbf{p}_0+\alpha_0 \mathbf{p}_1 - \frac{1}{3}\alpha_0 \mathbf{p}_0 + \frac{1}{3}\beta_1\mathbf{s}_0 +\frac{2}{3}\beta_0 \mathbf{s}_1$\\
$\mathbf{q}_1 = \frac{1}{3}\alpha_0 \mathbf{p}_3+\alpha_1 \mathbf{p}_2 - \frac{1}{3}\alpha_1 \mathbf{p}_3 + \frac{2}{3}\beta_1\mathbf{s}_1 +\frac{1}{3}\beta_0 \mathbf{s}_2$
\end{frame}

\begin{frame}
Dve trikotni Bézierjevi ploskvi stopnje $n$:
$$R(\mathbf{u})=\sum_{i+j+k=n} \mathbf{a}_{i,j,k}B_{i,j,k}^n(\mathbf{u})$$ in
$$S(\mathbf{u})=\sum_{i+j+k=n}\mathbf{b}_{i,j,k}B_{i,j,k}^n(\mathbf{u}),$$
ki se stikata v $C(u)=R(u,0)=S(u,0)$.
\end{frame}

\begin{frame}
\begin{flalign*}
&\mathbf{a}_{0,1,0}^{n-1}(u)=\sum_{i=0}^{n-1}\mathbf{a}_{i,1,n-i-1} B_i^{n-1}(u)\\
&\mathbf{b}_{0,1,0}^{n-1}(u)=\sum_{i=0}^{n-1}\mathbf{b}_{i,1,n-i-1} B_i^{n-1}(u)\\
&\mathbf{b}_{0,0,1}^{n-1}(u)=\sum_{i=0}^{n-1}\mathbf{b}_{i,0,n-i} B_i^{n-1}(u)\\
&\mathbf{b}_{1,0,0}^{n-1}(u)=\sum_{i=0}^{n-1}\mathbf{b}_{i+1,0,n-i-1} B_i^{n-1}(u)
\end{flalign*}
\end{frame}

\begin{frame}
$\sum_{i=0}^{n}\frac{n-i}{n}((1-\lambda_0)\mathbf{p}_i+\lambda_0\mathbf{r}_i)B_i^{n}(u) + \sum_{i=0}^{n}\frac{i}{n}((1-\lambda_1)\mathbf{p}_{i-1}+\lambda_1\mathbf{r}_{i-1})B_{i}^{n}(u)=$\\
$=\sum_{i=0}^{n}\frac{n-i}{n}((1-\mu_0)\mathbf{q}_i+\mu_0\mathbf{q}_{i+1})B_i^{n}(u)+\sum_{i=0}^{n}\frac{i}{n}((1-\mu_1)\mathbf{q}_{i-1}+\mu_1\mathbf{q}_{i})B_{i}^{n}(u)$

\vspace{5mm}
\begin{flalign*}
&\frac{n-i}{n}((1-\lambda_0)\mathbf{p}_i+\lambda_0\mathbf{r}_i)+\frac{i}{n}((1-\lambda_1)\mathbf{p}_{i-1}+\lambda_1\mathbf{r}_{i-1})=&\\
&=\frac{n-i}{n}((1-\mu_0)\mathbf{q}_i+\mu_0\mathbf{q}_{i+1})+\frac{i}{n}((1-\mu_1)\mathbf{q}_{i-1}+\mu_1\mathbf{q}_{i})
\end{flalign*}
\end{frame}

\end{document}

%\begin{frame}
%\begin{izrek}
%Naj bosta $R(x,y)$ in $S(u,v)$ regularni $C^n$ parametrizaciji dveh ploskev, ki se stikata v krivulji $C(y)=R(x_0,y)=S(u_0,y)$. Ploskvi $R$ in $S$ sta $G^n$-zvezni vzdolž skupnega roba natanko tedaj ko obstajajo funkcije $p_i(v)$, $q_i(v)$, $i=1,\ldots, n$, da velja
%\begin{align*}
%\frac{\partial^k S}{\partial u_s^k}\Bigr|_{\substack{C}}=& \sum_{i=1}^k\sum_{|\mathbf{m_i}|=k} A_{\mathbf{m_i}}^k \sum_{h=0}^i {i \choose h} p_{m_1}(v) \cdots p_{m_h}(v) q_{m_{h+1}}(v) \cdots q_{m_i} \\
%&\cdot\frac{\partial^i R}{\partial u_r^h \partial v_r^{i-h}},
%\end{align*}
%kjer $k=1,\ldots,n$ ter\\
%$\mathbf{m_i} = (m1,m2,\ldots,m_i)$, $|\mathbf{m_i}|=m_1+m_2+\cdots+m_i$ in $A_{\mathbf{m_i}}^k = \frac{k!}{i!m_1!\cdots m_i!}$.
%\end{izrek}
%\end{frame}

%\begin{frame}
%\begin{lema}
%Naj bo $f(u,v)$ funkcija razreda $C^n$ in $u(t)$ in $v(t)$ reparametrizaciji razreda $C^n$. Potem
%\begin{align*}
%\frac{d^k f}{dt^k}=& \sum_{i=1}^k\sum_{|\mathbf{m_i}|=k} A_{\mathbf{m_i}}^k \sum_{h=0}^i {i \choose h} u^{(m_1)}\cdots u^{(m_h)}(v) v^{(m_{h+1})}\cdots v^{(m_i)} \\
%&\cdot\frac{\partial^i f}{\partial u_r^h \partial v_r^{i-h}},
%\end{align*}
%kjer $k=1,\ldots,n$ ter\\
%$\mathbf{m_i} = (m1,m2,\ldots,m_i)$, $|\mathbf{m_i}|=m_1+m_2+\cdots+m_i$ in $A_{\mathbf{m_i}}^k = \frac{k!}{i!m_1!\cdots m_i!}$.
%\end{lema}
%\end{frame}

%kako to izgleda za n=1 (n=2?)
%geometrijska interpretacija